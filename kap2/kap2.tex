\chapter{Reihen}

\begin{Definition}
Eine Reihe ist eine Folge, deren Glieder die Partialsummen einer anderen Folge ist. Das bedeutet, dass das $n-te$ Glied der Reihe, die Summe der ersten $n$ Glieder einer anderen Folge ist. \\
Man hat also:

\begin{enumerate}
\item Mit Startglied $a_{0}$: $s_{n}=\sum\limits_{i=0}^{n-1}a_{i}$
\item Mit Startglied $a_{1}$: $s_{n}=\sum\limits_{i=1}^{n}a_{i}$
\item Mit Startglied $a_{x}$: $s_{n}=\sum\limits_{i=x}^{x+n-1}a_{i}$
\end{enumerate}

\end{Definition}

\begin{Bemerkung}
In manchen Fällen steht $s_{n}$ für die Partialsumme einer anderen Folge bis zum $n-ten$ Glied.
Dann gilt für ein beliebiges Startglied $a_{x}$ der Folge: $s_{n}=\sum\limits_{i=x}^{n}a_{i}$
\end{Bemerkung}

		\section{Artithmetische Reihen}

	\subsection{Gauß'sche Summenformel}

Die Gauß'sche Summenformel bezeichnet die Summe der $n$ ersten natürlichen Zahlen, also:\\
$1+2+3+...+n=\sum\limits_{k=1}^{n}k=\frac{n(n+1)}{2}$\\\\

\begin{minipage}[c]{0.5\textwidth}
	\underline{Begründung:}\\
	\begin{tabular}{c∣c∣c∣c∣c∣c∣c}
		&1&2&3&4&...&n\\\hline
		&n&n-1&n-2&n-3&...&1\\\hline
		\sum&n+1&n+1&n+1&n+1&...&n+1\\
	\end{tabular}
\end{minipage}
\begin{minipage}{0.5\textwidth}
So sieht man also, dass wenn man die vorher bestimmte Reihe mit sich selbst addiert (ein Mal davon "falschrum"), man $n$ Mal $n+1$ bekommt. Um dann den Wert einer einzelnen Reihe zu bekommen teilt man durch zwei.
\end{minipage}

	\subsection{Allgemein}

\begin{Definition}
Wenn $s_{n}$ die Summe der ersten $n$ Folgeglieder einer arithmetische Folge ist, heißt sie arithmetische Reihe.\\
Sei eine arithmetische Folge $a$ mit Startglied $a_{x}$ und $s$, die entsprechende Reihe, dann gilt\\
$s_{n}=$
\end{Definition}

\begin{Beweis}
\end{Beweis}

		\section{Geometrische Reihen}

\begin{Definition}
Wenn $s_{n}$ die Summe der ersten $n$ Folgeglieder einer geometrischen Folge ist, heißt sie geometrischen Reihe.\\
Sei eine geometrische Folge $a$ mit Startglied $a_{x}$ und $s$, die entsprechende Reihe, dann gilt\\
$s_{n}=\sum\limits_{i=x}^{n+x-1}a_{i}=a_{x}\cdot \dfrac{1-q^n}{1-q}$
\end{Definition}

\begin{Beweis}
\underline{Allgemein:}\\
\\
$(1-q)(1+q+q^2+q^3+...+q^n)\\
=(1\textcolor{red}{-q)+(q}\textcolor{orange}{-q^2)+(q^2}\textcolor{green}{-q^3)+(q^3}\textcolor{brown}{-q^4)+.}.\textcolor{blue}{.+(q^n}-q^{n+1})\\
=1+(\textcolor{red}{-q+q})+(\textcolor{orange}{-q^2+q^2})+(\textcolor{green}{-q^3+q^3})+...+(\textcolor{blue}{-q^n+q^n})-q^{n+1}\\
=1-q^{n+1}$\\
\\
Man hat also $\sum\limits_{k=0}^nq^k=1+q+q^2+q^3+...+q^n=\dfrac{1-q^{n+1}}{1-q}$\\
Entsprechend ergibt sich $\sum\limits_{k=0}^{n-1}q^k=\underbrace{1+q+q^2+q^3+...+q^{n-1}}_{n\quad Summanden}=\dfrac{1-q^n}{1-q}$\\
\\
Somit gilt für eine Reihe $s$, die die Partialsumme einer geometrischen Folge $a$, mit Quotient $q$ und Anfangsglied $a_{x}$, ist, folgendes:\\
$s_{n}=\sum\limits_{i=x}^{x+n-1}a_{i}\\
=a_{x}+a_{x+1}+a_{x+2}+...+a_{x+n-1}\\
=a_{x}+a_{x}\cdot q+a_{x}\cdot q^{2}+...+a_{x}\cdot q^{n-1}\\
=a_{x}\cdot(1+q+q^2+...+q^{n-1})\\
=a_{x}\cdot \sum\limits_{k=0}^{n-1}q^k\\
=a_{x}\cdot \dfrac{1-q^{n}}{1-q}$
\end{Beweis}

