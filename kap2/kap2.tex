\chapter{Reihen}

\begin{Definition}
Eine Reihe ist eine Folge, deren Glieder die Partialsummen einer anderen Folge ist. Das bedeutet, dass das $n-te$ Glied der Reihe, die Summe der ersten $n$ Glieder einer anderen Folge ist. \\
Man hat also:

\begin{enumerate} 
\item Mit Startglied $a_{0}$: $s_{n}=\sum\limits_{i=0}^{n-1}a_{i}$
\item Mit Startglied $a_{1}$: $s_{n}=\sum\limits_{i=1}^{n}a_{i}$
\item Mit Startglied $a_{x}$: $s_{n}=\sum\limits_{i=x}^{n+x-1}a_{i}$
\end{enumerate}

\end{Definition}

\begin{Bemerkung}
In manchen Fällen steht $s_{n}$ für die Partialsumme einer anderen Folge bis zum $n-ten$ Glied.
Dann gilt für ein beliebiges Startglied $a_{x}$ der Folge: $s_{n}=\sum\limits_{i=x}^{n}a_{i}$
\end{Bemerkung}

		\section{Artithmetische Reihen}

\begin{Definition}
Wenn $s_{n}$ die Summe der ersten $n$ Folgeglieder einer arithmetischen Folge ist, heißt sie arithmetische Reihe.
\end{Definition}

	\subsection{Gauß'sche Summenformel}

Die Gauß'sche Summenformel bezeichnet die Summe der $n$ ersten natürlichen Zahlen, also:\\
$1+2+3+...+n=\sum\limits_{k=1}^{n}k=\frac{n(n+1)}{2}$\\

\begin{minipage}{0.5*\linewidth}
\underline{Begründung:}\\
\begin{tabular}{c∣c∣c∣c∣c∣c∣c}
&1&2&3&4&...&n\\
\hline
&n&n-1&n-2&n-3&...&1\\
\hline
\sum&n+1&n+1&n+1&n+1&...&n+1\\
\end{tabular}
\end{minipage}
\hfill
\begin{minipage}{0.5*\linewidth}
So sieht man also, dass wenn man die vorher bestimmte Reihe mit sich selbst addiert (ein Mal davon "falschrum"), man $n$ Mal $n+1$ bekommt. Um dann den Wert einer einzelnen Reihe zu bekommen teilt man durch zwei.
\end{minipage}

		\section{Geometrische Reihen}

\begin{Definition}
Wenn $s_{n}$ die Summe der ersten $n$ Folgeglieder einer geometrischen Folge ist, heißt sie geometrischen Reihe.
\end{Definition}

