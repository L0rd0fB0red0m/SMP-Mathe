\chapter{Exponentialfunktionen}

\begin{Definition}
Man bezeichnet als Exponentialfunktion eine Funktion der Form $x\rightarrow a^x$ mit $a\in \R^+\backslash 1$\\
$x$ ist die Variable und wird \textit{Exponent} oder \textit{Hochzahl} genannt.\\
$a$ nennt man \textit{Basis} oder \textit{Grundzahl}, sie ist für jede Funktion fest forgegeben.\\
Die natürliche Exponentialfunktion wird durch die Funktionsvorschrift $f(x)=e^x$ beschrieben.
\end{Definition}

Hier Graphen für a<1\\
a>1\\
a=e\\


		\section{Wiederholung: Potenzgesetze}
Seien $a$, $b\in\R$, sowie $n$, $m\in\N$, dann gilt:
\begin{enumerate}
\begin{minipage}{0.5\textwidth}
\item$a^0=1$ (für $a\neq 0$)
\item$a^1=a$
\item$a^n=\underbrace{a\cdot a\cdot ... \cdot a}_{n Mal}$
\item$a^m\cdot a^n=a^{m+n}$
\item$(a^n)^m=a^{n\cdot m}$
\end {minipage}
\begin{minipage}{0.5\textwidth}
\item$a^{-n}=\dfrac{1}{a^n}$
\item$\dfrac{a^n}{a^m}=a^{n-m}$
\item$\left(\dfrac{a}{b}\right)^n=\dfrac{a^n}{b^n}$
\item $a^{\frac{1}{n}}=\sqrt[n]{a}$
\item$a^{\frac{m}{n}}=\sqrt[n]{a^m}$
\end {minipage}
\end{enumerate}

		\section{Die Eulersche Zahl ($e$)}

\begin{Definition}
$e = 2,71828182845904523536028747135266249775724709369995...$\\
\\
$e$ ist eine irrationale, transzendente und reelle Zahl, die die Basis des (natürlichen) Logarithmus und der (natürlichen) Exponentialfunktion ist.\\
Die Darstellung, der man am Häufigsten begegnet ist diese:
$e=\lim\limits_{t\to \infty}\left(1+\dfrac{1}{t}\right)^t;t\in\R$
\end{Definition}
\\
Benannt nach dem bekannten Mathematiker Leonhard Euler ist diese Zahl eine der wichtigsten Konstanten der Mathematik.
Sie ist die Basis des natürlichen Logarithmus und der natürlichen Exponentialfunktion. Diese (spezielle) Exponentialfunktion wird aufgrund dieser Beziehung zur Zahl $e$ häufig kurz $e$-Funktion genannt.\\
\begin{Definition}
  Eine reelle Zahl heißt (oder allgemeiner eine komplexe Zahl) transzendent,
  wenn sie nicht Nullstelle eines Polynoms mit ganzzahligen Koeffizienten ist.
  Andernfalls handelt es sich um eine algebraische Zahl. Jede reelle transzendente Zahl ist überdies irrational.
\end{Definition}

	\subsection{Verschiedene Darstellungen}

$e$ ist darstellbar bzw. ergibt sich durch:
\begin{itemize}
\item $\sum\limits_{k=0}^{\infty}\dfrac{1}{k!}$
\item $\lim\limits_{t\to \infty}\left(1+\dfrac{1}{t}\right)^t;t\in\R$
\item $\lim\limits_{n\to \infty}\left(1+\dfrac{1}{n}\right)^n;n\in\N$
\end{itemize}

	\subsection{Herleitung zur Zahl $e$}

Wir definieren eine Folge $(e_{n})n\in\N$ durch $e_{n}=\left(1+\dfrac{1}{n}\right)^n$ und beweisen ihre Konvergenz.

\begin{itemize}
\item\begin{align*}\dfrac{e_{n+1}}{e_{n}}&=\dfrac{{\left(1+\dfrac{1}{n+1}\right)}^{n+1}}{{\left(1+\dfrac{1}{n}\right)}^{n}}\\
&=\left(\dfrac{n(n+2)}{(n+1)^2}\right)^{n+1}\cdot\left(1+\dfrac{1}{n}\right)\\
&=\textcolor{blue}{\left(1-\dfrac{1}{(n+1)^2}\right)^{n+1}}\cdot\textcolor{red}{\dfrac{n+1}{n}}
\end{align*}
\item Die Umformung ermöglicht uns auf den Term $\left(1-\dfrac{1}{(n+1)^2}\right)^{n+1}$ die Ungleichung von Bernoulli anzuwenden. Diese besagt Folgendes: $(1+x)^n>1+nx$ für $n\geq2$ und $x>-1$\\
\begin{align*}\Rightarrow\textcolor{blue}{\left(1-\dfrac{1}{(n+1)^2}\right)^{n+1}}&\textcolor{blue}{  >} 1+(n+1)\cdot\left(-\dfrac{1}{(n+1)^2}\right)\\
&=1-\dfrac{1}{n+1}\\
&=\textcolor{blue}{\dfrac{n}{n+1}}
\end{align*}
\item Dies kann man an den ersten Ausdruck anwenden:\\
\begin{align*}
\Rightarrow\dfrac{e_{n+1}}{e_{n}}&=\left(1-\dfrac{1}{(n+1)^2}\right)^{n+1}\cdot\dfrac{n+1}{n}\\
&\textcolor{blue} {>\dfrac{n}{n+1}}\cdot\textcolor{red}{\dfrac{n+1}{n}}\\
&=1
\end{align*}
\textbf{$\Leftrightarrow e_{n}$ ist streng monoton steigend}
\end{itemize}
Sei eine Folge $f_{n}$ mit $f_{n}=\left(1+\dfrac{1}{n}\right)^{n+2}$, deren Monotonieverhalten wir untersuhen wollen.\\
Dazu formen wir den Term so um, dass die Bernoulli-Ungleichung angewandt werden kann.
\begin{itemize}
\item\begin{align*}\dfrac{f_{n}}{f_{n+1}}&=\dfrac{\left(1+\dfrac{1}{n}\right)^{n+1}}{\left(1+\dfrac{1}{n}\right)^{n+2}}\\
&=\left(\dfrac{1+\dfrac{1}{n}}{1+\dfrac{1}{n+1}}\right)^{n+2}\cdot\dfrac{1}{1+\dfrac{1}{n}}\\
&=\left(\dfrac{\dfrac{n+1}{n}}{\dfrac{n+2}{n+1}}\right)\cdot\dfrac{n}{n+1}\\
&=\left(\dfrac{(n+1)^2}{n(n+2)}\right)^{n+2}\cdot\dfrac{n}{n+1}\\
&=\left(\dfrac{n^2+2n+1}{n^2+2n}\right)^{n+2}\cdot\dfrac{n}{n+1}\\
&=\textcolor{blue}{\left(1+\dfrac{1}{n(n+2)}\right)^{n+2}}\cdot\textcolor{red}{\dfrac{n}{n+1}}
\end{align*}
\item Bernoulli: $(1+x)^n>1+nx$ für $n\geq2$ und $x>-1$\\
\begin{align*}\Rightarrow \textcolor{blue}{\left(1+\dfrac{1}{n(n+2)}\right)^{n+2} &\textcolor{blue}{>}1+(n+2)\cdot\dfrac{1}{n(n+2)}\\
&=\textcolor{blue}{\dfrac{n+1}{n}}
\end{align*}
\item$\Rightarrow\dfrac{f_n}{f_{n+1}}=\left(1+\dfrac{1}{n(n+2)}\right)^{n+2}\cdot\dfrac{n}{n+1}>\textcolor{blue}{\dfrac{n+1}{n}}\cdot\textcolor{red}{\dfrac{n}{n+1}}=1$
\item $\dfrac{f_n}{f_{n+1}}>1\Leftrightarrow\dfrac{f_{n+1}}{f_{n}}<1$\textbf{$\Leftrightarrow f_n$ ist streng monoton fallend}
\end{itemize}
Als Letztes zeigt man, dass $f_n$ zu $e_n$ eine obere Schranke ist:\\
$f_n>e_n\\
\left(1+\dfrac{1}{n}\right)^{n+1}>\left(1\dfrac{1}{n}\right)^n$

Eine streng monoton steigende Folge, die eine monoton fallende Folge, als ober Schranke hat, konvergiert. Es gilt also, dass $e_n$ konvergent ist, $e_n$ besitzt einen besonderen Grenzwert, der \textit{Eulersche Zahl} genannt wird.

	\subsection{$e$ ist irrational}

Wie schon erwähnt, handelt es sich bei der Eulerschen Zahl um eine transzendente Zahl, daraus ergibt sich, dass sie ebenfalls irrational ist. Trotzdem ist es interessant, dies zu beweisen.

\begin{Beweis}
\underline{Teil A:} unnötig\\\\
Sei $f$ eine Funktion mit $f(x)=xe^{1-x}$ mit Schaubild $\mathcal{C}$\\
\begin{enumerate}
\item Monotonieverhalten:\\
\begin{align*}
f\prime(x)&=e^{1-x}-xe^{1-x}\\
&=(1-x)\cdot e^{1-x}
\end{align*}
\\
\begin{tabular}{rlcl}
$\forall x<1$: &$f\prime(x)>0&&\Leftrightarrow f \nearrow$\\
für $x=1$: &$f\prime(x)=0$&und VZW $+\rightarrow-&\Leftrightarrow$ Hochpunkt\\
$\forall x>1$: &$f\prime(x)<0 &&\Leftrightarrow f \searrow$\\
\end{tabular}
\\

Grenzwerte:\\
\begin{itemize}
\item$\lim\limits_{x\to +\infty}\underbrace{x}_{\rightarrow+\infty}\underbrace{e^{1-x}}_{\rightarrow0}=0$ (Croissance comparée)\\\\
\item$\lim\limits_{x\to -\infty}\underbrace{x}_{\rightarrow-\infty}\underbrace{e^{1-x}}_{\rightarrow+\infty}=-\infty$\\
\end{itemize}
\item Hier Graph von f
\item Man hat $I_1$,  das Integral mit $I_1=\int_0^1f(x)\mbox{d}x$\\
\begin{minipage}{0.5\textwidth}
\begin{align*}
I_1&=\int_0^1xe^{1-x}\mbox{d}x\\
&=\left[-xe^{1-x}\right]_0^1-\int_0^1-e^{1-x}\mbox{d}x\\
&=-1e^{1-1}-0\cdote^1-\left[e^{1-x}\right]_0^1\\
&=-1-1+e\\
&=e-2
\end{align*}
\end{minipage}
\begin{minipage}{0.5\textwidth}
\begin{alignat*}{2}
u(x)&=x \quad& u\prime(x)&=1\\
v'(x)&=e^{1-x}\quad & v(x)&=-e^{1-x}
\end{alignat*}
\end{minipage}
\end{enumerate}
\underline{Teil B:}\\\\
Sei $I_n$ das Integral mit $I_n=\int_0^1x^ne^{1-x}\mbox{d}x$, $n\geq 1$
\begin{enumerate}
\item %1
\begin{enumerate}
\item $\zz$: $\forall x\in[0;1]$ gilt $x^n\leq x^ne^{1-x} \leq ex^n$\\
\begin{center}
\begin{array}{cccccc}
$x^n &\stackrel{?}{\leq}& x^ne^{1-x} &\stackrel{?}{\leq}& ex^n&\qquad|:x^n\\
1        &\stackrel{!}{\leq}&e^{1-x}         &\stackrel{!}{\leq}&e        &$
\end{array}
\end{center}
N.R.: für $x\in[0;1]$ ist $(1-x)\in[0;1]$ und entsprechen $e^{1-x}\in[e^{0};e^{1}]=[1;e]$
\item Sei $J_n$ das Integral mit $J_n=\int_0^1x^n\mbox{d}x$
\begin{align*}
J_n&=\int_0^1x^n\mbox{d}x\\
&=\left[\dfrac{1}{n+1}\cdot x^{n+1}\right]_0^1\\
&=\dfrac{1^{n+1}}{n+1}-\dfrac{0^{n+1}}{n+1}\\
&=\dfrac{1}{n+1}
\end{align*}
\item $\zz$: $\forall n\geq1$ gilt $\dfrac{1}{n+1}\leq I_n \leq \dfrac{e}{n+1}$\\\\
\begin{center}
\begin{array}{ccccccl}
$&x^n&\leq& f(x)=x^ne^{1-x}&\leq& ex^n& \qquad\qquad|\int_0^1()\mbox{d}x\\\\
\Leftrightarrow&\int_0^1 x^n\mbox{d}x &\leq& I_n &\leq& \int_0^1 ex^n\mbox{d}x&\\\\
\Leftrightarrow&J_n &\leq& I_n &\leq& e\cdot\int_0^1x^n\mbox{d}x=e\cdot J_n&\\\\
\Leftrightarrow&\dfrac{1}{n+1}&\leq& I_n &\leq&  \dfrac{e}{n+1}&$
\\\\
\end{array}
\end{center}
\end{enumerate}
\item $\zz$: $\forall n\geq 1$ gilt $I_{n+1}=(n+1)I_n-1$\\
\begin{minipage}{0.55\textwidth}
\begin{align*}
I_{n+1}&=\int_0^1x^{n+1}\cdot e^{1-x}\mbox{d}x\\
&=\left[-e^{1-x}\cdot x^{n+1}\right]_0^1-\int_0^1-(n+1)x^ne{x-1}\mbox{d}x\\
&=-e^{1-1}\cdot 1^{n+1}-0+(n+1)\cdot I_n\\
&=(n+1)I_n-1
\end{align*}
\end{minipage}
\begin{minipage}{0.45\textwidth}
\begin{alignat*}{2}
u(x)&=x^{n+1}\quad& u\prime(x)&=(n+1)x^n\\
v'(x)&=e^{1-x}\quad & v(x)&=-e^{1-x}
\end{alignat*}
\end{minipage}
\item%3
\begin{enumerate}
\item%3a
\item%3b
\item%c
\end{enumerate}
\item%4
\end{enumerate}
\underline{Herkunft der Aufgabenstellung:} Buch? S.199 Nr. 78, d'après le bac (Blatt von Mme Treynard)
\end{Beweis}

\section{Eigenschaften}
		

\begin{itemize}
\item$f(x)=a^x$
\item$a\in\R^+\backslash 1$
\item$D_f=\R$
\item$W_f=\R^+$
\item für $0>a>1$: $f$ ist streng monoton fallend\\
für $a>1$: $f$ ist streng monoton wachsend
\item für $0>a>1$: $\lim\limits_{x\to +\infty}f(x)=-\infty$\\
			$\lim\limits_{x\to -\infty}f(x)=0$\\
für $a>1$: $\lim\limits_{x\to +\infty}f(x)=0$\\
	        $\lim\limits_{x\to -\infty}f(x)=+\infty$
\item $f(0)=1$
\item x-Achse ist waagerechte Asmyptote

\end{itemize}

\subsubsection{Zusammengesetzte Funktionen}

Sei $f(x)= a^x$ mit $a>1$ und $g(x)=x^n$ mit n $\in \R$:\\

\left.\parbox{0.5\textwidth}{
\begin{itemize}
\item $\lim\limits_{x\to +\infty} f(x)\cdot g(x) = \lim\limits_{x\to +\infty} = a^x \cdot x^n= +\infty $ 
\item $\lim\limits_{x\to -\infty} f(x)\cdot g(x) = \lim\limits_{x\to -\infty}a^x \cdot x^n= 0$ \\\\
\item $\lim\limits_{x\to +\infty} \dfrac{f(x)}{g(x)} = \lim\limits_{x\to +\infty}\dfrac{a^x}{x^n} = +\infty $ 
\item $\lim\limits_{x\to -\infty} \dfrac{f(x)}{g(x)} = \lim\limits_{x\to -\infty}\dfrac{a^x}{x^n} = 0$ \\\\
\item $\lim\limits_{x\to +\infty} \dfrac{g(x)}{f(x)} = \lim\limits_{x\to +\infty}\dfrac{x^n}{a^x} =  0$ 
\item $\lim\limits_{x\to -\infty} \dfrac{g(x)}{f(x)} = \lim\limits_{x\to -\infty}\dfrac{x^n}{a^x} = \pm\infty$
\end{itemize}
} \right\}$ \quad \textbf{Croissances comparées}

\begin{Bemerkung}
Diese Grenzwerte gelten auch wenn g ein Polynom beliebig hohen Grades ist.\\
Allgemein kann man also sagen, dass eine Exponentialfunktion ihr Wachstum bezüglich dem jedes Polynoms immer durchsetzt.
\end{Bemerkung}

\begin{Bemerkung}
\danger Das Vorzeichen der anderen Funktion(en) muss natürlich immer beachtet werden.\\
Da die Wertemenge einer Exponentialfunktion immer $\R^+$ ist, hängt das Vorzeichen ausschließlich von der(den) anderen Funktion(en) ab. Wenn der (höchste) Exponent also eine gerade Zahl ist, dann bleiben die Grenzwerte dieselben ($0$ oder $+\infty$). Im Falle eines höchsten Exponenten, der ungerade ist, ändert sich dementsprechend das Vorzeichen ($0$ oder $-\infty$) des Grenzwertes für $x\to -\infty$. Der Grenzwert für $x\to+\infty$ bleibt derselbe, selbst wenn der höchste Exponent eine negative Zahl ist.
\end{Bemerkung}

Für $f(x)= a^x$ mit $0<a<1$ und $g(x)=x^n$ mit n $\in \R$:\\
Es gilt weiterhin, dass sich das Wachstum der Exponentialfunktion durchsetzt, anhand dieser Aussage können die Grenzwerte von Funktionen, die Exponentialfunktionen mit Basis $<1$ beinhalten, leicht bestimmt werden.\\
\danger Auch hier gilt es, auf das Vorzeichen der anderen Funktion(en) zu achten.

		


	


		\section{Ableitungsregeln}

	\subsection{Aktivität}

Quelle: Déclic 1ère

	\subsection{Exponentialfonktionen mit natürlicher Basis}

	\subsection{Exponentialfunktionen mit beliebiger Basis}

