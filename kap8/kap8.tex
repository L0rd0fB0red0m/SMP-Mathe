\chapter{Arithmetik}

	\section{Die vollst"andige Induktion}

Die vollst"andige Induktion ist eine mathematische Beweismethode, nach der eine Aussage f"ur alle nat"urlichen Zahlen bewiesen wird, die gr"o"ser oder gleich einem bestimmten Startwert sind.\\
Daher wird der Beweis in zwei Etappen durchgef"uhrt; mit dem \textbf{Induktionsanfang} beweist man die Aussage f"ur die kleinste Zahl, mit dem \textbf{Induktionsschritt} f"ur die n"achste Zahl, also logischerweise f"ur alle darauffolgenden Zahlen.\\\\

\begin{Beweis}
 Beweis der Gau"sschen Summenformel\\
 
\mathrm{Z\kern-.3em\raise-0.5ex\hbox{Z}} : $S(n) = \sum\limits_{i=1}^n i  = \dfrac{n(n+1)}{2}$\\\\

\begin{array}{rl}
\textbf{Induktionsanfang}: & $1=\dfrac{1(1+1)}{2} = 1 $ \\\\
\textbf{Induktionsvorraussetzung}: & $f"ur ein beliebiges, aber festes $k\in\N$ gilt: $\sum\limits_{i=1}^k  = \dfrac{k(k+1)}{2}$\\\\
\textbf{Induktionsbehauptung}: & $man behauptet, dass $\forall n \in \N$ gilt: $\sum\limits_{i=1}^{n+1}  = \dfrac{(n+1)((n+1)+1)}{2}$\\\\
 \textbf{Induktionsschluss}: &  \sum\limits_{i=1}^n + (n+1) = \dfrac{n(n+1)}{2} + \dfrac{2(n+1)}{2} = \dfrac{(n+1)((n+1)+1)}{2}\\\\
 \end{array}
\end{Beweis}\\\\

\begin{Beweis}
Beweis der Summe ungerader Zahlen\\

\mathrm{Z\kern-.3em\raise-0.5ex\hbox{Z}} $\forall n \in \N  :  \sum\limits_{k=1}^n (2k-1) = n^2$ \\\\

\begin{array}{rl}
\textbf{Induktionsanfang}: & $\sum\limits_{k=1}^1 (2k-1) = 2\cdot 1 -1 = 1 = 1^2  $\\\\
\textbf{Induktionsvorraussetzung}:&$f"ur ein beliebiges, aber festes $i \in \N$ gilt:  \sum\limits_{k=1}^i (2k-1) = i^2 \\\\
\textbf{Induktionsbehauptung}: &$man behauptet, dass $\forall n \in \N :  \sum\limits_{k=1}^{n+1} (2k-1) = (n+1)^2$ \\\\
 \textbf{Induktionsschluss}: & $\sum\limits_{k=1}^{n+1} (2k-1) = \sum\limits_{k=1}^{n} (2k-1) + 2(n+1)-1 = n^2 +2n+1 = (n+1)^2 $ \\\\

 \end{array}


\end{Beweis}\\\\

\begin{Beweis}
Beweis der Bernoullischen Ungleichung\\

\mathrm{Z\kern-.3em\raise-0.5ex\hbox{Z}} : $ \forall n \in \N \quad n>0 :  \qquad (1+x)^n \geq 1+nx  \qquad;  x\geq -1$ \\\\

\begin{array}{rl}

\textbf{Induktionsanfang}: & $(1+x)^0 = 1 \geq 1 = 1+0x $\\\\
\textbf{Induktionsvorraussetzung}: & $Es gelte nun: $(1+x)^n \geq 1+nx ; n\in \N_{0} $\\\\
\textbf{Induktionsbehauptung}: & $ (1+x)^{n+1} \geq 1+(n+1)x $\\\\
\textbf{Induktionsschluss}: & $ (1+x)^{n+1} = (1+x)^n \cdot (1+x) {\overset{\text{I.V.}} \geq} (1+nx)\cdot(1+x) = nx^2+nx+x+1$\\\\ 
&$\geq 1+x+nx = 1+(n+1)x   $\\\\

\end{array}

\end{Beweis}\\\\


\begin{Beweis}
Beweis der Summe der Quadratzahlen\\

Mittels Induktion l"asst sich ''nur''  eine vorhandene Formel beweisen.\\

\mathrm{Z\kern-.3em\raise-0.5ex\hbox{Z}} : S(n) = \sum\limits_{i=1}^n i^2 = \dfrac{n(n+1)(2n+1)}{6}\\\\

\begin{array}{rl}
\textbf{Induktionsanfang}: & $ S(1) = \sum\limits_{i=1}^1 i^2 = 1^2 = 1 = \dfrac{1(1+1)(2+1)}{6}$\\\\
\textbf{Induktionsvorraussetzung}: & $keine Ahnung was hier rein soll$\\\\
\textbf{Induktionsbehauptung}: & $S(n+1)=\sum\limits_{i=1}^{n+1} i^2 = \dfrac{(n+1)(n+2)(2(n+1)+1)}{6}$\\\\
\textbf{Induktionsschluss}: & $S(n) +(n+1)^2 = \dfrac{n(n+1)(2n+1)}{6} +(n+1)^2 = \dfrac{2n^3 +9n^2+13n+6}{6}$\\\\
& $\dfrac{(n+1)(n+2)(2(n+1)+1)}{6}  = \dfrac{2n^3 +9n^2+13n+6}{6} $\\\\
\end{array}

\end{Beweis}

\begin{Beweis}
Beweis f"ur eine Absch"atzung der Summe der Quadratzahlen\\

\mathrm{Z\kern-.3em\raise-0.5ex\hbox{Z}} : $ \sum\limits_{i=1}^n i^2 > \dfrac{n^3}{3} \\\\

\begin{array}{rl}
\textbf{Induktionsanfang}: & $ 1^2 > \dfrac{1^3}{3} $ \\\\
\textbf{Induktionsvorraussetzung}: & $f"ur ein beliebiges, aber festes $k\in\N$ gilt: $\sum\limits_{i=1}^k i^2  > \dfrac{k^3}{3}$\\\\
\textbf{Induktionsbehauptung}: & $man behauptet, dass $\forall n \in \N$ gilt: $\sum\limits_{i=1}^{n+1} i^2  > \dfrac{(n+1)^3}{3}$\\\\
$ \textbf{Induktionsschluss}: &  \sum\limits_{i=1}^{n+1} i^2 = \overbrace{\sum\limits_{i=1}^{n} i^2}^{>\textcolor{red}{\dfrac{n^3}{3}}} + (n+1)^2 > \textcolor{red}{\dfrac{n^3}{3}} +(n+1)^2 $\\\\
&$= \dfrac{n^3+3n^2+6n+3}{3} $\\\\
&$= \dfrac{n^3+3n^2+3n+1+3n+2}{3}  $\\\\
&$ = \dfrac{(n+1)^3}{3} + \dfrac{3n+2}{3} \overbrace{>}^{\textcolor{red}{n\geq0}} \dfrac{(n+1)^3}{3} $
\end{array}

\end{Beweis}\\\\

\begin{Beweis}
Summe der Kubikzahlen

\end{Beweis}\\\\

\begin{Beweis}
Beweis einer Absch"atzung der Fakult"at \\
$\forall n\in \N,\quad n\geq 4 : \quad n! >n^2 $\\\\

\begin{array}{rl}

\textbf{Induktionsanfang}: & $n_{o}=4 : \quad 4! = 4\cdot3 \cdot 2 \cdot 1 \cdot 0! = 24>16=4^2 $\\\\
\textbf{Induktionsvorraussetzung}: & $\exists n\in\N, \quad n\geq 4:\quad n!>n^2  $\\\\
\textbf{Induktionsbehauptung}: & $ n!\,\geq n^2 \Rightarrow (n+1)! > (n+1)^2 = \textcolor{red}{ (n+1)\cdot(n+1)} $\\\\
\textbf{Induktionsschluss}: &  $(n+1)! = (n+1)\cdot n ! \quad>\quad \textcolor{red}{ (n+1)\cdot n^2}  $\\\\
&$ \Rightarrow n^2 \overset{\mathrm{?}}{>} (n+1)  $\\\\
\textbf{Mini-induktion}: &$n_{0}=4: \quad 4^2=16>5=4+1 $\\\\
&$ \Rightarrow (n^2)' \overset{\mathrm{?}}{>} (n+1)' $\\\\
&$\Leftrightarrow 2n \overset{\mathrm{!}}{>} 1 \qquad \forall n \in \N $\\
\end{array}

\end{Beweis}








Am Besten so viele Induktionen wie m"oglich?