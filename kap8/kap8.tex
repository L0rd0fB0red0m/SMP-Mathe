\chapter{Arithmetik}

	\section{Die vollst"andige Induktion}

Die vollst"andige Induktion ist eine mathematische Beweismethode, nach der eine Aussage f�r alle nat"urlichen Zahlen bewiesen wird, die gr"o�er oder gleich einem bestimmten Startwert sind.\\
Daher wird der Beweis in zwei Etappen durchgef"uhrt; mit dem \textbf{Induktionsanfang} beweist man die Aussage f"ur die kleinste Zahl, mit dem \textbf{Induktionsschritt} f"ur die n"achste Zahl, also logischerweise f"ur alle darauffolgenden Zahlen\\\\

\begin{Beweis}
 Beweis der Gau"sschen Summenformel\\
 
\mathrm{Z\kern-.3em\raise-0.5ex\hbox{Z}} : $S(n) = \sum\limits_{i=1}^n  = \dfrac{n(n+1)}{2}$\\\\

\begin{array}{rl}
\textbf{Induktionsanfang}: & $1=\dfrac{1(1+1)}{2} = 1 $ \\\\
\textbf{Induktionsvorraussetzung}: & $f"ur ein beliebiges, aber festes $k\in\N$ gilt: $\sum\limits_{i=1}^k  = \dfrac{k(k+1)}{2}$\\\\
\textbf{Induktionsbehauptung}: & $es soll daraus folgen, dass $\forall n \in \N$ gilt: $\sum\limits_{i=1}^{n+1}  = \dfrac{(n+1)((n+1)+1)}{2}$\\\\
 \textbf{Induktionsschluss}: &  \sum\limits_{i=1}^n + (n+1) = \dfrac{n(n+1)}{2} + \dfrac{2(n+1)}{2} = \dfrac{(n+1)((n+1)+1)}{2}\\\\
 \end{array}
\end{Beweis}\\\\

\begin{Beweis}
Beweis der Summe ungerader Zahlen

\end{Beweis}\\\\

\begin{Beweis}
Beweis der Bernoullischen Ungleichung

\end{Beweis}\\\\

\begin{Beweis}
Summe der Quadratzahlen

\end{Beweis}\\\\

\begin{Beweis}
Summe der Kubikzahlen

\end{Beweis}\\\\

Am Besten so viele Induktionen wie m"oglich?