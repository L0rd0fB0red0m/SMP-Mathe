\chapter{Arithmetik oder Zahlentheorie}

	\section{Teilbarkeit}

\begin{Definition}
Die ganze Zahl $a\in\Z$ teilt $b\in\Z$, wenn es ein $x$ gibt, mit $b=a\cdot x$. Mann schreibt:
$$a|b$$

$a$ ist ein \textbf{Teiler} von $b$ und $b$ ist \textbf{Vielfaches} von $a$
\end{Definition}

\begin{Theorem}
Aus dieser Teilbarkeitsrelation ergeben sich mehrere Eigenschaften:\\
Sei $a,b,c,t\in\Z$ und $t\neq0$:\\\\
(1)\qquad$a|b$ \quad und \quad$b|c$ \quad dann \quad $a|c$\\
(2)\qquad$a|b$ \quad dann \quad $a|bc$\\
(3)\qquad$at|bt\quad\Leftrightarrow\quad a|b$\\
(4)\qquad$a|b$\quad dann \quad $b=0$ \quad oder\quad $|a| \leq |b| $\\
(5)\qquad$a|b$\quad und\quad $b|a$ \quad dann\quad $a=\pm b$
\end{Theorem}

\begin{Beweis}
\begin{itemize}
\item (1) \quad Wenn $a|b$ und $b|c$, dann gibt es $x,y\in\Z$ mit $b=a\cdot x$ und $c=b\cdot y$. Also gilt auch $c=b\cdot y=a\cdot(xy)$ und somit $a|c$\\ 
\item (2) \quad Weil offensichtlich $b|bc$ gilt, folgt die Aussage sofort aus Aussage (3) \\
\item (3)  \quad $at|bt\quad \Leftrightarrow\quad \exists x \in \Z : \, \, bt=atx \quad\Leftrightarrow \quad b=ax \quad\Leftrightarrow \quad a|b$ \\
\item (4) \quad Sei $b=ax$ mit $x\in\Z$. Wenn $x=0$, dann ist $b=0$. In alles anderen Fällen ist $|x|>1$ und daher $|b| = |a| \cdot |x| \geq |a|$ 
\item (5) \quad Wenn weder $a=0$ noch $b=0$ ist, dann folgt aus (4) $|a|\geq|b|\geq|a|$ und daher $a=\pm b$. Wenn also  $a=0$, dann folgt $b=0$
\end{itemize}
\end{Beweis}



	\section{Primzahlen}

Hier die Liste der Primzahlen bis 100: \\

2,     3,     5,     7,    11,    13,    17,    19,    23,    29,    31,    37,    41,    43,    47,    53,    59,    61,    67,    71,    73,    79,    83,    89,    97\\\\

\begin{Definition}
Eine natürliche Zahl heißt Primzahl, wenn sie in $\N$ genau zwei Teiler besitzt. 
\end{Definition}

\begin{Theorem}
Es gibt unendlich viele Primzahlen 
\end{Theorem}


\begin{Beweis}
Es gibt unendlich viele Primzahlen: Beweis nach Euklid\\\\

Jede ganze Zahl $n>1$ ist durch eine Primzahl teilbar. Entweder ist $n$ selber eine Primzahl oder $n=a\cdot b$ mit $a,b\in\N$.\\
Also ist $1<a=\dfrac{n}{b}<n$. Wenn man diesen Schritt endlich oft macht, kommt man am Ende auf ein neues $a'\in\P$.\\
$\A$: Nehmen wir jetzt an, es gäbe nur endlich viele Primzahlen $p_{1},p_{2},...,p_{r-1},p_{r}$.\\\\
Dann ist $$N=p_{1},p_{2},...,p_{r-1},p_{r} +1 = \left(\prod_{r=1}^{r}P_{r}\right) + 1 $$ eine ganze Zahl $>1$ und hat daher mindestens einen Primteiler $p_{l} \in\P$ und $l\in\{1,2,...,r-1,r\}$
Dann hat man:\\

$\Rightarrow$ $\left\{ \begin{array}{rccl}
$p_{l} & | & p_{1}\cdot p_{2}\cdot...\cdot p_{r-1}\cdot p_{r}$\\
$p_{l} & | & p_{1}\cdot p_{2}\cdot...\cdot p_{r-1}\cdot p_{r} +1$\\
\end{array}\right.\\


$\Leftrightarrow p_{l} | \, p_{1}\cdot p_{2} \cdot...\cdot p_{r-1} \cdot p_{r} - \left(  p_{1}\cdot p_{2} \cdot...\cdot p_{r-1} \cdot p_{r} +1 \right)$\\\\
$\Leftrightarrow p_{l} | \, (-1) \quad $oder$ \quad p_{l} | 1$ \qquad   WIDERSPRUCH, da $p_{l}=1$ aber $1\notin \P$ \\\\
$N$ muss also einen Primteiler haben ungleich $p_{r} \quad mit \quad r\in\{1,2,3,...,r-1,r\}$ oder \textbf{selber prim sein}. 

 
\end{Beweis}



	\section{Die vollst"andige Induktion}

Die vollst"andige Induktion ist eine mathematische Beweismethode, nach der eine Aussage f"ur alle nat"urlichen Zahlen bewiesen wird, die gr"o"ser oder gleich einem bestimmten Startwert sind.\\
Daher wird der Beweis in zwei Etappen durchgef"uhrt; mit dem \textbf{Induktionsanfang} beweist man die Aussage f"ur die kleinste Zahl, mit dem \textbf{Induktionsschritt} f"ur die n"achste Zahl, also logischerweise f"ur alle darauffolgenden Zahlen.\\\\

\begin{Beweis}
 Beweis der Gau"sschen Summenformel\\
 
\mathrm{Z\kern-.3em\raise-0.5ex\hbox{Z}} : $S(n) = \sum\limits_{i=1}^n i  = \dfrac{n(n+1)}{2}$\\\\

\begin{array}{rl}
\textbf{Induktionsanfang}: & $1=\dfrac{1(1+1)}{2} = 1 $ \\\\
\textbf{Induktionsvorraussetzung}: & $f"ur ein beliebiges, aber festes $k\in\N$ gilt: $\sum\limits_{i=1}^k  = \dfrac{k(k+1)}{2}$\\\\
\textbf{Induktionsbehauptung}: & $man behauptet, dass $\forall n \in \N$ gilt: $\sum\limits_{i=1}^{n+1}  = \dfrac{(n+1)((n+1)+1)}{2}$\\\\
 \textbf{Induktionsschluss}: &  \sum\limits_{i=1}^n + (n+1) = \dfrac{n(n+1)}{2} + \dfrac{2(n+1)}{2} = \dfrac{(n+1)((n+1)+1)}{2}\\\\
 \end{array}
\end{Beweis}\\\\

\begin{Beweis}
Beweis der Summe ungerader Zahlen\\

\mathrm{Z\kern-.3em\raise-0.5ex\hbox{Z}} $\forall n \in \N  :  \sum\limits_{k=1}^n (2k-1) = n^2$ \\\\

\begin{array}{rl}
\textbf{Induktionsanfang}: & $\sum\limits_{k=1}^1 (2k-1) = 2\cdot 1 -1 = 1 = 1^2  $\\\\
\textbf{Induktionsvorraussetzung}:&$f"ur ein beliebiges, aber festes $i \in \N$ gilt:  \sum\limits_{k=1}^i (2k-1) = i^2 \\\\
\textbf{Induktionsbehauptung}: &$man behauptet, dass $\forall n \in \N :  \sum\limits_{k=1}^{n+1} (2k-1) = (n+1)^2$ \\\\
 \textbf{Induktionsschluss}: & $\sum\limits_{k=1}^{n+1} (2k-1) = \sum\limits_{k=1}^{n} (2k-1) + 2(n+1)-1 = n^2 +2n+1 = (n+1)^2 $ \\\\

 \end{array}


\end{Beweis}\\\\

\begin{Beweis}
Beweis der Bernoullischen Ungleichung\\

\mathrm{Z\kern-.3em\raise-0.5ex\hbox{Z}} : $ \forall n \in \N \quad n>0 :  \qquad (1+x)^n \geq 1+nx  \qquad;  x\geq -1$ \\\\

\begin{array}{rl}

\textbf{Induktionsanfang}: & $(1+x)^0 = 1 \geq 1 = 1+0x $\\\\
\textbf{Induktionsvorraussetzung}: & $Es gelte nun: $(1+x)^n \geq 1+nx ; n\in \N_{0} $\\\\
\textbf{Induktionsbehauptung}: & $ (1+x)^{n+1} \geq 1+(n+1)x $\\\\
\textbf{Induktionsschluss}: & $ (1+x)^{n+1} = (1+x)^n \cdot (1+x) {\overset{\text{I.V.}} \geq} (1+nx)\cdot(1+x) = nx^2+nx+x+1$\\\\ 
&$\geq 1+x+nx = 1+(n+1)x   $\\\\

\end{array}

\end{Beweis}\\\\


\begin{Beweis}
Beweis der Summe der Quadratzahlen\\

Mittels Induktion l"asst sich ''nur''  eine vorhandene Formel beweisen.\\

\mathrm{Z\kern-.3em\raise-0.5ex\hbox{Z}} : S(n) = \sum\limits_{i=1}^n i^2 = \dfrac{n(n+1)(2n+1)}{6}\\\\

\begin{array}{rl}
\textbf{Induktionsanfang}: & $ S(1) = \sum\limits_{i=1}^1 i^2 = 1^2 = 1 = \dfrac{1(1+1)(2+1)}{6}$\\\\
\textbf{Induktionsvorraussetzung}: & $keine Ahnung was hier rein soll$\\\\
\textbf{Induktionsbehauptung}: & $S(n+1)=\sum\limits_{i=1}^{n+1} i^2 = \dfrac{(n+1)(n+2)(2(n+1)+1)}{6}$\\\\
\textbf{Induktionsschluss}: & $S(n) +(n+1)^2 = \dfrac{n(n+1)(2n+1)}{6} +(n+1)^2 = \dfrac{2n^3 +9n^2+13n+6}{6}$\\\\
& $\dfrac{(n+1)(n+2)(2(n+1)+1)}{6}  = \dfrac{2n^3 +9n^2+13n+6}{6} $\\\\
\end{array}

\end{Beweis}

\begin{Beweis}
Beweis f"ur eine Absch"atzung der Summe der Quadratzahlen\\

\mathrm{Z\kern-.3em\raise-0.5ex\hbox{Z}} : $ \sum\limits_{i=1}^n i^2 > \dfrac{n^3}{3} \\\\

\begin{array}{rl}
\textbf{Induktionsanfang}: & $ 1^2 > \dfrac{1^3}{3} $ \\\\
\textbf{Induktionsvorraussetzung}: & $f"ur ein beliebiges, aber festes $k\in\N$ gilt: $\sum\limits_{i=1}^k i^2  > \dfrac{k^3}{3}$\\\\
\textbf{Induktionsbehauptung}: & $man behauptet, dass $\forall n \in \N$ gilt: $\sum\limits_{i=1}^{n+1} i^2  > \dfrac{(n+1)^3}{3}$\\\\
$ \textbf{Induktionsschluss}: &  \sum\limits_{i=1}^{n+1} i^2 = \overbrace{\sum\limits_{i=1}^{n} i^2}^{>\textcolor{red}{\dfrac{n^3}{3}}} + (n+1)^2 > \textcolor{red}{\dfrac{n^3}{3}} +(n+1)^2 $\\\\
&$= \dfrac{n^3+3n^2+6n+3}{3} $\\\\
&$= \dfrac{n^3+3n^2+3n+1+3n+2}{3}  $\\\\
&$ = \dfrac{(n+1)^3}{3} + \dfrac{3n+2}{3} \overbrace{>}^{\textcolor{red}{n\geq0}} \dfrac{(n+1)^3}{3} $
\end{array}

\end{Beweis}\\\\


\begin{Beweis}
Beweis einer Absch"atzung der Fakult"at \\
$\forall n\in \N,\quad n\geq 4 : \quad n! >n^2 $\\\\

\begin{array}{rl}

\textbf{Induktionsanfang}: & $n_{o}=4 : \quad 4! = 4\cdot3 \cdot 2 \cdot 1 \cdot 0! = 24>16=4^2 $\\\\
\textbf{Induktionsvorraussetzung}: & $\exists n\in\N, \quad n\geq 4:\quad n!>n^2  $\\\\
\textbf{Induktionsbehauptung}: & $ n!\,\geq n^2 \Rightarrow (n+1)! > (n+1)^2 = \textcolor{red}{ (n+1)\cdot(n+1)} $\\\\
\textbf{Induktionsschluss}: &  $(n+1)! = (n+1)\cdot n ! \quad>\quad \textcolor{red}{ (n+1)\cdot n^2}  $\\\\
&$ \Rightarrow n^2 \overset{\mathrm{?}}{>} (n+1)  $\\\\
\textbf{Mini-induktion}: &$n_{0}=4: \quad 4^2=16>5=4+1 $\\\\
&$ \Rightarrow (n^2)' \overset{\mathrm{?}}{>} (n+1)' $\\\\
&$\Leftrightarrow 2n \overset{\mathrm{!}}{>} 1 \qquad \forall n \in \N $\\
\end{array}

\end{Beweis}








Am Besten so viele Induktionen wie m"oglich?