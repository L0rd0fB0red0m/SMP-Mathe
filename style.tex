%Kapitel-Titelblatt
\makeatletter
\def\thickhrulefill{\leavevmode \leaders \hrule height 1.2ex \hfill \kern \z@}
\def\@makechapterhead#1{
  \vspace*{10\p@}%
  {\parindent \z@ \centering \reset@font
        \thickhrulefill\quad
        \huge\scshape\bfseries\textit{\@chapapp{}  \thechapter}\normalsize
        \quad \thickhrulefill
        \par\nobreak
        \vspace*{10\p@}%
        \interlinepenalty\@M
        \hrule
        \vspace*{10\p@}%
        \Huge \bfseries #1 \par\nobreak
        \par
        \vspace*{10\p@}%
        \hrule
        \vskip 50\p@
  }}


%Bemerkung:
\newenvironment{Bemerkung}{\par\vspace{10pt}\small % Vertical white space above the remark and smaller font size
\begin{list}{}{
\leftmargin=35pt % Indentation on the left
\rightmargin=25pt}\item\ignorespaces % Indentation on the right
\makebox[-2.5pt]{\begin{tikzpicture}[overlay]
\node[draw=ocre!60,line width=1pt,circle,fill=ocre!25,font=\sffamily\bfseries,inner sep=2pt,outer sep=0pt] at (-15pt,0pt){\textcolor{ocre}{ \textbackslash \textbackslash}};\end{tikzpicture}} % Orange R in a circle
\advance\baselineskip -1pt}{\end{list}\vskip5pt} % Tighter line spacing and white space after remark


%Beweis:
\newenvironment{Beweis}
  {\begin{tcolorbox}[colback=grey!5!white,colframe=grey!75!black,opacityframe=0.5, opacitybacktitle=0.5,sharpish corners,boxrule=0pt,frame hidden,title=Beweis]}
  {\end{tcolorbox}}


%Definition:
\newenvironment{Definition}
  {\begin{tcolorbox}[colback=red!5!white,colframe=red!75!black,sharpish corners,boxrule=0pt,frame hidden,title=Definition \thesubsection]}
  {\end{tcolorbox}}}


%Theorem:
\newenvironment{Theorem}
  {\begin{tcolorbox}[colback=blue!5!white,colframe=blue!75!black,sharpish corners,boxrule=0pt,frame hidden,title=Theorem]}
  {\end{tcolorbox}}}
%%%%%%%%%%%%%%%%%%%%%%%%%%%%%%%%%%%%%%%%%%%%%%%%%%%%%%%%%%%%%%%%%%%%%
%Es fehlen:
%Beispiele
%Fancy titles?
%\newenvironment{Beweis}
%{\par\noindent\normalfont\par\nopagebreak%
%  \begin{mdframed}[
%     linewidth=3pt,
%     linecolor=pink,
%     bottomline=false,topline=false,rightline=false,
%     innerrightmargin=0pt,innertopmargin=0pt,innerbottommargin=0pt,
%     innerleftmargin=1em,% Distanz zwischen vertikaler Strich & Inhalt
%     skipabove=.5\baselineskip
%   ]}
%{\end{mdframed}}
