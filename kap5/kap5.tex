\chapter{Vektorielle Geometrie}


\section{Vektoren}
\begin{Beweis}
Ein Vektor ist Element eines Vektorraums.
\end{Beweis}\\
Vektorräume, wir erinnern uns zurück. Verknüpfungen, inverse Elemente und die dazugehörenden Gesetze, konsequente Definitionen und mathematische Korrektheit, die guten alten Zeiten...\\
Tatsächlich kann ein Vektor in den meisten Fällen als Verschiebung bezeichnet werden, \textbf{nicht aber als Pfeil oder Strich!}\\
\subsection{Besondere Vektoren}
\subsubsection{Der Ortsvektor}
Der Vektor von $O$ auf den Punkt $P$, geschrieben als $\vec{OP}$ oder $\vec{o}$.\\
Hat $P$ die Koordinaten $(P_1|P_2|...|P_n)$, so besitzt $\vec{o}$ die Darstellung $\left(\begin{array}{c} P_1 \\ P_2 \\ ...\\P_n\end{array}\right)$.
\subsubsection{Der Nullvektor}
Der Vektor mit Wert $\left(\begin{array}{c} 0 \\ 0 \\ ...\\0\end{array}\right)$, er hat keine und alle Richtungen zugleich.
\begin{Bemerkung}
Er ist somit das neutrale Element der Vektoraddition.
\end{Bemerkung}
\subsubsection{Der Verbindungsvektor}
Der Vektor $\vec{AB}$ ist der Vektor, der den Punkt $A$ auf den Punnḱt $B$ abbildet. Er ist definiert als: $\vec{AB}=\vec{OB}-\vec{OA}$.
\\Rechnung? b1-a1 b2-12 ...?
\subsubsection{Der Gegenvektor}
Der Gegenvektor zu $\vec{AB}$ ist $\vec{BA}$, definiert als  $-\vec{AB}$.
\begin{Bemerkung}
Er ist somit das inverse Element der Vektoraddition.
\end{Bemerkung}
\section{Winkel zwischen Vektoren}
\section{Geraden}
\section{Ebenen}
\section{Skalarprodukte??}
