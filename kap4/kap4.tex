\chapter{Trigonometrie}
\@fleqntrue

\section{Kurze Wiederholung}
\begin{Definition}
  Im Kreis mit Radius 1 gelte:\\
  $\cos(\alpha)= x_{_M}$\\
  $\sin(\alpha)= y_{_M}$\\
  $\tan(\alpha)=\dfrac {sin(\alpha)} {cos(\alpha)}$
\end{Definition}\\
\definecolor{qqwuqq}{rgb}{0,0.39,0}
\definecolor{ttzzqq}{rgb}{0.2,0.6,0}
\definecolor{ffqqqq}{rgb}{1,0,0}
\definecolor{ttttff}{rgb}{0.2,0.2,1}
\definecolor{uququq}{rgb}{0.25,0.25,0.25}
\begin{tikzpicture}[line join=round,x=2.0833333333333335cm,y=2.0833333333333335cm]
  \draw[->,color=black] (-1.2,0) -- (1.2,0) node[below]{$x$};
  \foreach \x in {-1,1}
  \draw[shift={(\x,0)},color=black] (0pt,2pt) -- (0pt,-2pt) node[below] {\footnotesize $\x$};
  \draw[->,color=black] (0,-1.2) -- (0,1.2) node[left]{$y$};
  \foreach \y in {-1,1}
  \draw[shift={(0,\y)},color=black] (2pt,0pt) -- (-2pt,0pt) node[left] {\footnotesize $\y$};
  \draw[color=black] (0pt,-10pt) node[right] {\footnotesize $0$};
  \clip(-1.2,-1.2) rectangle (1.2,1.2);
  \draw [shift={(0,0)},color=qqwuqq,fill=qqwuqq,fill opacity=0.1] (0,0) -- (0:0.21) arc (0:57.13:0.21) -- cycle;
  \draw [line width=0.4pt] (0,0) circle (2.08cm);
  \draw (0,0)-- (0.54,0.84);
  \draw [line width=1.6pt,color=ttttff] (0,0.84)-- (0.54,0.84);
  \draw [line width=1.6pt,color=ffqqqq] (0.54,0.84)-- (0.54,0);
  \draw [line width=1.6pt,color=ttzzqq] (1,-1.2) -- (1,1.2);
  \begin{scriptsize}
    \fill [color=uququq] (0,0) circle (1.5pt);
    \draw[color=uququq] (-0.11,-0.11) node {$O$};
    \fill [color=black] (0.54,0.84) circle (1.5pt);
    \draw[color=black] (0.61,0.92) node {$M$};
    \fill [color=uququq] (0,0.84) circle (1.5pt);
    \draw[color=uququq] (0.03,0.94) node {$A$};
    \fill [color=uququq] (0.54,0) circle (1.5pt);
    \draw[color=uququq] (0.7,0.04) node {$B$};
    \draw[color=ttttff] (0.3,0.74) node {cos(α)};
    \draw[color=ffqqqq] (0.74,0.4) node {sin(α)};
    \draw[color=ttzzqq] (0.82,1.1) node {tan(α)};
    \draw[color=qqwuqq] (0.12,0.07) node {$\alpha$};
  \end{scriptsize}
\end{tikzpicture}
\\\\\\
Es ergeben sich folgende (wissenswerte) Werte:
\\
\begin{center}
  \begin{tabu} to 0.9\textwidth{|X[c]|X[c]|X[c]|X[c]|X[c]|X[c]|X[c]|X[c]|X[c]|}

    \hline
    & 0$^\circ$ & 30$^\circ$ & 45$^\circ$ & 60$^\circ$ & 90$^\circ$ & 180$^\circ$ & 270$^\circ$ & 360$^\circ$\\
    & 0 & $\dfrac{\pi}{6}$ & $\dfrac{\pi}{4}$ & $\dfrac{\pi}{3}$ & $\dfrac{\pi}{2}$ & \pi & $\dfrac{3\pi}{2}$ & $2\pi$\\
    \hline
    $\sin(\alpha)$ & 0 & $\dfrac{1}{2}$ & $\dfrac{1}{\sqrt{2}}$ & $\dfrac{\sqrt{3}}{2}$ & 1 & 0 & -1 & 0\\
    \hline
    $\cos(\alpha)$ & 1 & $\dfrac{\sqrt{3}}{2}$ & $\dfrac{1}{\sqrt{2}}$ & $\dfrac{1}{2}$ & 0 & -1 & 0 & 1\\
    \hline
    $\tan(\alpha)$ & 0 & $\dfrac{{1}}{\sqrt{3}}$ & 1 & $\sqrt{3}$ & X & 0 & X & 0\\
    \hline
  \end{tabu}

\end{center}
\section{Addtions- und Verdopplungssätze}
\begin{Theorem}
  $\cos(a - b) = \cos(a)\cos(b) + \sin(a)\sin(b)$\\
  $\sin(a - b) = \sin(a)\cos(b) - \cos(a)\sin(b)$\\
  $\cos(a + b) = \cos(a)\cos(b) - \sin(a)\sin(b)$\\
  $\sin(a + b) = \sin(a)\cos(b) + \cos(a)\sin(b)$
\end{Theorem}
\\
Hieraus ergeben sich einige weitere Relationen, wie z.B. $\sin(2a)$. Diese lassen sich jedoch schnell und leicht herleiten.
\section{Allgemeine Sinus- und Kosinussätze}
In einem beliebigen Dreieck gelten abgewandelte Formen der aus der 8. Klasse bekannten Sätze:
\begin{Theorem}
  $\dfrac {a} {sin(\alpha)}=\dfrac {b} {sin(\beta)}=\dfrac {c} {sin(\gamma)}$\\\\
  $c²=a²+b²-2abcos(\gamma)$
\end{Theorem}
\definecolor{qqwuqq}{rgb}{0,0.39,0}
\begin{tikzpicture}[line cap=round,line join=round,>=triangle 45,x=0.53cm,y=0.64cm]
  \clip(-0.4,-4.3) rectangle (9,3);
  \draw [shift={(4.22,1.92)},color=qqwuqq,fill=qqwuqq,fill opacity=0.1] (0,0) -- (-134.14:1.5) arc (-134.14:-47.61:1.5) -- cycle;
  \draw [shift={(-0.38,-2.82)},color=qqwuqq,fill=qqwuqq,fill opacity=0.1] (0,0) -- (1.04:1.5) arc (1.04:45.86:1.5) -- cycle;
  \draw [shift={(8.4,-2.66)},color=qqwuqq,fill=qqwuqq,fill opacity=0.1] (0,0) -- (132.39:1.5) arc (132.39:181.04:1.5) -- cycle;
  \draw (4.22,1.92)-- (8.4,-2.66);
  \draw (4.22,1.92)-- (-0.38,-2.82);
  \draw (8.4,-2.66)-- (-0.38,-2.82);
  \begin{scriptsize}
    \draw[color=black] (6.7,-0.44) node {$a$};
    \draw[color=black] (1.76,-0.08) node {$b$};
    \draw[color=black] (4.06,-2.28) node {$c$};
    \draw[color=qqwuqq] (4.4,1.05) node {$\gamma$};
    \draw[color=qqwuqq] (0.7,-2.45) node {$\alpha$};
    \draw[color=qqwuqq] (7.5,-2.2) node {$\beta$};
  \end{scriptsize}
\end{tikzpicture}\\
Man bemerkt, dass sich die bekannten Relationen ergeben, wenn einer der Winkel den Wert $\dfrac{\pi}{2}$ annimmt.
\section{Sinusfunktionen}
Zur Vollständigen Funktionsdiskussion einer Sinus-Funktion sind einige Besonderheiten zu beachten:
\begin{enumerate}
  \item Amplitude und Periodizität\\
  Eine Funktion der Form $f(x)=a\cdot\sin(b(x-c))+d$ hat:
  \begin{itemize}
    \item die Periode $P = \dfrac{2\pi}{|b|}$
    \item die Amplitude $A = |a|$
    \item die Verschiebung entlang der $x$-Achse um $d$ und entlang der $y$-Achse um $c$
    \end{itemize}
  \item Symmetrieeigenschaften\\
  Hier sollte zumindest bekannt sein, dass $f(x)=\sin(x)$ punktsymmetrisch zum Origo ist, und dass $f(x)=\cos(x)$ Achsensymmetrisch zur $y$-Achse ist.
  \item Die Null-, Extrem- und Wendestellen sind in Form einer Menge anzugeben. (Es sei denn, die Aufgabenvorschrift fordert explizit auf eine Begrenzung auf ein angegebenes Intervall auf)\\
  \begin{Beispiel}
    Die Nullstellen der Funktion $f(x)=\sin(x)$ lassen sich dartstellen als: $x \in \{k\pi|k \in \Z\}$
  \end{Beispiel}
  \item Bei der Teilung durch eine Sinusfunktion können Definitionslücken an dessen Nullstellen entstehen. Auch diese können in der bereits gezeigten Form angegeben werden.
  \end{enumerate}
\section{Polarkoordinaten}
In der Kursstufe beschränken wir uns auf die Benutzung von Polarkoordinaten für Punkte in der Ebene (2D).
\begin{Definition}
  Polarkoordinaten sind eine Form der eindeutigen Punktangaben, doch anstatt wie kartesische Koordinaten 2 Entfernungen $x$ und $y$ zu verwenden, haben sie die Form $(r|\varphi)$. $r$ ist hierbei die Entfernung zum Origo und $\varphi$ ein orientierter Winkel (in $rad$).
\end{Definition}
\definecolor{zzzzzz}{rgb}{0.6,0.6,0.6}
\definecolor{ffqqqq}{rgb}{1,0,0}
\definecolor{qqzzqq}{rgb}{0,0.6,0}
\begin{tikzpicture}[line cap=round,line join=round,x=1.0cm,y=1.0cm]
  \draw[->,color=black] (-0.2,0) -- (2.64,0);
  \foreach \x in {,1,2}
  \draw[shift={(\x,0)},color=black] (0pt,-2pt);
  \draw[color=black] (2.47,0.04) node [anchor=south west] { x};
  \draw[->,color=black] (0,-0.29) -- (0,2.63);
  \draw[color=black] (0.05,2.43) node [anchor=west] { y};
  \clip(-0.2,-0.29) rectangle (2.64,2.63);
  \draw [shift={(0,0)},color=ffqqqq,fill=ffqqqq,fill opacity=0.1] (0,0) -- (0:0.41) arc (0:46.48:0.4) -- cycle;
  \draw [color=qqzzqq] (0,0)-- (2.05,2.16);
  \draw [color=zzzzzz] (0,2.16)-- (2.05,2.16);
  \draw [color=zzzzzz] (2.05,2.16)-- (2.05,0);
  \begin{scriptsize}
    \fill [color=black] (2.05,2.16) circle (1.5pt);
    \draw[color=black] (2.12,2.29) node {$P$};
    \draw[color=qqzzqq] (1.16,1.04) node {r};
    \draw[color=ffqqqq] (0.5,0.17) node {$\varphi$};
    \draw[color=zzzzzz] (0.96,2.02) node {y};
    \draw[color=zzzzzz] (1.9,1.03) node {x};
  \end{scriptsize}
\end{tikzpicture}
\subsection{Umrechnung}
\subsubsection{Kartesisch $\rightarrow$ Polar}
\begin{itemize}
  \item  $r=\sqrt{x^2+y^2}$
  \item $\varphi = \tan(\dfrac{y}{x})$
\end{itemize}
\subsubsection{Polar $\rightarrow$ Kartesisch}
\begin{itemize}
  \item $x = r\cdot\cos(\varphi)$
  \item  $y=r\cdot\sin(\varphi)$
\end{itemize}
\section{Beispiel einer Funktionsdiskussion}
Sei die Funktion $f(x)=2\cos(x)+2\sin(x)\cos(x)$, ihr Schaubild sei K.\\
Untersuchen Sie K im Intervall $[0;2\pi]$ auf gemeinsame Punkte mit der $x$-Achse, sowie Extrem- und Wendepunkte. Zeichnen Sie K im Intervall $[0;2\pi]$. Untersuchen Sie K auf Symmetrie.
\subsection{Definitionsmenge}
  $D=\R$
\subsection{Periodizität und Amplitude}
Die Periode von $f$ ist $P=2\pi$. Die Amplitude $A$ beträgt $\dfrac{3}{2}\sqrt{3}$.
\subsection{Nullstellen}
\subsubsection{Notwendige und hinreichende Bedingung}
\begin{align*}
  f(x)&=0\\
  \Leftrightarrow 2\cos(x)+2\sin(x)\cos(x)&=0\\
  \Leftrightarrow 2\cos(x)(1+\sin(x))&=0\\
  \stackrel{S.d.N}{\Rightarrow}
  \left\{\begin{array}{rccl}
    2\cos(x)&=&0&\\
    1+\sin(x)&=&0&
  \end{array}\right\\
  \Leftrightarrow
  \left\{ \begin{array}{rccl}
    \cos(x)&=&0&\\
    \sin(x)&=&-1&
  \end{array}\right\\
  \Leftrightarrow
  \left\{ \begin{array}{rccl}
    x_1&=&\dfrac{1}{2}\pi&\\\\
    x_2&=&\dfrac{3}{2}\pi&
  \end{array}\right\\
  \Rightarrow \mathbb{L}=\{(\dfrac{1}{2}\pi|0);(\dfrac{3}{2}\pi|0)\}
\end{align*}
\subsection{Ableitungen}
  \begin{align*}
    f'(x)&=-2\sin(x)+2(\cos(x)\cos(x)-\sin(x)\sin(x))\\
    &=-2\sin(x)+2(\cos^2(x)-\sin^2(x))\\
    &=-2\sin(x)+2(1-\sin^2(x)-\sin^2(x))\\
    &=-4\sin^2(x)-2\sin(x)+2\\
    f''(x)&=-4(\cos(x)\sin(x)+\sin(x)\cos(x))-2\cos(x)\\
    &=-8\sin(x)\cos(x)-2\cos(x)\\
    f'''(x)&=-8(\cos(x)\cos(x)-\sin(x)\sin(x))+2\sin(x)\\
    &=-8(1-\sin^2(x)-\sin^2(x))+2\sin(x)\\
    &=16\sin^2(x)+2\sin(x)-8
  \end{align*}\\
\subsection{Extremstellen}
\subsubsection{Notwendige Bedingung}
\begin{align*}
  f'(x)&=0\\
  \Leftrightarrow4\sin^2(x)-2\sin(x)+2&=0\\
  Substitution: y=\sin(x)\\
  \Rightarrow4y^2-2y+2&=0\\
  \stackrel{ABC-Formel}{\Rightarrow}y_{1,2}&=\dfrac{2\pm\sqrt{(-2)^2-4*(-4)*2}}{-8}\\
  Resusbsitution:\\
  \Rightarrow
  \left\{\begin{array}{rccl}
    \sin(x)&=\dfrac{2+\sqrt{20}}{-8}\\
    \sin(x)&=\dfrac{2-\sqrt{20}}{-8}
  \end{array}\right\\
  \Rightarrow\mathbb{L}=\{\dfrac{1}{6}\pi;\dfrac{5}{6}\pi;\dfrac{3}{2}\pi\}
\end{align*}
\subsubsection{Hinreichende Bedingung}
\begin{align*}
  f''(x)&\neq0\\
  \Rightarrow
  \left\{\begin{array}{rccl}
    f''(\dfrac{1}{6}\pi)&\stackrel{?}{=}0\\\\
    f''(\dfrac{5}{6}\pi)&\stackrel{?}{=}0\\\\
    f''(\dfrac{3}{2}\pi)&\stackrel{?}{=}0
  \end{array}\right\\
  \Leftrightarrow
  \left\{\begin{array}{rccl}
    8\sin(\dfrac{1}{6}\pi)\cos(\dfrac{1}{6}\pi)-2\sin(\dfrac{1}{6}\pi)&\stackrel{?}{=}0\\\\
    8\sin(\dfrac{5}{6}\pi)\cos(\dfrac{5}{6}\pi)-2\sin(\dfrac{5}{6}\pi)&\stackrel{?}{=}0\\\\
    8\sin(\dfrac{3}{2}\pi)\cos(\dfrac{3}{2}\pi)-2\sin(\dfrac{3}{2}\pi)&\stackrel{?}{=}0
  \end{array}\right\\
  \Leftrightarrow
  \left\{\begin{array}{rccl}
    8*\dfrac{1}{2}*\dfrac{\sqrt{3}}{2}-2*\dfrac{\sqrt{3}}{2}&\stackrel{!}{\neq}&0&,  <0 \Rightarrow HP\\
    8*\dfrac{1}{2}*-\dfrac{\sqrt{3}}{2}-2*-\dfrac{\sqrt{3}}{2}&\stackrel{!}{\neq}&0&,  >0 \Rightarrow TP\\
    8*(-1)(0)-2(0)&\stackrel{!}{=}&0& \Rightarrow \text{kein } EP
  \end{array}\right\\
\end{align*}
\subsubsection{Ergebnis}
Auf dem Intervall $[0;2\pi]$  besitzt K den Hochpunkt H$(\dfrac{1}{6}\pi|f(\dfrac{1}{6}\pi))$ und den Tiefpunk T$(\dfrac{5}{6}\pi|f(\dfrac{5}{6}\pi))$.\\
$\Leftrightarrow$ H$(\dfrac{1}{6}\pi|\dfrac{3}{2}\sqrt{3})$ und T$(\dfrac{5}{6}\pi|-\dfrac{3}{2}\sqrt{3})$.
\subsection{Wendestellen}
\subsubsection{Notwendige Bedingung}
\begin{align*}
  f''(x)=0\\
  \Leftrightarrow -8\sin(x)\cos(x)-2\cos(x)=0\\
  \Leftrightarrow \cos(x)(-2-8\sin(x))=0\\
  \stackrel{SdN}{\Rightarrow}
  \left\{\begin{array}{rccl}
    \cos(x)&=&0\\
    \sin(x)&=&-\dfrac{1}{4}
  \end{array}\right\\
  \Rightarrow \mathbb{L}=\{\dfrac{1}{2}\pi;\dfrac{3}{2}\pi;\sim3,394;\sim6,031\}
\end{align*}
\subsubsection{Hinreichende Bedingung}

\begin{align*}
  f'''(x)&\neq0\\
  \Rightarrow
  \left\{\begin{array}{rccl}
    f'''(\dfrac{1}{2}\pi)&\stackrel{?}{=}0\\\\
    f'''(\dfrac{3}{2}\pi)&\stackrel{?}{=}0\\\\
    f'''(3,394)&\stackrel{?}{=}0\\\\
    f'''(6,031)&\stackrel{?}{=}0
  \end{array}\right\\
  \Leftrightarrow
  \left\{\begin{array}{rccl}
    16\sin^2(\dfrac{1}{2}\pi)+\sin(\dfrac{1}{2}\pi)-8&\stackrel{?}{=}0\\\\
    16\sin^2(\dfrac{3}{2}\pi)+\sin(\dfrac{3}{2}\pi)-8&\stackrel{?}{=}0\\
    16\sin^2(3,394)+\sin(3,394)-8&\stackrel{?}{=}0\\
    16\sin^2(6,031)+\sin(6,031)-8&\stackrel{?}{=}0
  \end{array}\right\\
  \Leftrightarrow
  \left\{\begin{array}{rccl}
    16*1+1-8&\stackrel{!}{\neq}&0&,>0\Rightarrow WP\\
    16*1-1-8&\stackrel{!}{\neq}&0&,>0,\text{ außerdem: } f'(\dfrac{3}{2}\pi)=0\Rightarrow Sattelpunkt\\
    -7,5&\stackrel{!}{=}&0&,<0\Rightarrow WP\\
    -7,5&\stackrel{!}{=}&0&,<0\Rightarrow WP&
  \end{array}\right\\
\end{align*}
\subsubsection{Ergebnis}
Auf dem Intervall $[0;2\pi]$  besitzt K die Wendepunkte $(\dfrac{1}{2}\pi|f(\dfrac{1}{2}\pi))$, $(3,394|f(3,394))$, $(6,031|f(6,031))$ und den Sattelpunkt $(\dfrac{3}{2}\pi|f(\dfrac{3}{2}\pi))$.\\
\Leftrightarrow $W_1(\dfrac{1}{2}\pi|0)$, $W_2(3,394|-1,452)$, $W_3(6,031|1,452)$, $S(\dfrac{3}{2}\pi|0)$.\\
\subsection{Schaubild}
\definecolor{ccqqqq}{rgb}{0.8,0,0}
\definecolor{cqcqcq}{rgb}{0.75,0.75,0.75}
\begin{tikzpicture}[line cap=round,line join=round,>=triangle 45,x=1.0cm,y=1.0cm]
  \draw [color=cqcqcq,dash pattern=on 1pt off 1pt, xstep=1.0cm,ystep=1.0cm] (-0.62,-3.28) grid (7.22,3.26);
  \draw[->,color=black] (-0.62,0) -- (7.22,0);
  \foreach \x in {,1,2,3,4,5,6,7}
  \draw[shift={(\x,0)},color=black] (0pt,-2pt) node[below] {\footnotesize $\x$};
  \draw[color=black] (7.04,0.04) node [anchor=south west] { x};
  \draw[->,color=black] (0,-3.28) -- (0,3.26);
  \foreach \y in {-3,-2,-1,1,2,3}
  \draw[shift={(0,\y)},color=black] (2pt,0pt) -- (-2pt,0pt) node[left] {\footnotesize $\y$};
  \draw[color=black] (0.05,3.05) node [anchor=west] { y};
  \draw[color=black] (0pt,-10pt) node[right] {\footnotesize $0$};
  \draw[color=ccqqqq] (1.5,2.5) node {K};
  \clip(-0.62,-3.28) rectangle (7.22,3.26);
  \draw[line width=1.6pt,color=ccqqqq, smooth,samples=100,domain=-0.6249016652708163:7.217846281536262] plot(\x,{2*cos(((\x))*180/pi)+2*sin(((\x))*180/pi)*cos(((\x))*180/pi)});
\end{tikzpicture}
\subsection{Symmetrie}
K ist punktsymmetrisch zu $W_1$, denn es gilt:
\begin{align*}
  f(\dfrac{1}{2}\pi+x)&=-1*f(\dfrac{1}{2}\pi-x)\\
  \Leftrightarrow
  2\cos(\dfrac{1}{2}\pi+x)+2\sin(\dfrac{1}{2}\pi+x)\cos(\dfrac{1}{2}\pi+x)&=-1*(2\cos(\dfrac{1}{2}\pi-x)+2\sin(\dfrac{1}{2}\pi-x)\cos(\dfrac{1}{2}\pi-x))\\
  \Leftrightarrow
  -2\sin(x)-2\cos(x)\sin(x)&=-1*(2\sin(x)+2\cos(x)\sin(x))
\end{align*}
K ist außerdem zu $S$ punktsymmetrisch, denn es gilt:
\begin{align*}
  f(\dfrac{3}{2}\pi+x)&=-1*f(\dfrac{3}{2}\pi-x)\\
  \Leftrightarrow
  2\cos(\dfrac{3}{2}\pi+x)+2\sin(\dfrac{3}{2}\pi+x)\cos(\dfrac{3}{2}\pi+x)&=-1*(2\cos(\dfrac{3}{2}\pi-x)+2\sin(\dfrac{3}{2}\pi-x)\cos(\dfrac{3}{2}\pi-x))\\
  \Leftrightarrow
  2\sin(x)+2\cos(x)\sin(x)&=-1*(-2\sin(x)-2\cos(x)\sin(x))
\end{align*}
