\chapter{Trigonometrie}
\section{Kurze Wiederholung}
\begin{Bws}
Im Kreis mit Radius 1 gelte:\\
$\cos(\alpha)= x_{_M}$\\
$\sin(\alpha)= y_{_M}$\\
$\tan(\alpha)=\dfrac {sin(\alpha)} {cos(\alpha)}$
\end{Bws}\\

\definecolor{qqwuqq}{rgb}{0,0.39,0}
\definecolor{ttzzqq}{rgb}{0.2,0.6,0}
\definecolor{ffqqqq}{rgb}{1,0,0}
\definecolor{ttttff}{rgb}{0.2,0.2,1}
\definecolor{uququq}{rgb}{0.25,0.25,0.25}
\begin{tikzpicture}[line cap=round,line join=round,>=triangle 45,x=2.0833333333333335cm,y=2.0833333333333335cm]
\draw[->,color=black] (-1.2,0) -- (1.2,0);
\foreach \x in {-1,1}
\draw[shift={(\x,0)},color=black] (0pt,2pt) -- (0pt,-2pt) node[below] {\footnotesize $\x$};
\draw[->,color=black] (0,-1.2) -- (0,1.2);
\foreach \y in {-1,1}
\draw[shift={(0,\y)},color=black] (2pt,0pt) -- (-2pt,0pt) node[left] {\footnotesize $\y$};
\draw[color=black] (0pt,-10pt) node[right] {\footnotesize $0$};
\clip(-1.2,-1.2) rectangle (1.2,1.2);
\draw [shift={(0,0)},color=qqwuqq,fill=qqwuqq,fill opacity=0.1] (0,0) -- (0:0.21) arc (0:57.13:0.21) -- cycle;
\draw [line width=0.4pt] (0,0) circle (2.08cm);
\draw (0,0)-- (0.54,0.84);
\draw [line width=1.6pt,color=ttttff] (0,0.84)-- (0.54,0.84);
\draw [line width=1.6pt,color=ffqqqq] (0.54,0.84)-- (0.54,0);
\draw [line width=1.6pt,color=ttzzqq] (1,-1.2) -- (1,1.2);
\begin{scriptsize}
\fill [color=uququq] (0,0) circle (1.5pt);
\draw[color=uququq] (-0.11,-0.11) node {$O$};
\fill [color=black] (0.54,0.84) circle (1.5pt);
\draw[color=black] (0.61,0.92) node {$M$};
\fill [color=uququq] (0,0.84) circle (1.5pt);
\draw[color=uququq] (0.03,0.94) node {$A$};
\fill [color=uququq] (0.54,0) circle (1.5pt);
\draw[color=uququq] (0.7,0.04) node {$B$};
\draw[color=ttttff] (0.3,0.77) node {cos(α)};
\draw[color=ffqqqq] (0.74,0.42) node {sin(α)};
\draw[color=ttzzqq] (0.85,0.95) node {tan(α)};
\draw[color=qqwuqq] (0.16,0.07) node {$\alpha$};
\end{scriptsize}
\end{tikzpicture}
\\\\\\
Es ergeben sich folgende (wissenswerte) Werte:

TAAABELLLE


\section{Polarkoordinaten}
adf
\section{Addtions- und Versopplungssätze}
\begin{Bws}
$\cos(a - b) = \cos(a)\cos(b) + \sin(a) \sin(b)$\\
$\sin(a - b) = \sin(a)\cos(b) - \cos(a)\sin(b)$\\
$\cos(a + b) = \cos(a)\cos(b) - \sin(a)\sin(b)$\\
$\sin(a + b) = \sin(a)\cos(b) + \cos(a)\sin(b)$
\end{Bws}

\section{Allgemeine Sinus- und Kosinussätze}

Diese Sätze gelten in einem beliebigen Dreieck:
\begin{Bws}
$\dfrac {a} {sin(\alpha)}=\dfrac {b} {sin(\beta)}=\dfrac {c} {sin(\gamma)}$\\\\
$c²=a²+b²-2abcos(\gamma)$
\end{Bws}\\\\
Winkelnamen!!\\
\definecolor{qqwuqq}{rgb}{0,0.39,0}
\begin{tikzpicture}[line cap=round,line join=round,>=triangle 45,x=0.510204081632653cm,y=0.64516129032258cm]
\clip(-1.3,-5.52) rectangle (10.46,3.78);
\draw [shift={(4.22,1.92)},color=qqwuqq,fill=qqwuqq,fill opacity=0.1] (0,0) -- (-134.14:1.5) arc (-134.14:-47.61:1.5) -- cycle;
\draw [shift={(-0.38,-2.82)},color=qqwuqq,fill=qqwuqq,fill opacity=0.1] (0,0) -- (1.04:1.5) arc (1.04:45.86:1.5) -- cycle;
\draw [shift={(8.4,-2.66)},color=qqwuqq,fill=qqwuqq,fill opacity=0.1] (0,0) -- (132.39:1.5) arc (132.39:181.04:1.5) -- cycle;
\draw (4.22,1.92)-- (8.4,-2.66);
\draw (4.22,1.92)-- (-0.38,-2.82);
\draw (8.4,-2.66)-- (-0.38,-2.82);
\begin{scriptsize}
\draw[color=black] (6.12,-0.44) node {$a$};
\draw[color=black] (1.76,-0.08) node {$b$};
\draw[color=black] (4.06,-2.28) node {$c$};
\draw[color=qqwuqq] (4.68,1.32) node {γ};
\draw[color=qqwuqq] (0.84,-2.5) node {α};
\draw[color=qqwuqq] (7.66,-2.12) node {β};
\end{scriptsize}
\end{tikzpicture}\\
Man bemerkt, dass sich die bekannten Relationen ergeben, wenn man für einen der Winkel den Wert $\dfrac{\pi}{2}$ nimmt.
