\chapter{Komplexe Zahlen}
\chapterauthor{Pascal}


\section{Einführung}

	\paragraph{} Die Ursache für die Einführung der komplexen Zahlen ist vergleichbar mit der jeglicher anderer Zahlenmengen (abgesehen
	von $\N$). Alles basiert auf einer Rechnung oder einer Menge an Rechnungen, welche für die vorhandenen Zahlenmengen \textbf{keine Lösung besitzen} oder aber diese Zahlenmengen einem nicht erlauben eine Lösung zu finden. Ein Beispiel hierfür ist die Einführung der
	negativen Zahlen. Gleichungen der Form $2 + x = 1$ waren eine Zeit lang nicht lösbar. Ebenso galt einmal, dass $x^2 - 2 = 0$ keine
	Lösung besitzt, da hierfür die reellen Zahlen benötigt werden. Analog dazu wird die Erweiterung auf die komplexen Zahlen begründet.
	Dies wird an folgendem klassischen Beispiel erläutert:

	\begin{align*}
		\qquad \qquad x^2 + 1 &= 0 \\
			\Leftrightarrow x &= \sqrt{-1} \quad \text{\Lightning}
	\end{align*}

	\paragraph{} Um dieser und anderen Gleichungen eine Lösung zuzuteilen ist es nicht nur notwendig die Zahlenmenge zu erweitern, sondern
	auch, neue Symbole und Zeichen einzuführen um die neuen Zahlen zu kennzeichnen. Für die $\Z$ ist es das Symbol \dq-\dq, für $\Q$
	\dq,\dq und $\dfrac{a}{b}$, für $\R$ $\sqrt{a}$ und die Zeichen $\pi$ und $e$ zum Beispiel (auch wenn beide sich anders darstellen lassen). (Ein) gewisse(r) Mathematiker (Leibniz glaube ich) hat entschieden, dass das einzige benötigte Zeichen $i$ sein sollte, da er sie imaginäre Zahl nannte (Für den Fall, dass ich hier ein bisschen was durcheinander bringe, sind Direktverbesserungen kommentarlos erlaubt und erwünscht).
 	Und dem war so, weshalb nun gilt:
											$$i \in \C : i^2 = -1; \C \definedBy \R \cup \left\{i\right\}$$

	\begin{Definition}
		Komplexe Zahlen werden standartmäßig mit dem Buchstaben $z$ dargestellt. Die allgemeine Formel einer solchen Zahl lautet:
		\\
		\begin{tikzpicture}
			\draw (0,0) node {" "};
			\draw (7.9, 0) node {$z = x + y \cdot i; x,y \in \R; i \in \C$};
			\draw[line width=0.8pt][decorate,decoration={brace,amplitude=3pt},rotate around={180:(7.6,-0.25)}] (7.6,-0.25) -- (8.1,-0.25);
			\draw[line width=0.8pt][decorate,decoration={brace,amplitude=2pt},rotate around={180:(6.7,-0.25)}] (6.7,-0.25) -- (7,-0.25);
		    \draw[line width=0.4pt] (6.55,-0.5) -- (5.1,-1.4) node[color=blue,left] {$Re(z)$};
		    \draw[line width=0.4pt] (7.35,-0.5) -- (9.2,-1.4) node[color=blue,right] {$Im(z)$};
		\end{tikzpicture}
		\\
		Hierbei wird $x = Re(z)$ Realteil und $y = Im(z)$ und zwar \textbf{nur $y$} Imaginärteil genannt.
	\end{Definition}

	\begin{Bemerkung}
		Eine Zahl $z = x + yi \in \C$ heißt rein imaginäre Zahl für $x = 0$. Analog dazu wird sie für $y = 0$ als reell bezeichnet.
	\end{Bemerkung}


\section{Der Körper der komplexen Zahlen}

	\paragraph{} Da wir nun wissen (oder halt auch nicht), weshalb wir die komplexen Zahlen benötigen, gilt es nun die verschiedenen Verknüpfungen, mit welchen wir zwei komplexe Zahlen verbinden, zu definieren. Wie alle anderen bekannten Mengen (außer $\N$), ist $\C$ teil eines \textbf{Körpers} welcher die zwei Verknüpfungen, denen wir allgemein die Namen \textbf{Addition} und \textbf{Multiplikation} geben. Das 3-Tupel (oder auch Tripel) ($\C$, +, *) ist somit der Körper, mit welchem wir arbeiten werden (jemals gefragt warum keine weiteren Verknüpfungen eingeführt wurden?). Jedoch muss auch erstmal bewiesen werden, dass dieses Tripel ein Körper ist:

	\begin{Beweis}
		\underline{\textbf{Die Addition}}:

		\paragraph{} Für die Addition gilt bekanntlich zu beweisen, dass ($\C$, $\oplus$) eine abelsche oder auch kommutative Gruppe ist:

		\begin{enumerate}[1)]
			\item \begin{align*}
						\forall z_1, z_2 \in \C: z_1 \oplus z_2 &\definedBy (x_1 + y_1i) \oplus (x_2 + y_2i) \\
														   		&= ((x_1 + x_2) + (y_1 + y_2)i) \\
														   		&\in \C \\
						\Rightarrow \forall z_1, z_2 \in \C: \oplus : \C \times \C \rightarrow \C & \quad \text{(Abgeschlossenheit)}
				  \end{align*}
			\item \begin{align*}
				  		(z_1 \oplus z_2) \oplus z_3 &= ((x_1 + x_2) + (y_1 + y_2)i) \oplus (x_3 + y_3i) \\
										  			&= ((x_1 + x_2 + x_3) + (y_1 + y_2 + y_3)i) \\
												    &= (x_1 + y_1i) \oplus ((x_2 + x_3) + (y_2 + y_3)i) \\
												    &= (x_1 + y_1i) \oplus ((x_2 + y_2i) \oplus (x_3 + y_3i)) \\
												    &= z_1 \oplus (z_2 \oplus z_3) \\
						\Rightarrow \forall z_1, z_2, z_3 \in \C: (z_1 + &z_2) \oplus  z_3 = z_1 \oplus (z_2 + z_3) \quad \text{(Assoziativität)}
				  \end{align*}
			\item \begin{align*}
						0 = (0 + 0i) \in \C, \forall z \in \C:& \\
												   z \oplus 0 &= (x + yi) \oplus (0 + 0i) \\
													  		  &= ((x + 0) + (y + 0)i) \\
															  &= (x + yi) \\
															  &= z \\
						\Rightarrow \exists! 0: z \oplus 0 = z; z \in \C \quad &\text{(Neutrales Element)}
				  \end{align*}
			\item \begin{align*}
	  					-z = -(x + yi) \in \C, z \in \C &: \\
								  z \oplus (-z) =& (x + yi) \oplus (-(x + yi)) \\
								  			  =& ((x - x) + (y - y)i) \\
											  =& (0 + 0i) \\
											  =& 0  \\
						\Rightarrow \forall z \in \C \exists -z: z \oplus -z = 0 \quad \text{(Inverses Element)} \qquad &
	  			  \end{align*}
			\item \begin{align*}
						z_1 \oplus z_2 &= (x_1 + y_1i) \oplus (x_2 + y_2i) \\
									   &= ((x_1 + x_2) + (y_1 + y_2)i) \\
									   &= ((x_2 + x_1) + (y_2 + y_1)i) \\
									   &= (x_2 + y_2i) \oplus (x_1 + y_1i) \\
									   &= z_2 \oplus z_1 \\
						\Rightarrow \forall z_1, z_2 \in \C: z_1 \oplus z_2 = z_2 \oplus z_1 \quad \text{(Kom}&\text{mutativität)}
				  \end{align*}
		\end{enumerate}
		\newpage
		\underline{\textbf{Die Multiplikation}}:

		\paragraph{} Für die Multiplikation gilt ebenfalls zu beweisen, dass ($\C$, $\otimes$) eine kommutative Gruppe ist:

		\begin{enumerate}[1)]
			\item \begin{align*}
						\forall z_1, z_2 \in \C: z_1 \otimes z_2 &\definedBy (x_1 + y_1i) \otimes (x_2 + y_2i) \\
														   		 &= ((x_1x_2 - y_1y_2) + (x_1y_2 + x_2y_1)i) \\
														   		 &\in \C \\
						\Rightarrow \forall z_1, z_2 \in \C: \otimes : \C \times \C \rightarrow \C & \quad \text{(Abgeschlossenheit)}
				  \end{align*}
			\item \begin{align*}
				  		(z_1 \otimes z_2) \otimes z_3 &= ((x_1x_2 - y_1y_2) + (x_1y_2 + x_2y_1)i) \otimes (x_3 + y_3i) \\
										  			  &= ((x_1x_2x_3 - y_1y_2x_3 - x_1y_2y_3 - x_2y_1y_3) + (x_1x_2y_3 - y_1y_2y_3 + x_1y_2x_3 + x_2y_1x_3)i) \\
												      &= (x_1 + y_1i) \otimes ((x_2x_3 - y_2y_3) + (x_2y_3 + x_3y_2)i) \\
												      &= (x_1 + y_1i) \otimes ((x_2 + y_2i) \otimes (x_3 + y_3i)) \\
												      &= z_1 \otimes (z_2 \otimes z_3) \\
						\Rightarrow \forall z_1, z_2, z_3 \in \C: (z_1 + &z_2) \otimes  z_3 = z_1 \otimes (z_2 + z_3) \quad \text{(Assoziativität)}
				  \end{align*}
			\item \begin{align*}
						1 = (1 + 0i) \in \C, \forall z \in \C:& \\
												   z \otimes 1 &= (x + yi) \otimes (1 + 0i) \\
													  		   &= ((1 \cdot x - 0 \cdot y) + (1 \cdot y + 0 \cdot x)i) \\
															   &= (x + yi) \\
															   &= z \\
						\Rightarrow \exists! 1: z \otimes 1 = z; z \in \C \quad &\text{(Neutrales Element)}
				  \end{align*}
			\item \begin{align*}
	  					z^{-1} = (x + yi)^{-1} \in \C, z \in \C:& \\
								  z \otimes z^{-1} &= (x + yi) \otimes \dfrac{1}{(x + yi)} \\
								  			  	   &= \dfrac{(x + yi) \cdot (x - yi)}{(x + yi) \cdot (x - yi)} \\
												   &= \dfrac{((x^2 + y^2) + (0i))}{((x^2 + y^2) + (0i))} \\
											  	   &= (1 + 0i) \\
											  	   &= e \\
						\Rightarrow \forall z \in \C \exists z^{-1}: z \otimes z^{-1} = 1 \quad \text{(Inverses Element)} \qquad &
	  			  \end{align*}
			\item \begin{align*}
						z_1 \otimes z_2 &= (x_1 + y_1i) \otimes (x_2 + y_2i) \\
									    &= ((x_1x_2 - y_1y_2) + (x_1y_2 + x_2y_1)i) \\
									    &= ((- y_1y_2 + x_1x_2) + (x_2y_1 + x_1y_2)i) \\
									    &= (x_2 + y_2i) \otimes (x_1 + y_1i) \\
									    &= z_2 \otimes z_1 \\
						\Rightarrow \forall z_1, z_2 \in \C: z_1 \otimes z_2 = z_2 \otimes z_1 \quad \text{(Kom}&\text{mutativität)}
				  \end{align*}
		\end{enumerate}
	\end{Beweis}

	\newpage

\section{Die Gauß'sche Zahlenebene}


	\subsection{Hinführung}

		\paragraph{} Da jetzt der Körper der komplexe Zahlen definiert wurde, gilt es ihn nun in unser bekanntes Zahlensystem zu integrieren.
		Hierfür kann man zum Beispiel beobachten, wie sich komplexe Zahlen auf der \textbf{reellen Zahlengerade} (welche eine Veranschaulichung des euklidischen Vektorraums $\R^1$ ist) verhalten. Nun sollte erstmal der Begriff des \textbf{Zeiger}s eingeführt werden.
		\\
		\begin{Definition}
			Als Zeiger definieren wir ein visuelles Hilfsmittel, welches dem Darstellungsmodul eines Vektors, einem Pfeil, sehr ähnlich ist, jedoch sich durch eine kleine Nuance von Letzterem zu unterscheiden ist.
		\end{Definition}
		\\
		\begin{center}
			\begin{tikzpicture}[>=triangle 45]
				\draw[->,line width=0.8pt] (-5.5,0) -- (6,0) node[above] {$Re(z)$};
				\foreach  \x in {-5,...,5}
	    			\draw (\x,0.2) -- (\x,-0.2) node[below] {$\x$};
				\draw[->,line width=0.8pt,color=red] (0,0) -- (3,0);
			\end{tikzpicture}
		\end{center}
		\\
		\paragraph{} Zunächst muss verstanden werden, welche Transformation jeweils Addition und Multiplikation im $\R^1$ auf reelle Zahlen ausführt. Das lässt sich am einfachsten anhand einer Graphik erklären:
		\\
		\begin{center}
			\begin{tikzpicture}[>=triangle 45]
				\draw[->,line width=0.8pt] (-5.5,0) -- (6,0) node[above] {$Re(z)$};
				\foreach  \x in {-5,...,5}
	    			\draw (\x,0.2) -- (\x,-0.2) node[below] {$\x$};
				\draw[->,line width=1.5pt,color=red] (0,0) -- (-3,0);
				\draw[->,line width=1.5pt,color=orange] (0,0) -- (4,0);
				\draw[->,line width=1.5pt,color=red,dashed] (0,0) -- (3,0);
				\draw[->,line width=1.5pt,color=orange,dashed] (0,0) -- (2,0);
				\draw[line width=1.3pt,color=red] (1.5,0.4) node {3};
				\draw[line width=1.3pt,color=red] (-1.5,0.4) node {-3};
				\draw[->,color=red] (2.4,0) arc (0:180:2.4);
				\draw[line width=2pt,color=white] (1.84,1.54) arc (40:55:2.4);
				\draw[->,color=orange] (0.3,0) arc (180:0:1.5);
				\draw[line width=0.8pt,color=orange] (1.8,1.8) node {$\cdot(2) = $ Skalierung mit Faktor $2$};
				\draw[line width=0.8pt,color=red] (0,2.7) node {$\cdot (-1) = 180 \degree$ Drehung};
			\end{tikzpicture}
		\end{center}

		\paragraph{} Da gilt, dass: $i^2 = -1$ ist die Multiplikation einer reellen Zahl mit $i^2$ äquivalent zu einer $180$ - Grad Drehung. Daraus lässt sich schließen, dass die Multiplikation mit $i$ einer $90$ - Grad Drehung entspricht, da: $a \cdot i^2 = a \cdot i \cdot i; a\in\R$ und man davon ausgeht, dass $\cdot(i)$ beides mal dieselbe Transformation anwendet. Somit entsteht eine wunderschöne Ebene, in der alle Zahlen zusammen spielen und sich amüsieren können:
		\\
		\begin{center}
			\begin{tikzpicture}[>=triangle 45]
				\draw[->,line width=0.8pt] (-5.5,0) -- (5.7,0) node[above] {$Re(z)$};
				\draw[->,line width=0.8pt] (0,-5.5) -- (0,5.7) node[right] {$Im(z)$};
				\foreach  \x in {-5,...,-1,1,2,...,5}
	                \draw [line width=0.8pt] (\x,0.1) -- (\x,-0.1) node[below] {$\x$};
	            \foreach  \x in {-5,...,-1,1,2,...,5}
	                \draw [line width=0.8pt] (-0.1,\x) -- (0.1,\x) node[right] {$\x i$};
				\draw[line width=0.8pt] (-0.2,-0.2) node {$O$};
				\draw[->,line width=1.5pt,color=red] (0,0) -- (-3,0);
				\draw[->,line width=1.5pt,color=red,dashed] (0,0) -- (3,0);
				\draw[line width=1.3pt,color=red] (1.5,0.3) node {3};
				\draw[line width=1.3pt,color=red] (-1.5,0.3) node {-3};
				\draw[->,line width=1.5pt,color=red] (0,0) -- (0,-3);
				\draw[->,line width=1.5pt,color=red] (0,0) -- (0,3);
				\draw[line width=1.3pt,color=red] (-0.3,1.5) node {$3i$};
				\draw[line width=1.3pt,color=red] (-0.4,-1.5) node {$-3i$};
				\draw[->,color=red] (2.4,0) arc (0:90:2.4);
				\draw[->,color=red] (0,2.4) arc (90:180:2.4);
				\draw[line width=1.3pt,color=red] (1.9,2) node {$\cdot(i)$};
				\draw[line width=1.3pt,color=red] (-1.9,2) node {$\cdot(i)$};
				\draw[->,color=red] (2.4,0) arc (360:270:2.4);
				\draw[->,color=red] (0,-2.4) arc (270:180:2.4);
				\draw[line width=1.3pt,color=red] (1.9,-2) node {$\cdot(-i)$};
				\draw[line width=1.3pt,color=red] (-1.9,-2) node {$\cdot(-i)$};
			\end{tikzpicture}
		\end{center}

		\paragraph{} Eine komplexe zahl $z = x + yi$ kann somit über einen Zeiger in der Gauß'schen Zahlenebene dargestellt werden. Für $z = 2 + 3i$
		gilt beispielsweise:

		\begin{center}
			\begin{tikzpicture}[>=triangle 45]
				\draw[->,line width=0.8pt] (-5.5,0) -- (5.7,0) node[above] {$Re(z)$};
				\draw[->,line width=0.8pt] (0,-5.5) -- (0,5.7) node[right] {$Im(z)$};
				\foreach  \x in {-5,...,-1,1,2,...,5}
	                \draw [line width=0.8pt] (\x,0.1) -- (\x,-0.1) node[below] {$\x$};
	            \foreach  \x in {-5,...,-1,1,2,...,5}
	                \draw [line width=0.8pt] (-0.1,\x) -- (0.1,\x) node[right] {$\x i$};
				\draw[line width=0.8pt] (-0.2,-0.2) node {$O$};
				\draw[->,line width=1.5pt,color=red,dashed] (0,0) -- (2,0);
				\draw[->,line width=1.5pt,color=red,dashed] (2,0) -- (2,3);
				\draw[->,line width=1.5pt,color=red] (0,0) -- (2,3);
				\draw[line width=1.3pt,color=red] (1,0.3) node {$2$};
				\draw[line width=1.3pt,color=red] (1.65,0.9) node {$+$};
				\draw[line width=1.3pt,color=red] (2.3,1.5) node {$3i$};
				\draw[line width=1.3pt,color=red] (0.85,2.35) node {$2 + 3i$};
			\end{tikzpicture}
		\end{center}

	\subsection{Rechenoperationen}

		\paragraph{} Die bereits angesprochenen basischen Rechenoperationen lassen sich in der Gauß'schen Zahlenebene als Transformation darstellen. Weshalb dies interessant ist, wird im Verlauf dieses Abschniits hoffentlich noch deutlich.
		\paragraph{} Die Addition ist die einfachste Transformation, weil sie equivalent zur Linearkombination bei Vektoren ist, da gilt: $$(z_1 + z_2) = ((x_1 + x_2) + (y_1 + y_2)i)$$ Bildlich bedeutet das:

		\begin{center}
			\begin{tikzpicture}[>=triangle 45]
				\draw[->,line width=0.8pt] (-5.5,0) -- (5.7,0) node[above] {$Re(z)$};
				\draw[->,line width=0.8pt] (0,-5.5) -- (0,5.7) node[right] {$Im(z)$};
				\foreach  \x in {-5,...,-1,1,2,...,5}
	                \draw [line width=0.8pt] (\x,0.1) -- (\x,-0.1) node[below] {$\x$};
	            \foreach  \x in {-5,...,-2,2,3,...,5}
	                \draw [line width=0.8pt] (-0.1,\x) -- (0.1,\x) node[right] {$\x i$};
				\draw[line width=0.8pt] (-0.1,1) -- (0.1,1) node[right] {$i$};
				\draw[line width=0.8pt] (-0.1,-1) -- (0.1,-1) node[right] {$-i$};
				\draw[line width=0.8pt] (-0.2,-0.2) node {$O$};
				\draw[->,line width=1.5pt,color=black,dashed] (0,0) -- (5,0);
				\draw[->,line width=1.5pt,color=black,dashed] (5,0) -- (5,4);
				\draw[->,line width=1.5pt,color=orange,dashed] (0,0) -- (3,0);
				\draw[->,line width=1.5pt,color=red,dashed] (0,0) -- (2,0);
				\draw[->,line width=1.5pt,color=red,dashed] (2,0) -- (2,3);
				\draw[->,line width=1.5pt,color=orange,dashed] (3,0) -- (3,1);
				\draw[->,line width=1.5pt,color=red] (0,0) -- (2,3);
				\draw[->,line width=1.5pt,color=orange] (0,0) -- (3,1);
				\draw[->,line width=1.5pt,color=black] (0,0) -- (5,4);
				\draw[line width=1.3pt,color=red] (1.5,0.3) node {$2$};
				\draw[line width=1.3pt,color=red] (2.3,1.5) node {$3i$};
				\draw[line width=1.3pt,color=red] (2.2,3.2) node {$2 + 3i$};
				\draw[line width=1.3pt,color=orange] (2.3,0.3) node {$3$};
				\draw[line width=1.3pt,color=orange] (3.3,0.5) node {$i$};
				\draw[line width=1.3pt,color=orange] (3.2,1.2) node {$3 + i$};
				\draw[line width=1.3pt,color=black] (4.2,0.3) node {$2 + 3 = 5$};
				\draw[line width=1.3pt,color=black] (6,2) node {$3i + i = 4i$};
				\draw[line width=1.3pt,color=black] (5.2,4.2) node {$5 + 4i$};
			\end{tikzpicture}
		\end{center}

		\paragraph{} Im Gegensatz dazu ist die Multiplikation weniger intuitiv. Aus logistischen Gründen wird sie trotz allem in diesem Abschnitt besprochen. Bildlich stellt sie eine Drehstreckung dar (auch dies wird gegen Ende des Kapitels genauer beschrieben), da gilt: $$(z_1 \cdot z_2) = r_1_{\alpha} \cdot r_2_{\beta} = (r_1 \cdot r_2)_{\alpha + \beta}$$

		\begin{center}
			\begin{tikzpicture}[>=triangle 45]
				\draw[line width=0.8pt,color=black,fill=black,fill opacity=0.1] (0,0) -- (4,0) arc (0:135:4) -- cycle;
				\draw[line width=0.8pt,color=orange,fill=orange,fill opacity=0.3] (0,0) -- (1.2,0) arc (0:116.565:1.2) -- cycle;
				\draw[line width=0.8pt,color=red,fill=red,fill opacity=0.3] (0,0) -- (2,0) arc (0:18.435:2) -- cycle;
				\draw[->,line width=0.8pt] (-5.5,0) -- (5.7,0) node[above] {$Re(z)$};
				\draw[->,line width=0.8pt] (0,-5.5) -- (0,5.7) node[right] {$Im(z)$};
				\foreach  \x in {-5,...,-1,1,2,...,5}
	                \draw [line width=0.8pt] (\x,0.1) -- (\x,-0.1) node[below] {$\x$};
	            \foreach  \x in {-5,...,-2,2,3,...,5}
	                \draw [line width=0.8pt] (-0.1,\x) -- (0.1,\x) node[right] {$\x i$};
				\draw[line width=0.8pt] (-0.1,1) -- (0.1,1) node[right] {$i$};
				\draw[line width=0.8pt] (-0.1,-1) -- (0.1,-1) node[right] {$-i$};
				\draw[line width=0.8pt] (-0.2,-0.2) node {$O$};
				\draw[->,line width=1.5pt,color=red] (0,0) -- (3,1);
				\draw[->,line width=1.5pt,color=orange] (0,0) -- (-1,2);
				\draw[->,line width=1.5pt,color=black] (0,0) -- (-5,5);
				\draw[line width=1.3pt,color=red] (3.2,1.2) node {$3 + i$};
				\draw[line width=1.3pt,color=orange] (-1.2,2.2) node {$-1 + 2i$};
				\draw[line width=1.3pt,color=black] (-5.2,5.2) node {$-5 + 5i$};
				\draw[line width=0.8pt,color=red] (1.5,0.25) node {$\alpha$};
				\draw[line width=0.8pt,color=orange] (0.5,0.6) node {$\beta$};
				\draw[line width=0.8pt,color=black] (1.5,2) node {$\alpha + \beta$};
				\draw[line width=0.8pt,color=black] (-3.3,2.2) node {$r = r_1 \cdot r_2$};
			\end{tikzpicture}
		\end{center}
