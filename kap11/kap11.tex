\chapter{Integrale}
\section{Stammfunktionen}
Eine Funktion $F$ heißt Stammfunktion der Funktion $f$, wenn $F'(x)=f(x)$
gilt.

Ist $F$ irgendeine Stammfunktion von $f$, dann ist auch $F(x)+C$ (mit
konstantem $C$) eine Stammfunktion, denn beim Ableiten fällt ja $C$ als
konstanter Summand weg. Jede Funktion hat also unendlich viele
Stammfunktionen, die sich aber nur um einen konstanten Summanden
unterscheiden.

\section{Begriff des Integrals}

\begin{figure}[htbp]\centering
  \includegraphics{math3.eps}
  \caption{Integral und Flächeninhalt}
  \label{fig:6}
\end{figure}

In Abb.~\ref{fig:6} soll die Fläche zwischen der x-Achse und der Funktion
$y=f(x)$ zwischen $x=a$ und $x=b$ berechnet werden.

Um eine Näherung für diese Fläche zu bekommen, teilt man sie in Streifen auf
(in der Abb.~\ref{fig:6} sind es 6 Streifen) und nimmt die Summe der
Flächeninhalte der Rechtecke als Näherung für die Fläche. Wenn man die Zahl
der Streifen so erhöht, dass die Breite jedes Streifens gegen Null geht, dann
bekommt man als Grenzwert die gesuchte Fläche. In Abb.~\ref{fig:6} ist die
Näherung:
\[
A_6=y_1(x_1-x_0)+y_2(x_2-x_1)+y_3(x_3-x_2)
+y_4(x_4-x_3)+y_5(x_5-x_4)+y_6(x_6-x_5)
\]
Beachtet man, dass $y_k=f(x_k)$ ist und schreibt man $\Delta x_1=x_1-x_0$,
$\Delta x_2=x_2-x_1$ usw., dann bekommt man für $n$ Streifen:
\[
A_n=f(x_1)\Delta x_1+f(x_2)\Delta x_2+\cdots+f(x_n)\Delta x_n=
\sum_{k=1}^n f(x_k)\Delta x_k
\]

Der Grenzwert dieser $A_n$ ist nun die gesuchte Fläche:
\begin{equation}
  \label{eq:33}
  \int_a^b \dD y=
  \int_a^b f(x)\dD x=\lim_{n\to\infty}\sum_{k=1}^n f(x_k)\Delta x_k
\end{equation}
Beim Grenzübergang ist nur zu beachten, dass mit Erhöhung der Streifenzahl $n$
auch alle Breiten $\Delta x_k$ gegen Null gehen, also
$\lim_{n\to\infty}\max\Delta x_k=0$ gilt.

Schaut man sich die Abb.~\ref{fig:6} an, dann stellt man fest, dass statt
$f(x_k)$ als Seite der Rechtecke auch Rechtecke verwendet werden könnten,
deren Seite irgendein $f(u_k)$ ist, wobei $x_{k-1}\le u_k\le x_k$ ist.

Grenzwerte der Art von Gl.~\eqref{eq:33} heißen \emph{Integrale}. Funktionen,
für die dieser Grenzwert für jede Art der Unterteilungsfolge und jede Wahl der
$f(u_k)$ existiert und immer denselben Wert ergibt, heißen \emph{integrierbar}.

Als Physiker sagt man, durch Integration werden Elemente der Form  $\dD
y=f(x)\dD x$
aufaddiert.

\section[Der Hauptsatz]{Der Hauptsatz der Differential- und
  Integralrechnung}
Die Bestimmung der Grenzwerte in Gleichung~\eqref{eq:33} ist eine mühselige
Angelegenheit. Aber alles wendet sich zum Guten, denn Integrale können mittels
Stammfunktionen bestimmt werden.

Um dies klar zu machen betrachten wir irgendeine Stammfunktion $F$ der
Randfunktion $f$. Nach der Grundformel~\eqref{eq:34} ist dann
\[
\dD F(x)=F'(x)\dD x=f(x)\dD x\Rpf
\Delta F(x_k)\approx f(x_k)\Delta x_k\Rpf
F(x_k)-F(x_{k-1})\approx f(x_k)\Delta x_k
\]
Die Näherung ist umso besser je kleiner $\Delta x_k$ ist.

Wendet man diese Beziehung auf die Näherungssumme $A_n$ an, dann bekommt man:
\[
A_n\approx(F(x_1)-F(x_0))+(F(x_2)-F(x_1))+\cdots (F(x_n)-F(x_{n-1})
\]
In dieser Gleichung heben sich alle Summanden bis auf $F(x_0)$ und $F(x_n)$
heraus, also ist, weil $x_0=a$ und $x_n=b$ ist, $A_n\approx F(b)-F(a)$.

Diese Näherung wird umso besser, je kleiner die $\Delta x$ werden, also hat
man für den Grenzwert:
\begin{merkbox}
\begin{equation}
  \label{eq:35}
  \int_a^b f(x)\dD x
  =\lim_{n\to\infty}\sum_{k=1}^n f(x_k)\Delta x_k = F(b) - F(a)
  =\left[F(x)\right]_a^b
\end{equation}
\end{merkbox}
Dies ist der \textbf{Hauptsatz der Differential- und Integralrechnung}. Mit
ihm wird das Problem der Integration auf das Auf"|finden einer Stammfunktion
zurückgeführt. Als Mathematiker müsste man noch anfügen, dass diese Beziehung
nicht immer gilt, sondern nur unter gewissen Voraussetzungen.

Der Hauptsatz ist sicher dann richtig, wenn die Funktion $f$ stetig ist.
In diesem Fall kann man auch so formulieren:
\[
I_a(x)=\int_a^x f(\xi)\dD \xi \qquad\Rpf\qquad I_a'(x)=f(x)
\]
Die Funktion $I_a$ nennt man hier die Integralfunktion. Man kann sie sich
vorstellen als die Funktion die jeder Stelle $x$ die Fläche zwischen $a$ und
$x$ zuordnet. Unter den obigen Voraussetzungen ist also die Ableitung der
Integralfunktion gleich der Randfunktion. Ist die Randfunktion $f$ nicht im
ganzen Intervall stetig, sondern nur ">stückweise stetig"<, dann ist zwar die
Integralfunktion immer noch stetig, aber in den Trennpunkten der
Stetigkeitsbereiche nicht mehr notwendig differenzierbar.

Häufig kommen auch sogenannte \emph{unbestimmte Integrale} vor. Ein
unbestimmtes Integral ist einfach eine Stammfunktion. Man schreibt dann
\[
\int f(x)\dD x= F(x)+C
\]
und nennt $C$ die \emph{Integrationskonstante}.

Eine Stammfunktion zu finden, ist allerdings häufig auch nicht sehr einfach.
Für viele aus elementaren Funktionen zusammengesetzte Funktionen bekommt man
leider Stammfunktionen, die nicht mehr elementar dargestellt werden können. Das gilt
schon für so einfach aussehende Integrale wie
\[
\int \frac{\sin x}{x}\dD x
\]

Da $\sin x /x$ in ihrem Definitionsbereich stetig ist, muss sie nach dem
Hauptsatz eine Stammfunktion haben. Sie lässt sich allerdings nicht mehr durch
elementare Funktionen ausdrücken, sondern gibt eine ">neue"< sogenannte
">höhere"< Funktion, die man Integralsinus nennt. Ihre Funktionswerte kann man
mit Tabellen oder mit Computerprogrammen bestimmen. Zunächst mag man meinen,
dies sei etwas besonderes, aber auch die Werte eines Logarithmus oder eines
Sinus kann man nur mit dem Taschenrechner berechnen oder aus einer
Funktionentafel entnehmen.

\section{Integrationsregeln}

Da Integration die Umkehrung des Ableitens ist (">Aufleiten"<) übertragen sich
die Ableitungsregeln auf das Integrieren:
\begin{itemize}
\item Konstante Faktoren bleiben stehen.
\item Summen werden summandenweise aufgeleitet.
\item $\D\int_a^b=\int_a^c+\int_c^b$
\item $\D \int_a^b=-\int_b^a$
\item Es gibt \emph{keine direkte} Umkehrung der Produkt-, Quotienten und
  Kettenregel, sondern hier muss man kompliziertere Verfahren verwenden.
\end{itemize}

Wichtige Integrale sind:
\begin{align}
  \label{eq:36}
  \int x^n\dD x&=\frac{x^{n+1}}{n+1}\\
  \int\sin x\dD x&= -\cos x\label{eq:37}\\
  \int\cos x\dD x&= \sin x\label{eq:38}\\
  \int\frac{\dD x}{\cos^2x}&= \tan x\label{eq:39}\\
  \int\frac{1}{x}\dD x&= \ln|x|\label{eq:40}\\
  \int\frac{\dD x}{1+x^2}&= \arctan x\label{eq:41}
\end{align}

In Formelsammlungen stehen viele Integrale, es gibt auch Integraltafeln, die
hunderte von Integralen fertig angeben. CAS-Programme können alle sehr gut
integrieren.

Aus der Formel~\eqref{eq:35} erkennt man, dass das Integral negativ wird, wenn
im betrachteten Intervall $f(x)<0$ ist. Dann verläuft die Kurve unterhalb der
x-Achse, und man muss als Fläche den Betrag nehmen. Integrale sind aber nicht
nur Flächen, mit ihnen kann man Volumina, Energien, Ladungen und und und
\ldots ausrechnen, und bei vielen dieser Größen sind negative Werte durchaus
sinnvoll!

\subsection{Produktintegration}
Nach der Produktregel ist $(uv)'=u'v+uv'$. Integriert man nun beide Seiten,
und beachtet, dass nach dem Hauptsatz $\int (uv)'\dD x=uv$ ist, dann bekommt
man die Formel für die Produktintegration, die auch \emph{partielle
  Integration} genannt wird.
\begin{equation}
  \label{eq:42}
  \int u'v\dD x= uv - \int uv'\dD x
\end{equation}

Mit dieser Regel kann man \zB Integrale folgender Form lösen:
\[
\int x\sin x\dD x
\]
Setzt man hier $u'=\sin x$, $v=x$, dann ist $u=-\cos x$ und $v'=1$. Setzt man
diese Werte in die Produktintegrations-Formel ein, dann folgt:
\begin{align*}
\int x\sin x\dD x&= (-\cos x)x-\int (-\cos x)\cdot1\dD x\\
&=-x\cos x+\int\cos x\dD x= -x\cos x+\sin x
\end{align*}

Ein anderes schönes Beispiel ist die Stammfunktion von $\ln x$. Dazu fasst man
$\ln x$ als $1\cdot\ln x$ auf und wählt $u'=1$, $v=\ln x$, also $u=x$ und
$v'=1/x$ und erhält:
\[
\int\ln x\dD x=x\ln x-\int x\cdot\frac{1}{x}\dD x=
x\ln x-\int 1\dD x=x\ln x -x
\]

\subsection{Substitutionsmethode}
Es soll die Stammfunktion von $x(1-x^2)^9$ bestimmt werden. Hier wäre zum
Ableiten die Kettenregel und die Produktregel nötig, beides gibt es aber in
reiner Form beim Integrieren nicht. Man kann aber häufig durch geschickte
Substitution das Integral auf eine Form bringen, für die dann eine
Stammfunktion angegeben werden kann. Der wesentliche Punkt dabei ist, dass
nicht nur $x$ sondern auch $\dD x$ der Substitution unterworfen werden muss.

In diesem Beispiel setzen wir $u=1-x^2$. Dann ist $\dD u=u'(x)\dD x=-2x\dD x$.
Somit ist $\dD x=-\dD u/2x$. Nun geht's los:
\[
\int x(1-x^2)^9\dD x=\int x u^9 \frac{-1}{2x}\dD u
=-\frac{1}{2}\int u^9\dD u=-\frac{1}{20}u^{10}=-\frac{1}{20}(1-x^2)^{10}
\]

Hätten wir hier die Stammfunktion nicht benötigt, sondern nur den Wert eines
bestimmten Integrals, dann hätte man sich die Rücksetzung sparen können, indem
man die Grenzen auf die neue Variable umschreibt:
\[
\int_{x=0}^1 x(1-x^2)^9\dD x=-\frac{1}{2}\int_{u=1}^0u^9\dD u
=-\frac{1}{20}\left[u^{10}\right]_1^0=-\frac{1}{20}(0-1)=\frac{1}{20}
\]

In diesem Beispiel haben wir einen Teilausdruck des Integrals gleich $u$
gesetzt. Im folgenden Beispiel machen wir es anders herum und setzen $x$
gleich einer Funktion von $u$. Es soll die Fläche einer Ellipse berechnet
werden. Für den oberen Ellipsenbogen gilt die Gleichung (Auflösen der
Mittelpunktsform~\eqref{eq:7} nach $y$):
\[
y=\frac{b}{a}\sqrt{a^2-x^2}
\]
Die Fläche unter dem oberen Ellipsenbogen ist die halbe Ellipsenfläche und
berechnet sich so:
\[
A=\int\limits_{x=-a}^a \frac{b}{a}\sqrt{a^2-x^2}\,\dD x=
\int\limits_{u=-\frac{\pi}{2}}^{\frac{\pi}{2}}
\frac{b}{a}\sqrt{a^2-a^2\sin^2u}\,a\cos u\,\dD u=
\int\limits_{-\frac{\pi}{2}}^{\frac{\pi}{2}}ab\cos^2u\,\dD u
\]
Dabei hat man $x=a\sin u$ gesetzt, dann wird $\dD x=a\cos u \dD u$. Nun muss
man nur noch die Stammfunktion von $\cos^2u$ ermitteln. Dazu kann man etwa die
Beziehung $\cos2u=2\cos^2u-1$ heranziehen, also
$\cos^2u=\frac{1}{2}(1+\cos2u)$. Diese hat die Stammfunktion
$\frac{1}{2}(u+\sin u\cos u)$. Damit bekommt man für die Fläche:
\[
A=\frac{ab}{2}\left[u+\sin u\cos u\right]_{-\frac{\pi}{2}}^{+\frac{\pi}{2}}=
\frac{ab}{2}[(\frac{\pi}{2}+0)-(-\frac{\pi}{2}+0)]=\frac{1}{2}\pi a b
\]
Die gesamte Fläche der Ellipse ist doppelt so groß, also $\pi a b$.

\subsection{Partialbruchzerlegung}
Die Partialbruchzerlegung ist ein allgemeines Verfahren zur Integration
gebrochen rationaler Funktionen. Es sei gegeben:
\begin{equation}
  \label{eq:68}
  f(x)=\frac{P(x)}{Q(x)}
  =\frac{a_0+a_1x+\dots+a_nx^n}{b_0+b_1x+\dots+b_mx^m}
\end{equation}
Im Falle $m\le n$ führt man zuerst eine Polynomdivision aus, dann bekommt man
eine ganzrationale Funktion und eine gebrochen rationale mit $m>n$, so dass
wir uns für das Weitere auf den Fall $m>n$ beschränken können.

Das Nennerpolynom kann im Komplexen vollständig in Linearfaktoren zerlegt
werden, im Reellen bleiben ggf. quadratische Faktoren übrig, die keine reellen
Nullstellen mehr haben. Seien nun $x_k$ die Nullstellen von $Q(x)$ mit der
Vielfachheit $\nu_k$ dann kann man $Q(x)$ stets auf die folgende Form bringen:
\begin{equation}
  \label{eq:69}
  Q(x)=b_m(x-x_1)^{\nu_1}(x-x_2)^{\nu_2}\dots(x-x_k)^{\nu_k}
  (x^2+\alpha_1x+\beta_1)^{\mu_1}\dots(x^2+\alpha_rx+\beta_r)^{\mu_r}
\end{equation}
Die gebrochen rationale Funktion $f(x)$ lässt sich nun als Summe von
Partialbrüchen schreiben. Dabei gilt:
\begin{enumerate}
\item Jeder Linearfaktor $(x-x_k)$ der Vielfachheit $\nu_k$ trägt zur
  Zerlegung die folgenden Summanden bei:
  \[
  \frac{A_1}{(x-x_k)}+\frac{A_2}{(x-x_k)^2}+\dots
  +\frac{A_{\nu_k}}{(x-x_k)^{\nu_k}}
  \]
\item Jeder quadratische Faktor mit der Vielfachheit $\mu_k$ trägt die
  folgenden Summanden bei:
  \[
  \frac{C_1x+D_1}{(x^2+\alpha x+\beta)}+
  \frac{C_2x+D_2}{(x^2+\alpha x+\beta)^2}+\dots+
  \frac{C_{\mu_k}x+D_{\mu_k}}{(x^2+\alpha x+\beta)^{\mu_k}}
  \]
\end{enumerate}

\noindent Die Linearfaktoren lassen sich sofort integrieren, bei den
quadratischen geht man so vor:
\[
(x^2+\alpha x+\beta)=\bigl(x+\tfrac{\alpha}{2}\bigr)^2+\delta^2
=(\delta y)^2+\delta^2=\delta^2(y^2+1)
\]
Hier hat man quadratisch ergänzt und $\delta\cdot y=x+\frac{\alpha}{2}$
substituiert. Mit dieser Substitution ist
\[
x=\delta y-\frac{\alpha}{2}\qquad y=\frac{x+\frac{\alpha}{2}}{\delta}\qquad
\dD x=\delta\dD y
\]
Nun wird das Integral zu:
\begin{align*}
  \int\frac{Cx+D}{(x^2+\alpha x+\beta)^\mu}\dD x
  &=\frac{1}{\delta^{2\mu-1}}
  \int\frac{C\delta y+D-\frac12C\alpha}{(y^2+1)^\mu}\dD y\\
  &=\frac{C}{\delta^{2\mu-2}}\int\frac{y\dD y}{(y^2+1)^\mu}
  +\frac{D-\frac12C\alpha}{\delta^{2\mu-1}}\int\frac{\dD y}{(y^2+1)^\mu}
\end{align*}
Für das erste der beiden verbleibenden Integrale bekommt man dann
\begin{equation}
  \label{eq:70}
  \int\frac{y\dD y}{(y^2+1)^\mu}=\frac12\int\frac{\dD u}{u^\mu}=
  \begin{cases}
    \dfrac{u^{1-\mu}}{2(1-\mu)}=&\text{für~}\mu=2, 3, \dots\\
    \frac{1}{2}\ln|u|& \text{für~} \mu=1
  \end{cases}
\end{equation}
Hier hat man $u=y^2+1$ gesetzt. Nun muss noch das zweite bestimmt werden.
\[
J_\mu=\int\frac{\dD y}{(y^2+1)^\mu}
\]
Für $\mu=1$ ist es bekannt, $J_1=\arctan y$.

Wir werden eine Rekursionsformel für die $J_\mu$ herleiten. Wir beginnen mit
der Beziehung
\[
J_{\mu+1}=\int\frac{(y^2+1)-y^2}{(y^2+1)^{\mu+1}}\dD y
=J_\mu-\int\frac{y^2}{(y^2+1)^{\mu+1}}\dD y
\]
Nun denken wir uns den zweiten Integranden als $y\cdot$Rest geschrieben und
integrieren partiell:
\[
\int y\cdot\frac{y\dD y}{(y^2+1)^{\mu+1}}=
\frac{-y}{2\mu(y^2+1)^\mu}+\frac{1}{2\mu}\int\frac{\dD y}{(y^2+1)^\mu}=
\frac{-y}{2\mu(y^2+1)^\mu}+\frac{1}{2\mu}\,J_\mu
\]
Setzt man das oben ein, so bekommt man die Rekursionsformel:
\begin{equation}
  \label{eq:71}
  J_{\mu+1}=\left(1-\frac{1}{2\mu}\right)J_\mu+\frac{y}{2\mu(y^2+1)^\mu}
\end{equation}
Insbesondere ist dann (rechne das nach!)
\begin{align*}
  J_1&=\arctan y\\
  J_2&=\frac{1}{2}\arctan y+\frac{y}{2(y^2+1)}\\
  J_3&=\frac38\left(\arctan y+\frac{y}{y^2+1}\right)+\frac{y}{4(y^2+1)^2}\\
  J_4&=\frac{5}{16}\left(\arctan y+\frac{y}{y^2+1}\right)+
       \frac{5}{24}\cdot\frac{y}{(y^2+1)^2}+\frac{y}{6(y^2+1)^3}
\end{align*}

\section{Beispiele zur Integration}
\NA Man bestimme die Fläche unter der Parabel $y=2x^2$ im Bereich $1\le x\le
4$.
\[
\int_1^4 2x^2\dD x=\left[\frac{2}{3}x^3\right]_1^4=\frac{2}{3}(64-1)=42
\]

\NA Bestimme die Formel für das Volumen einer Pyramide der Höhe $h$ und der
Grundfläche $G$.

Man stellt sich die Pyramide aus dünnen Schichten der Dicke $\dD x$ parallel
zur Grundfläche aufgebaut vor. Die x-Achse legt man durch die Spitze der
Pyramide, so dass die Höhe auf die x-Achse zu liegen kommt. Da die Flächen der
Schichten dann durch zentrische Streckung aus der Grundfläche entstehen, hat
eine Schicht im Abstand $x$ von der Spitze die Fläche $A(x)$, für die nach dem
Strahlensatz gilt:
\[
\frac{x^2}{h^2}=\frac{A(x)}{G}\quad\Rpf\quad A(x)=\frac{G}{h^2} x^2
\]
Das Volumen einer Schicht ist dann $\dD V=A(x)\dD x$. Das Volumen der Pyramide
ist die Summe der Volumina aller Schichten, diese Summe bekommt man durch
Integration:
\[
V=\int_0^h \dD V=\int_0^h \frac{G}{h^2}x^2\dD x
=\frac{G}{h^2}\cdot\frac{1}{3}\left[x^3\right]_0^h=
\frac{G}{3h^2}(h^3-0)=\frac{1}{3}Gh
\]

\NA Bestimme die Energie, die notwendig ist, um einen Körper der Masse $m$ im
Schwerefeld der Masse $M$ vom Abstand $R$ ins Unendliche zu bringen.

Um den Körper das Stückchen $\dD r$ vom Zentralkörper zu entfernen, braucht
man die Arbeit $\dD W=F\dD r$, wobei $F$ die Gravitationskraft ist. Die gesamte
notwendige Arbeit ist dann:
\[
W=\int_R^{\infty}\dD W=\int_R^{\infty}G\frac{Mm}{r^2}\dD r=
GMm\left[-\frac{1}{r}\right]_R^{\infty}=GMm(0-(-\frac{1}{R}))=\frac{GMm}{R}
\]

Dies war ein Beispiel eines sogenannten \emph{uneigentlichen Integrals}, bei
solchen liegen entweder die Grenzen im Unendlichen, oder die Funktion selbst
wird an einer Stelle unendlich.

\NA Bestimme $ \D\int_{-\infty}^{\infty}\E^{-x^2}\dD x$

Diese Aufgabe habe ich eingefügt, weil das ein wichtiges Integral ist. Seine
Berechnung ist aber nicht ganz einfach, man muss einen ">Umweg"< machen.
Zuerst schreiben wir:
\[
\int_{-\infty}^{\infty}\E^{-x^2}\dD x=
\sqrt{\left(\int_{-\infty}^{\infty}\E^{-x^2}\dD x\right)
  \left(\int_{-\infty}^{\infty}\E^{-y^2}\dD y\right)}=
\sqrt{\int_{-\infty}^{\infty}\int_{-\infty}^{\infty}
    \E^{-(x^2+y^2)}\dD x\dD y}
\]
Führt man nun Polarkoordinaten ein, also $x=r\cos\phi; y=r\sin\phi$, dann ist
das Flächenelement $r\dD\phi\dD r$ und $x^2+y^2=r^2$. Das letzte Integral wird
dadurch
\[
\int_0^{2\pi}\dD\phi\int_0^{\infty}r\E^{-r^2}\dD r
=2\pi\left[-\tfrac12\E^{-r^2}\right]_0^{\infty}=\pi \qRpf
\int_{-\infty}^{\infty}\E^{-x^2}\dD x=\sqrt{\pi}
\]

Mit Substitution kann man nun noch beweisen:
\[
\int_{-\infty}^{\infty}\E^{-a x^2}\dD x=\sqrt{\frac{\pi}{a}}
\]
Dazu setzt man $u^2=ax^2\Rpf\dD u=\sqrt{a}\dD x$. Dies kann nur gehen, wenn
$a>0$ ist. Man kann zeigen, dass die Formel sogar für komplexe $a$ stimmt,
sofern $\Re a>0$ ist.

\paragraph{Anmerkung:} $\E^{-x^2}$ hat keine elementare Stammfunktion, es gibt
aber eine Reihe von höheren Funktionen, die mit diesem Integral verwandt sind.
Zunächst ist dabei die aus der Wahrscheinlichkeitsrechnung bekannte
Normalverteilung (\caps{Gau"s}-Verteilung) zu erwähnen, die durch
\[
\Phi(x)=\frac{1}{\sqrt{2\pi}}\int_{-\infty}^x\E^{-\frac{1}{2}t^2}\dD t
\qqbox{mit}\Phi(\infty)=1
\]
definiert ist. Daneben gibt es die \caps{Gau"s}sche Fehlerfunktion
\[
\textrm{erf}\,(x)=\frac{2}{\sqrt{\pi}}\int_0^x\E^{-t^2}\dD t\qqbox{wobei}
\Phi(x)=\frac{1}{2}\biggl(1+\textrm{erf}\,\Bigl(\frac{x}{\sqrt{2}}\Bigr)\biggr)
\]

\NA Bestimme das Integral durch Partialbruchzerlegung.
Es sei gegeben:
\[
f(x)=\frac{x+2}{(x^2-1)(x^2+1)^2}=\frac{A}{x-1}+\frac{B}{x+1}+
\frac{Cx+D}{(x^2+1)^2}+\frac{Ex+F}{x^2+1}
\]
Nun müssen die Koeffizienten $A,\dots,F$ bestimmt werden. Dazu multipliziert
man zuerst mit dem Hauptnenner durch:
\[
x+2=A(x+1)(x^2+1)^2+B(x-1)(x^2+1)^2+(Cx+D)(x^2-1)+(Ex+F)(x^4-1)
\]
Zunächst kann man nun für $x$ spezielle Werte einsetzen, so dass möglichst
viele Ausdrücke Null werden:
\begin{align*}
  x=1: &&& 3=8A \qRpf A=\tfrac{3}{8}\\
  x=-1:&&& 1=-8B \qRpf B=-\tfrac{1}{8}\\
  x=0: &&& 2=\tfrac38+\tfrac18-D-F
\end{align*}
Hätte man nur einfache Vielfachheiten, so könnte man auf diese Weise alle
Koeffizienten bestimmen. Hier ist das leider nicht der Fall, so dass wir zur
">Ochsentour"< greifen müssen. Dazu wird die rechte Seite ausmultipliziert und
dann ein Koeffizientenvergleich gemacht:
\begin{gather*}
  x+2=(A-B-D-F)+(A+B-C-E)x+(2A-2B+D)x^2\\
  \phantom{x}+(2A+2B+C)x^3+(A-B+F)x^4+(A+B+E)x^5
\end{gather*}
Der Koeffizientenvergleich liefert nun ein Gleichungssystem (wobei wir unsere
schon bekannten Werte für $A$ und $B$ gleich einsetzen)
\begin{align*}
  2&=\tfrac{3}{8}+\tfrac{1}{8}-D-F\\
  1&=\tfrac{3}{8}-\tfrac{1}{8}-C-E\\
  0&=\tfrac{3}{4}+\tfrac{1}{4}+D\qRpf D=-1\\
  0&=\tfrac{3}{4}-\tfrac{1}{4}+C\qRpf C=-\tfrac{1}{2}\\
  0&=\tfrac{3}{8}+\tfrac{1}{8}+F\qRpf F=-\tfrac{1}{2}\\
  0&=\tfrac{3}{8}-\tfrac{1}{8}+E\qRpf E=-\tfrac{1}{4}
\end{align*}
Setzt man die sich aus den letzten vier Gleichungen ergebenden Werte nun in die
beiden ersten ein, so sind diese erfüllt. Wir sind fertig und haben das
Ergebnis
\[
f(x)=\frac{\frac38}{(x-1)}-\frac{\frac18}{x+1}-\frac{\frac12x+1}{(x^2+1)^2}
-\frac{\frac14x+\frac12}{x^2+1}
\]
Nach den Gleichungen~\eqref{eq:70} und \eqref{eq:71} ergibt sich so:
\begin{gather*}
\int f(x)\dD x=\frac{3}{8}\ln|x-1|-\frac18\ln|x+1|+\frac14\cdot\frac{1}{x^2+1}\\
\phantom{x}-\frac12\arctan x-\frac{x}{2(x^2+1)}-\frac{1}{8}\ln|x^2+1|
-\frac12\arctan x
\end{gather*}

Der Koeffizientenvergleich ist deshalb so leicht gegangen, weil wir schon
einige Werte mit der Einsetze-Methode ermittelt hatten. Man hätte statt des
Koeffizientenvergleichs auch weitere Gleichungen erzeugen können, indem man
einfach weitere Zahlen einsetzt. Wenn irgend möglich, sollte man versuchen mit
der Einsetze-Methode zum Ziel zu kommen. Am aller einfachsten ist es
allerdings, Mathematica anzuwerfen, dann hat man das Integral in
Sekundenbruchteilen.
