\chapter{Folgen}



\begin{Definition}
Eine Funktion , bei der nur natürlichen Zahlen eine reelle Zahl zugeordnet wird, nennt man Folge.\\
Folgen können auch nur für Teilbereiche von $\N$ definiert sein.\\
$(a_{n})_{n\in\N}$ bezeichnet die Folge, wobei $a:\N\rightarrow\R$
\end{Definition}

\begin{Bemerkung}
In einem Ausdruck muss das $n$ immer dasselbe bleiben!
\end{Bemerkung}
		\section{Verschiedene Darstellungen}


	\subsection{Explizite Darstellung}

\begin{Definition}
Wenn ein beliebiges Glied der Folge direkt berechenbar ist, ist ihre Darstellung explizit.
\end{Definition}

\begin{Beispiel}
\begin{enumerate}
\item  $a_{n}=3^n \Rightarrow a_{4}=3^4=91$
\item Die Folge der $n-ten$ positiven, ungeraden Zahl:\\
$a_{n}=1+2\cdot(n-1) \Rightarrow$ Die 8. positive, ungerade Zahl ist $ a_{8}=1+2\cdot(8-1)=15$
\end{enumerate}
\end{Beispiel}


	\subsection{Rekursive Darstellung}

\begin{Definition}
Wenn für die Berechnung des $n-ten$ Gliedes einer Folge das $(n-1)-te$ Glied benötigt wird, ist ihre Darstellung rekursiv.
In diesen Fällen braucht man immer ein Startglied, oft $a_{0}$ oder $ a_{1}$.
\end{Definition}

\begin{Beispiel}
\begin{enumerate}
\item  $a_{n}=3\cdot a_{n-1}+2;a_{0}=5$\\
\indent$a_{1}=3\cdot a_{1-1}+2=3\cdot a_{0}+2=3\cdot5+2=17$\\
\indent$a_{2}=3\cdot a_{2-1}+2=3\cdot a_{1}+2=3\cdot17+2=53$\\
\indent$a_{3}=3\cdot a_{3-1}+2=3\cdot a_{2}+2=3\cdot53+2=159$\\
\indent und so weiter...
\item Die Folge der $n-ten$ positiven, ungeraden Zahl:\\
$a_{n}=a_{n-1}+2;a_{1}=1$
\end{enumerate}
\end{Beispiel}

\begin{Bemerkung}
Für manche Folgen sind beide Darstellungen möglich, wobei die explizite Darstellung oftmals viel praktischer ist, da die Berechnung der Folgeglieder anhand der rekursiven Darstellung schnell sehr aufwendig wird.
\end{Bemerkung}

\begin{GTR-Tipp}
Wie man im GTR macht
\end{GTR-Tipp}

\subsubsection{Web-Diagramme}
Hier handelt es sich um ein graphisches Verfahren, das dazu dient, das Verhalten einer Folge, deren Darstellung rekursiv ist, zu untersuchen.\\
Dazu muss man der rekursiven Folgenvorschrift eine Funktion $f(a_{n-1})=a_{n}$ zuordnen, sodass - grob gesagt - " die Funktion das Gleiche mit x macht, dass die Folge macht, um von $a_{n}$ auf $a_{n+1}$ zu kommen". Zusätzlich zeichnet man in ein kartesisches Koordinatensystem die Hauptdiagonale ein (entspricht dem Graphen von $f(x)=x$).\\
Dann trägt man das erste Folgeglied auf die Abzissenachse ein und verbindet ihn mit der entsprechenden Funktion anhand eines vertikalen\\

\begin{Beispiel}
\end{Beispiel}

\begin{Bemerkung}
Dieses Verfahren kann ausschließlich bei rekusiven Folgen angewendet werden, bei denen keine zusätzliche Abhängigkeit von $n$ vorliegt (Beispiel: $a_{n}=3\cdot a_{n-1}+3+4\cdot n$) oder die Rekursivitätsebene den 1. Grad überschreitet, was bedeutet, dass $a_{n}$ nicht nur in Abhängigkeit von $a_{n-1}$ beschrieben wird, sondern von anderen Rekursivitätsebenen wie $a_{n-2}$ (Beispiel: die Fibonacci-Folge).\\
\end{Bemerkung}

\begin{GTR-Tipp}
Verwendung mit dem GTR
\end{GTR-Tipp}


		\section{Auffällige Folgen}


	\subsection{Arithmetische Folgen}

\begin{Definition}
Eine Folge wird arithmetisch genannt, wenn die Differenz zweier aufeinander folgender Glieder konstant ist.
\begin{enumerate}
\item Rekursive Darstellung:\\
\indent $a_{n}=a_{n-1}+d$
\item Explizite Darstellung:\\
\indent Mit Startglied $a_{0}$: $a_{n}=a_{0}+n\cdot d$\\
\indent Mit Startglied $a_{1}$: $a_{n}=a_{1}+(n-1)\cdot d$\\
\indent Mit Startglied $a_{x}$: $a_{n}=a_{x}+(n-x)\cdot d$
\end{enumerate}
\end{Definition}

\begin{Bemerkung}
Letzteres gilt auch für beliebige Folgeglieder, also ist $a_{n}=a_{p}+(n-p)\cdot d;n,p\in\N$
\end{Bemerkung}

\begin{Beispiel}
$a_{n}=a_{n-1}+3;a_{0}=0\Leftrightarrow a_{n}=0+n\cdot3$
\end{Beispiel}

\begin{Bemerkung}
Jedes Folgeglied einer solchen Folge ist das arithmetische Mittel seines Vorgängers und Nachgängers: $a_{n}=\dfrac{a_{n-1}+a_{n+1}}{2}$
\end{Bemerkung}

	\subsection{Geometrische Folgen}

\begin{Definition}
Eine Folge wird geometrisch genannt, wenn der Quotient zweier aufeinander folgender Glieder konstant ist.
\begin{enumerate}
\item Rekursive Darstellung:\\
\indent $a_{n}=a_{n-1}\cdot q$
\item Explizite Darstellung:\\
\indent Mit Startglied $a_{0}$: $a_{n}=a_{0}\cdot q^n$\\
\indent Mit Startglied $a_{1}$: $a_{n}=a_{1}\cdot q^{n-1}$\\
\indent Mit Startglied $a_{x}$: $a_{n}=a_{x}\cdot q^{n-x}$\\
\end{enumerate}
\end{Definition}

\begin{Bemerkung}
Letzteres gilt auch für beliebige Folgeglieder, also ist $a_{n}=a_{p}\cdot q^{n-p};n,p\in\N$
\end{Bemerkung}

\begin{Beispiel}
$a_{n}=a_{n-1}\cdot3;a_{0}=2\Leftrightarrow a_{n}=2\cdot3^n$
\end{Beispiel}

\begin{Bemerkung}
Jedes Folgeglied einer solchen Folge ist das geometrische Mittel seines Vorgängers und Nachgängers:\\
 $a_{n}=\sqrt{a_{n-1}\cdot a_{n+1}}$
\end{Bemerkung}

		\section{Klassifizierung von Folgen}


	\subsection{Monotonie}

\begin{Definition}
Eine Folge $(a_n)_n\in\N$ heißt monoton \begin{cases} \text{\textcolor{red}{steigend/wachsend}}\\\text{\textcolor{orange}{fallend/abnehmend}}\end{cases} wenn \begin{cases} \textcolor{red}{a_n+1\geq a_n}\\\textcolor{orange}{a_n+1\leq a_n}\end{cases}.\\\\
Gelten dabei sagar \textbf{strikte} Ordnungsrelationen ($>$ oder $<$), dann ist $(a_n)$ \textbf{streng} monoton wachsend beziehungsweise abnehmend.

\end{Definition}

	\subsection{Beschränktheit}

\begin{Definition}
Man nennt eine  Folge $(a_n)n\in\N$ nach  \begin{cases} \text{\textcolor{red}{oben}}\\\text{\textcolor{orange}{unten}}\end{cases} \textbf{beschränkt}\\ wenn es eine Zahl  \begin{cases} \textcolor{red}{S\in\R}\\\textcolor{orange}{s\in\R}\end{cases} gibt mit  \begin{cases} \textcolor{red}{a_n\leq S}\\\textcolor{orange}{a_n\geq s}\end{cases} $\forall n\in\N$.\\\\
\textcolor{red}{$S$} ist eine \textcolor{red}{obere }Schranke\\
\textcolor{orange}{$s$} ist eine \textcolor{orange}{untere} Schranke
\end{Definition}

\begin{Definition}
Die kleinste obere Schranke ist das \textbf{Supremum} der Menge $\{a_n;n\in\N\}$.\\
Die größte untere Schranke ist das \textbf{Infimum} der Menge $\{a_n;n\in\N\}$.
\end{Definition}


\begin{Definition}
Eine nach \textbf{oben und unten} beschränkte Folge heißt \textbf{beschränkte} Folge (suite bornée).}
\end{Definition}

\begin{Beispiel}
\end{Beispiel}



	\subsection{Konvergenz}

\begin{Definition}
Eine Folge $(a_n)n\in\N$ ist konvergent, wenn sie einen Grenzwert besitzt.\\
Man sagt $a_n$ konvergiert gegen $g=\lim\limits_{n\to\infty}a_n$.
\end{Definition}


 
\begin{Theorem}

$$\lim\limits_{n\to\infty}a_n=g\Leftrightarrow \lim\limits_{n\to\infty}(a_n-g)=0$$
\end{Theorem}
\\
\underline{Wörtlich:} Eine Folge $(a_n)$ konvergiert gegen $g$ genau dann, wenn $(a_n-g)$ gegen den Wert $0$ konvergiert.
\subsubsection{Monotone Konvergenz}


\begin{Theorem}
Eine monotone Folge ist genau dann konvergent, wenn sie beschränkt ist.
\end{Theorem}

\begin{Beweis}
\end{Beweis}

Viel besser zu verstehen, wenn man zwei Fälle unterscheidet:

\begin{Theorem}
$$\exists\, S\in\R \quad \exists \,n_0\in\N \quad \forall n\geq n_0\,:\, a_n\leq a_{n+1}\; \land \; a_n\leq S$$
$$\Rightarrow \exists\, g\in\R \,:\, \lim\limits_{n\to\infty}a_n=g\;\land\; g\leq S $$
 \begin{center}und\end{center}
$$\exists \, s\in\R \quad\exists \, n_0\in\N \quad \forall n\geq n_0\,:\, a_n\geq a_{n+1}\; \land \;a_n\geq s$$
$$\Rightarrow \exists \, g\in\R \,:\, \lim\limits_{n\to\infty}a_n=g\;\land\; g\geq s$$
\end{Theorem}

	\underline{Wörtlicch:}
\begin{itemize}
\item Eine monoton \textbf{wachsende} Folge $(a_n)$ ist genau dann \textbf{konvergent}, wenn sie nach \textbf{oben beschränkt} ist. Ihr Grenzwert $g$ ist kleiner oder gleich der oberen Schranke $S\in\R$.
\item  Eine monoton \textbf{fallende} Folge $(a_n)$ ist genau dann \textbf{konvergent}, wenn sie nach \textbf{unten beschränkt} ist. Ihr Grenzwert $g$ ist größer oder gleich der oberen Schranke $s\in\R$.
\end{itemize}

\subsubsection{Divergenz}
\begin{Definition}
Eine Folge $(a_n)$, die keinen Grenzwert $g\in\R$ besitzt (nicht kovergiert), wird \textbf{divergent} genannt.\\\\
Kann man ihr trotzdem einen Grenzwert wie $\pm\infty$ zuordnen ist sie \textbf{bestimmt divergent}.\\
Besitzt die Folge überhaupt keinen Grenzwert, so heißt sie \textbf{unbestimmt divergent}.
\end{Definition}

\begin{Bemerkung}
Man nennt einen Grenzwert $g=+\infty$ oder $g=-\infty$ einen \textbf{uneigentlichen Grenzwert}.
\end{Bemerkung}


\subsubsection{Epsilon-n0-Definition}
\begin{Definition}
$$\lim\limits_{n\to\infty}a_n=g \ \Leftrightarrow\ \forall  \epsilon > 0\quad \exists\, n_0 \in\N \quad \forall n\geq n_0\,:\, |a_n-g|<\epsilon$$
\end{Definition}
\subsubsection{Grenzwertsätze}


Die Grenzwertsätze führen die Grenzwerte komplizierter Folgen auf einfachere Grenzwertbetrachtungen bekannter Folgen zurück.\\
\begin{Theorem}
\end{Theorem}


