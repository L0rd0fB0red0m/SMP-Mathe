\chapter{Folgen}

Eine Funktion , bei der nur natürlichen Zahlen eine reelle Zahl zugeordnet wird, nennt man Folge.\\
Folgen können auch nur für Teilbereiche von $\N$ definiert sein.\\
$(a_{n})_{n\in\N}$ bezeichnet die Folge, wobei $a:\N\rightarrow\R$\\


		\section{Verschiedene Darstellungen}


	\subsection{Explizite Darstellung}

Wenn ein beliebiges Glied der Folge direkt berechenbar ist, ist ihre Darstellung explizit.
\begin{Beispiel}
\begin{enumerate}
\item  $a_{n}=3^n \Rightarrow a_{4}=3^4=91$
\item Die Folge der $n-ten$ positiven, ungeraden Zahl:\\
$a_{n}=1+2\cdot(n-1) \Rightarrow$ Die 8. positive, ungerade Zahl ist $ a_{8}=1+2\cdot(8-1)=15$
\end{Beispiel}


	\subsection{Rekursive Darstellung}

Wenn für die Berechnung des $n-ten$ Gliedes einer Folge das $(n-1)-te$ Glied benötigt wird, ist ihre Darstellung rekursiv.
In diesen Fällen braucht man immer ein Startglied, oft $a_{0}$ oder $ a_{1}$.\\
\begin{Beispiel}
\begin{enumerate}
\item  $a_{n}=3\cdot a_{n-1}+2;a_{0}=5\\
\indent$a_{1}=3\cdot a_{1-1}+2=3\cdot a_{0}+2=3\cdot5+2=17$\\
\indent$a_{2}=3\cdot a_{2-1}+2=3\cdot a_{1}+2=3\cdot17+2=53$\\
\indent$a_{3}=3\cdot a_{3-1}+2=3\cdot a_{2}+2=3\cdot53+2=159$\\
\indent und so weiter...
\item Die Folge der $n-ten$ positiven, ungeraden Zahl:\\
$a_{n}=a_{n-1}+2;a_{1}=1$

\end{Beispiel}

\begin{Bemerkung}
Bemerkung\\
Für manche Folgen sind beide Darstellungen möglich, wobei die explizite Darstellung oftmals viel praktischer ist, da die Berechnung der Folgeglieder anhand der rekursiven Darstellung schnell sehr aufwendig wird.
\end{Bemerkung}

\subsubsection{Web-Diagramme}
Hier handelt es sich um eine graphisches Verfahren, das dazu dient, das Verhalten einer Folge, deren Darstellung rekursiv ist, zu untersuchen.\\
Dazu muss man der rekursiven Folgenvorschrift eine Funktion $f(a_{n-1})=a_{n}$ zuordnen, sodass - grob gesagt - " die Funktion das Gleiche mit x macht, dass die Folge macht, um von $a_{n}$ auf $a_{n+1}$ zu kommen. Zusätzlich zeichnet man in ein kartesisches Koordinatensystem die Hauptdiagonale ein (entspricht dem Graphen von $f(x)=x$).\\
Dann trägt man das erste Folgeglied auf der Abzissenachse ein und verbindet ihn mit der entsprechenden Funktion anhand eines vertikalen

\begin{Bemerkung}
Bemerkung\\
Dieses Verfahren kann aber ausschließlich bei rekusiven Folgen angewendet werden, bei denen keine zusätzliche Abhängigkeit von $n$ vorliegt (Beispiel: $a_{n}=3\cdot a_{n-1}+3+4\cdot n$) oder die Rekursivitätsebene den 1. Grad überschreitet, was bedeutet, dass $a_{n}$ nicht nur in Abhängigkeit von $a_{n-1}$ beschrieben wird, sondern zusätzlich von mindestens $a_{n-2}$ (Beispiel: die Fibonacci-Folge).\\
\end{Bemerkung}



		\section{Auffällige Folgen}


	\subsection{Arithmetische Folgen}

Eine Folge wird arithmetisch genannt, wenn die Differenz zweier aufeinander folgender Glieder konstant ist.\\

\begin{enumerate}
\item Rekursive Darstellung:\\
\indent $a_{n}=a_{n-1}+d$
\item Explizite Darstellung:\\
\indent Mit Startglied $a_{0}$: $a_{n}=a_{0}+n\cdot d$\\
\indent Mit Startglied $a_{1}$: $a_{n}=a_{1}+(n-1)\cdot d$\\
\indent Mit Startglied $a_{x}$: $a_{n}=a_{x}+(n-x)\cdot d$\\
\end{enumerate}

\begin{Bemerkung}
Bemerkung\\
Letzteres gilt auch für beliebige Folgeglieder, also ist $a_{n}=a_{p}+(n-p)\cdot d;n,p\in\N$
\end{Bemerkung}

\begin{Beispiel}
$a_{n}=a_{n-1}+3;a_{0}=0\Leftrightarrow a_{n}=0+n\cdot3$
\end{Beispiel}

	\subsection{Geometrische Folgen}

Eine Folge wird geometrisch genannt, wenn der Quotient zweier aufeinander folgender Glieder konstant ist.\\

\begin{enumerate}
\item Rekursive Darstellung:\\
\indent $a_{n}=a_{n-1}\cdot q$
\item Explizite Darstellung:\\
\indent Mit Startglied $a_{0}$: $a_{n}=a_{0}\cdot q^n$\\
\indent Mit Startglied $a_{1}$: $a_{n}=a_{1}\cdot q^{n-1}$\\
\indent Mit Startglied $a_{x}$: $a_{n}=a_{x}\cdot q^{n-x}$\\
\end{enumerate}

\begin{Bemerkung}
Bemerkung\\
Letzteres gilt auch für beliebige Folgeglieder, also ist $a_{n}=a_{p}\cdot q^{n-p};n,p\in\N$
\end{Bemerkung}

\begin{Beispiel}
$a_{n}=a_{n-1}\cdot3;a_{0}=2\Leftrightarrow a_{n}=2\cdot3^n$
\end{Beispiel}

		\section{Klassifizierung von Folgen}


	\subsection{Monotonie}

	\subsection{Beschränktheit}

	\subsection{Konvergenz}

\subsubsection{Definition}
\subsubsection{Epsilon-n0-Definition}
\subsubsection{Grenzwertsätze}


		\section{Vollständige Induktion}
