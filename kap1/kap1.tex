\chapter{Folgen}

Eine Funktion , bei der nur natürlichen Zahlen eine reelle Zahl zugeordnet wird, nennt man Folge.\\
Folgen können auch nur für Teilbereiche von $\N$ definiert sein.\\
$(a_{n})_{n\in\N}$ bezeichnet die Folge, wobei $\N\rightarrow\R$\\


		\section{Verschiedene Darstellungen}


	\subsection{Explizite Darstellung}

Wenn ein beliebiges Glied der Folge direkt berechenbar ist, ist ihre Darstellung explizit.
\begin{Bemerkung}
Beispiele\\

\begin{enumerate}
\item  $a_{n}=3^n \Rightarrow a_{4}=3^4=91$
\item Die Folge der $n-ten$ positiven, ungeraden Zahl:\\
$a_{n}=1+2\cdot(n-1) \Rightarrow$ Die 8. positive, ungerade Zahl ist $ a_{8}=1+2\cdot(8-1)=15$

\end{Bemerkung}


	\subsection{Rekursive Darstellung}

Wenn für die Berechnung des $n-ten$ Gliedes einer Folge das $(n-1)-te$ Glied benötigt wird, ist ihre Darstellung rekursiv.\\
In diesen Fällen braucht man immer ein Startglied, oft $a_{0}$ oder $ a_{1}$.
\begin{Bemerkung}
Beispiele\\

\begin{enumerate}
\item  $a_{n}=3\cdot a_{n-1}+2;a_{0}=5\\
\indent$a_{1}=3\cdot a_{1-1}+2=3\cdot a_{0}+2=3\cdot5+2=17$\\
\indent$a_{2}=3\cdot a_{2-1}+2=3\cdot a_{1}+2=3\cdot17+2=53$\\
\indent$a_{3}=3\cdot a_{3-1}+2=3\cdot a_{2}+2=3\cdot53+2=159$\\
\indent und so weiter...
\item Die Folge der $n-ten$ positiven, ungeraden Zahl:\\

\end{Bemerkung}


		\section{Auffällige Folgen}


	\subsection{Arithmetische Folgen}

	\subsection{Geometrische Folgen}


		\section{Klassifizierung von Folgen}


	\subsection{Monotonie}

	\subsection{Beschränktheit}

	\subsection{Konvergenz}

\subsubsection{Definition}
\subsubsection{Epsilon-n0-Definition}
\subsubsection{Grenzwertsätze}


		\section{Vollständige Induktion}
