\documentclass[main.tex]{subfiles}
\begin{document}
\chapter{Folgen}
\chapterauthor{Clara}



\begin{Definition}
Eine Funktion, bei der nur natürlichen Zahlen eine reelle Zahl zugeordnet wird, nennt man Folge.\\
Folgen können auch nur für Teilbereiche von $\N$ definiert sein.\\
$(a_{n})_{n\in\N}$ bezeichnet die Folge, wobei $a:\N\rightarrow\R$
\end{Definition}

\begin{GTR-Tipp}
	Um statt Funktionen Folgen zu behandeln:
	\begin{itemize}
		\item\textcolor{gray}{ON} $\rightarrow \textcolor{red}{\text{MODE}} \rightarrow \text{Zeile 4: statt \textcolor{gray}{FUNC} \textcolor{red}{SEQ} auswählen}$
		\item in \textcolor{gray}{Y=}
		\item Mit $nMin$ den Startindex angeben
		\item Mit $u(n)$ die Folgenvorschrift angeben
		\item Falls die Folge rekursiv anzugeben ist, muss mit $u(nMin)$ das Startglied angegeben werden
	\end{itemize}
\end{GTR-Tipp}

\begin{Bemerkung}
Es ist üblich eine rekursive Folge mit $a_{n+1}$ in Abhängigkeit von $a_n$ zu haben. Diese Darstellung ist mit dem GTR nicht direkt verwendbar. Man muss es \"umschreiben\" um $a_n$ in Abhängigkeit von $a_{n-1}$ zu haben.
\end{Bemerkung}

		\section{Verschiedene Darstellungen}


	\subsection{Explizite Darstellung}

\begin{Definition}
Wenn ein beliebiges Glied der Folge direkt berechenbar ist, ist ihre Darstellung explizit.
\end{Definition}

\begin{Beispiel}
	\begin{enumerate}
		\item $a_{n}=3^n \Rightarrow a_{4}=3^4=81$
		\item Die Folge der $n-ten$ positiven, ungeraden Zahl:\\
		$a_{n}=2\cdot(n-1)+1 \Rightarrow$ Die 8. positive, ungerade Zahl ist $ a_{8}=2\cdot(8-1)+1=15$
	\end{enumerate}
\end{Beispiel}


	\subsection{Rekursive Darstellung}

\begin{Definition}
Wenn für die Berechnung des $n-ten$ Gliedes eines (oder sogar mehrere) der vorherigen Glieder benötigt wird, ist ihre Darstellung rekursiv.
In diesen Fällen braucht man immer ein Startglied, oft $a_{0}$ oder $ a_{1}$.
\end{Definition}

\begin{Beispiel}
\begin{enumerate}
\item  $a_{n}=3\cdot a_{n-1}+2;\;a_{0}=5$\\
\indent$a_{1}=3\cdot a_{1-1}+2=3\cdot a_{0}+2=3\cdot5+2=17$\\
\indent$a_{2}=3\cdot a_{2-1}+2=3\cdot a_{1}+2=3\cdot17+2=53$\\
\indent$a_{3}=3\cdot a_{3-1}+2=3\cdot a_{2}+2=3\cdot53+2=159$\\
\indent und so weiter...
\item Die Folge der $n-ten$ positiven, ungeraden Zahl:\\
$a_{n}=a_{n-1}+2;\;a_{1}=1$
\end{enumerate}
\end{Beispiel}

\begin{Bemerkung}
Für manche Folgen sind beide Darstellungen möglich, wobei die explizite Darstellung oftmals viel praktischer ist, da die Berechnung der Folgeglieder anhand der rekursiven Darstellung schnell sehr aufwendig wird.
\end{Bemerkung}

\subsubsection{Web-Diagramme}


Hier handelt es sich um ein graphisches Verfahren, das dazu dient, das Verhalten einer Folge, deren Darstellung rekursiv ist, zu untersuchen.  Es ermögicht die Beobachtung des Langzeitverhaltens (Konvergenz, Divergenz oder Oszillation) einer rekursiven Folge.\\\\
\begin{minipage}{0.5\textwidth}
Dazu muss man der rekursiven Folgenvorschrift eine \textcolor{blue}{Funktion $f(a_{n-1})=a_{n}$} zuordnen, sodass - grob gesagt - " die Funktion das Gleiche mit x macht, dass die Folge macht, um von $a_{n}$ auf $a_{n+1}$ zu kommen". Man muss verstehen, dass es sich hierbei nicht um den Graphen der Folge handelt, die Werte der Folgenglieder sind nicht wie gewohnt abzulesen. Zusätzlich zeichnet man in ein kartesisches Koordinatensystem die \textcolor{blue}{Hauptdiagonale} ein (entspricht dem Graphen von $f(x)=x$).  Exemplarisch wird hier die Folge $a_{n+1}=2\sqrt{a_n}+2$ behandelt, dementsprechend ist die Hilfsfunktion hier $f$ mit $f(x)=2\sqrt{x} +2$.
\end{minipage}
\begin{minipage}{0.5\textwidth}
	\definecolor{ttqqqq}{rgb}{0.2,0,0}
	\definecolor{sexdts}{rgb}{0.1803921568627451,0.49019607843137253,0.19607843137254902}
	\begin{tikzpicture}[line cap=round,line join=round,>=triangle 45,x=1cm,y=1cm,scale=0.5]
	\begin{axis}[
	x=1cm,y=1cm,
	axis lines=middle,
	ymajorgrids=true,
	xmajorgrids=true,
	xmin=-5.088547334373463,
	xmax=10.65108937607874,
	ymin=-3.575119239735864,
	ymax=9.474639170629912,
	xtick={-4,-2,...,22},
	ytick={-4,-2,...,10},]
	\clip(-5.088547334373463,-4.075119239735864) rectangle (22.65108937607874,10.874639170629912);
	\draw[line width=2pt,color=sexdts,smooth,samples=100,domain=2.505408374793447e-7:22.65108937607874] plot(\x,{2*sqrt((\x))+2});
	\draw[line width=2pt,color=ttqqqq,smooth,samples=100,domain=-5.088547334373463:22.65108937607874] plot(\x,{(\x)});
	\begin{scriptsize}
	\draw[color=sexdts] (0.8405353212957103,2.9444911186724285) node {$f$};
	\draw[color=ttqqqq] (-1.446396560176685,-3.1751691937860826) node {$Hauptdiagonale$};
	\end{scriptsize}
	\end{axis}
	\end{tikzpicture}
\end{minipage}


\begin{minipage}{0.5\textwidth}
\definecolor{rvwvcq}{rgb}{0.08235294117647059,0.396078431372549,0.7529411764705882}
\definecolor{ffqqqq}{rgb}{1,0,0}
\definecolor{ttqqqq}{rgb}{0.2,0,0}
\definecolor{sexdts}{rgb}{0.1803921568627451,0.49019607843137253,0.19607843137254902}
\begin{tikzpicture}[line cap=round,line join=round,>=triangle 45,x=1cm,y=1cm,scale=0.5]
\begin{axis}[
x=1cm,y=1cm,
axis lines=middle,
ymajorgrids=true,
xmajorgrids=true,
xmin=-1.7059310726041948,
xmax=13.731302722239086,
ymin=-1.91447455212563,
ymax=10.670477401431077,
xtick={-1,0,...,21},
ytick={-2,-1,...,10},]
\clip(-1.8059310726041948,-2.01447455212563) rectangle (21.731302722239086,10.670477401431077);
\draw[line width=2pt,color=sexdts,smooth,samples=100,domain=1.4376221380591118e-7:21.731302722239086] plot(\x,{2*sqrt((\x))+2});
\draw[line width=2pt,color=ttqqqq,smooth,samples=100,domain=-1.8059310726041948:21.731302722239086] plot(\x,{(\x)});
\draw [line width=2pt] (0,0)-- (0,2);
\draw [line width=2pt] (0,2)-- (2,2);
\draw [line width=2pt] (2,2)-- (2,0);
\begin{scriptsize}
\draw [fill=ffqqqq] (0,0) circle (2.5pt);
\draw[color=ffqqqq] (0.22437993412656146,0.3841672124633506) node {$a_0$};
\draw [fill=rvwvcq] (0,2) circle (2.5pt);
\draw[color=rvwvcq] (0.15251051795910106,2.378543511110368) node {$A$};
\draw [fill=rvwvcq] (2,2) circle (2.5pt);
\draw[color=rvwvcq] (2.146886816606127,2.378543511110368) node {$B$};
\draw [fill=ffqqqq] (2,0) circle (2.5pt);
\draw[color=ffqqqq] (2.2187562327735875,0.3841672124633506) node {$a_1$};
\end{scriptsize}
\end{axis}
\end{tikzpicture}
\end{minipage}
\begin{minipage}{0.5\textwidth}
Dann trägt man den Wert des ersten Folgegliedes auf die Abzissenachse ein und verbindet ihn mit der entsprechenden Funktion anhand eines vertikalen Striches. Der Ordinatenwert, der so erhalten wird (Punkt \textcolor{blue}{A}), entspricht dem Wert von $a_1$. Um den Vorgang erneuern zu können, muss der gefundene Wert wieder auf die Abzissenachse, dazu benutzt man die Hauptdiagonale als \" Spiegel". Ein horizontaler Strich bis zur Hauptdiagonale (zum Punkt \textcolor{blue}{B})  und ein vertikaler bis zur Abzissenachse lösen das Problem. So ist der Wert von $a_1$ aud der Abzissenachse, dort wird er für den nächsten Schritt benötigt.
\end{minipage}
\\\\\\
Dasselbe muss mehrmals wiederholt werden, so wird jeweils das nächste Folgeglied auf die Abzissenachse abgebildet (\textcolor{red}{rote} Punkte). Daraus kann man dann eine \textcolor{blue}{Tendenz} erkennen\textcolor{blue}{, die die Entwicklung der Folgeglieder beschreibt}. Je nach dem, was für eine Tendenz zu erkennen ist, kann man verschiedene Schlüsse bezüglich der Entwicklung der Folge schließen. In diesem Falle wird deutlich, dass die Folge konvergiert, der Grenzwert ist die Schnittstelle zwischen der Hilfsfunktion und der Hauptdiagonalen, es fehlt nur noch diesen zu berechnen.
\\\\
%web2 weg
\begin{minipage}{0.6\textwidth}
\definecolor{ffqqqq}{rgb}{1,0,0}
\definecolor{ttqqqq}{rgb}{0.2,0,0}
\definecolor{sexdts}{rgb}{0.1803921568627451,0.49019607843137253,0.19607843137254902}
\begin{tikzpicture}[line cap=round,line join=round,>=triangle 45,x=1cm,y=1cm,scale=0.7]
\begin{axis}[
x=1cm,y=1cm,
axis lines=middle,
ymajorgrids=true,
xmajorgrids=true,
xmin=-1.6667190270420253,
xmax=12.536582509046294,
ymin=-1.3943671167960805,
ymax=8.554893100362574,
xtick={-1,0,...,17},
ytick={-1,0,...,8},]
\clip(-1.6667190270420253,-1.3943671167960805) rectangle (17.536582509046294,8.954893100362574);
\draw[line width=2pt,color=sexdts,smooth,samples=100,domain=3.636824072042055e-8:17.536582509046294] plot(\x,{2*sqrt((\x))+2});
\draw[line width=2pt,color=ttqqqq,smooth,samples=100,domain=-1.6667190270420253:17.536582509046294] plot(\x,{(\x)});
\draw [line width=2pt] (0,0)-- (0,2);
\draw [line width=2pt] (2,2)-- (2,0);
\draw [line width=2pt] (0,2)-- (2,2);
\draw [line width=2pt] (2,2)-- (2.0002386442865516,4.828595866705989);
\draw [line width=2pt] (2.0002386442865516,4.828595866705989)-- (4.832521331888258,4.832521331888258);
\draw [line width=2pt] (4.824827755036873,0)-- (4.834981558980159,6.397718298836414);
\draw [line width=2pt] (4.834981558980159,6.397718298836414)-- (6.4,6.4);
\draw [line width=2pt] (6.439718755130408,0)-- (6.400072931869821,7.059673085040108);
\draw [line width=2pt] (7.062828081198973,0)-- (7.050714692879745,7.310636381029959);
\draw [line width=2pt] (6.400072931869821,7.059673085040108)-- (7.048509591757752,7.054150668760428);
\draw [line width=2pt] (7.050714692879745,7.310636381029959)-- (7.312337285505,7.312337285505);
\draw [line width=2pt] (7.312337285505,7.312337285505)-- (7.337053956685568,0);
\begin{scriptsize}
\draw [fill=ffqqqq] (0,0) circle (2.5pt);
\draw[color=ffqqqq] (0.16565707372976055,0.35738443554168037) node {$a_0$};
\draw [fill=ffqqqq] (2,0) circle (2.5pt);
\draw[color=ffqqqq] (2.159282271369464,0.35738443554168037) node {$a_1$};
\draw [fill=ffqqqq] (4.824827755036873,0) circle (2.5pt);
\draw[color=ffqqqq] (4.988470970961102,0.35738443554168037) node {$a_2$};
\draw [fill=ffqqqq] (6.439718755130408,0) circle (2.5pt);
\draw[color=ffqqqq] (6.600961939640273,0.35738443554168037) node {$a_3$};
\draw [fill=ffqqqq] (7.062828081198973,0) circle (2.5pt);
\draw[color=ffqqqq] (7.231299318305767,0.35738443554168037) node {$a_5$};
\draw [fill=ffqqqq] (7.337053956685568,0) circle (2.5pt);
\draw[color=ffqqqq] (7.495161476816905,0.35738443554168037) node {$a_6$};
\end{scriptsize}
\end{axis}
\end{tikzpicture}
\end{minipage}
\begin{minipage}{0.4\textwidth}
$$h(x)=x; \quad f(x)=2\sqrt{x}+2$$
$$h(x)=g(x)\Leftrightarrow x-2\sqrt{x}-2=0$$
$$\Rightarrow x_1=\left(\dfrac{2+\sqrt{12}}{2}\right)^2$$
$$g=x_1$$
\end{minipage}
\\\\\\
Es gibt andere mögliche Tendenzen, hier ein paar Beispiele:\\\\
\begin{minipage}{0.5\textwidth}
Die divergierende Treppe, es gibt also keinen Grenzwert, man kann den uneigendlichen Grenzwert aber ablesen, hier ist der Grenzwert der Folge $a_n$\\
 (mit $a_{n+1}=-\dfrac{3}{2}a_n-4$ und $a_0=5$) $-\infty$
\end{minipage}
\begin{minipage}{0.5\textwidth}
\definecolor{rvwvcq}{rgb}{0.08235294117647059,0.396078431372549,0.7529411764705882}
\definecolor{ffqqqq}{rgb}{1,0,0}
\definecolor{wrwrwr}{rgb}{0.3803921568627451,0.3803921568627451,0.3803921568627451}
\definecolor{ttqqqq}{rgb}{0.2,0,0}
\begin{tikzpicture}[line cap=round,line join=round,>=triangle 45,x=1cm,y=1cm,scale=0.5]
\begin{axis}[
x=1cm,y=1cm,
axis lines=middle,
ymajorgrids=true,
xmajorgrids=true,
xmin=-5.564895824577204,
xmax=10.351157204568363,
ymin=-8.78942481077331,
ymax=9.29739460798684,
xtick={-6,-4,...,28},
ytick={-8,-6,...,10},]
\clip(-7.064895824577204,-8.78942481077331) rectangle (28.351157204568363,10.29739460798684);
\draw[line width=2pt,color=ttqqqq,smooth,samples=100,domain=-7.064895824577204:28.351157204568363] plot(\x,{(\x)});
\draw[line width=2pt,color=wrwrwr,smooth,samples=100,domain=-7.064895824577204:28.351157204568363] plot(\x,{3/2*(\x)-4});
\draw [line width=2pt] (5,0)-- (4.999511757557879,3.4992676363368176);
\draw [line width=2pt] (4.999511757557879,3.4992676363368176)-- (3.490167444195548,3.490167444195548);
\draw [line width=2pt] (3.490167444195548,3.490167444195548)-- (3.493252072731324,0);
\draw [line width=2pt] (3.4921577126709917,1.2382365690064878)-- (1.2326800880447495,1.2326800880447495);
\draw [line width=2pt] (1.2326800880447495,1.2326800880447495)-- (1.2275059703280837,-2.1587410445078743);
\draw [line width=2pt] (1.2275059703280837,-2.1587410445078743)-- (-2.1638122523457697,-2.1638122523457697);
\draw [line width=2pt] (-2.1638122523457697,-2.1638122523457697)-- (-2.157820383689419,-7.2367305755341285);
\draw [line width=2pt] (-2.157820383689419,-7.2367305755341285)-- (-7.246229313030961,-7.246229313030961);
\draw [line width=2pt] (-2.157820383689419,-7.2367305755341285)-- (-2.172139496975313,0);
\begin{scriptsize}
\draw [fill=ffqqqq] (5,0) circle (2.5pt);
\draw[color=ffqqqq] (5.290170003147623,0.6593618561624863) node {$a_0$};
\draw [fill=ffqqqq] (3.493252072731324,0) circle (2.5pt);
\draw[color=ffqqqq] (3.803236479023191,0.6593618561624863) node {$a_1$};
\draw [fill=rvwvcq] (-7.246229313030961,-7.246229313030961) circle (2.5pt);
\draw[color=rvwvcq] (-6.875649739688639,-5.937215959952919) node {$J$};
\draw [fill=ffqqqq] (-2.172139496975313,0) circle (2.5pt);
\draw[color=ffqqqq] (-1.8741460676337316,0.6593618561624863) node {$a_3$};
\draw [fill=ffqqqq] (1.2307994510983928,0) circle (2.5pt);
\draw[color=ffqqqq] (1.5322834603604218,0.6593618561624863) node {$a_2$};
\end{scriptsize}
\end{axis}
\end{tikzpicture}
\end{minipage}



\begin{minipage}{0.5\textwidth}
\definecolor{ffqqqq}{rgb}{1,0,0}
\definecolor{wrwrwr}{rgb}{0.3803921568627451,0.3803921568627451,0.3803921568627451}
\definecolor{wvvxds}{rgb}{0.396078431372549,0.3411764705882353,0.8235294117647058}
\begin{tikzpicture}[line cap=round,line join=round,>=triangle 45,x=1cm,y=1cm,scale=0.8]
\begin{axis}[
x=1cm,y=1cm,
axis lines=middle,
ymajorgrids=true,
xmajorgrids=true,
xmin=-1.4635665987461786,
xmax=8.87634424568403,
ymin=-1.6012600965995993,
ymax=6.32408946253023,
xtick={-1,0,...,15},
ytick={-1,0,...,7},]
\clip(-1.4635665987461786,-1.6012600965995993) rectangle (15.097634424568403,7.32408946253023);
\draw[line width=2pt,color=wvvxds,smooth,samples=100,domain=-1.4635665987461786:15.097634424568403] plot(\x,{4/(\x)+1});
\draw[line width=2pt,color=wrwrwr,smooth,samples=100,domain=-1.4635665987461786:15.097634424568403] plot(\x,{(\x)});
\draw [line width=2pt] (1,0)-- (1,5);
\draw [line width=2pt] (1,5)-- (5,5);
\draw [line width=2pt] (5,5)-- (5,0);
\draw [line width=2pt] (5,1.8)-- (1.8115586238362207,1.8115586238362207);
\draw [line width=2pt] (1.819217423943969,0)-- (1.819217423943969,3.1987476303564684);
\draw [line width=2pt] (1.819217423943969,3.1987476303564684)-- (3.188287472262904,3.188287472262904);
\draw [line width=2pt] (3.188287472262904,3.188287472262904)-- (3.182039920366344,0);
\draw [line width=2pt] (3.1864592764550546,2.255311821982552)-- (2.2426555359698286,2.2426555359698286);
\draw [line width=2pt] (2.2364079840732676,0)-- (2.2364079840732676,2.788582418094674);
\draw [line width=2pt] (2.2364079840732676,2.788582418094674)-- (2.791956440140071,2.791956440140071);
\draw [line width=2pt] (2.791956440140071,2.791956440140071)-- (2.792662064245665,0);
\begin{scriptsize}
\draw[color=wvvxds] (0.8120182746863592,7.267199840694417) node {$f$};
\draw [fill=ffqqqq] (1,0) circle (2.5pt);
\draw[color=ffqqqq] (1.1407138675155035,0.3140238385394505) node {$a_1$};
\draw [fill=ffqqqq] (5,0) circle (2.5pt);
\draw[color=ffqqqq] (5.1356295342081815,0.3140238385394505) node {$a_2$};
\draw [fill=ffqqqq] (1.819217423943969,0) circle (2.5pt);
\draw[color=ffqqqq] (1.9624528495883644,0.3140238385394505) node {$a_3$};
\draw [fill=ffqqqq] (3.182039920366344,0) circle (2.5pt);
\draw[color=ffqqqq] (3.3151616354621507,0.3140238385394505) node {$a_4$};
\draw [fill=ffqqqq] (2.2364079840732676,0) circle (2.5pt);
\draw[color=ffqqqq] (2.379643409717663,0.3140238385394505) node {$a_5$};
\draw [fill=ffqqqq] (2.792662064245665,0) circle (2.5pt);
\draw[color=ffqqqq] (2.935897489890061,0.3140238385394505) node {$a_6$};
\end{scriptsize}
\end{axis}
\end{tikzpicture}
\end{minipage}
\begin{minipage}{0.5\textwidth}
Die konvergierende Spirale, es gibt also einen Grenzwert, den man erneut mit der Schnittstelle zwischen Funktion und Hauptdiagonale ermitteln kann. Hier konvergiert die Folge  $a_n$ mit $a_{n+1}=\dfrac{4}{a_n}+1$ und $a_1=1$
\end{minipage}

\begin{Bemerkung}
Dieses Verfahren kann ausschließlich bei rekusiven Folgen angewendet werden, bei denen keine zusätzliche Abhängigkeit von $n$ vorliegt (Beispiel: $a_{n}=3\cdot a_{n-1}+4n+3$) oder die Rekursivitätsebene den 1. Grad überschreitet, was bedeutet, dass $a_{n}$ nicht nur in Abhängigkeit von $a_{n-1}$ beschrieben wird, sondern von anderen Rekursivitätsebenen wie $a_{n-2}$ (Beispiel: die Fibonacci-Folge).\\
\end{Bemerkung}

\begin{GTR-Tipp}
	Mit den GTR ist dieses Verfahren auch möglich, die Arbeit der Fertigstellung der Striche wird vom Rechner übernommen. Das Bild ist vom Benutzer nur noch zu deuten, gegebenenfalls ist die Schnittstelle auszurechnen.
	\begin{itemize}
		\item Der Rechner muss auf \textcolor{red}{SEQ} stehen
		\item in \textcolor{gray}{Y=} die rekursive Folge angeben
		\item In \textcolor{gray}{2nd }\textcolor{red}{FORMAT} von \textcolor{gray}{Time} auf \textcolor{red}{Web} stellen
		\item \textcolor{red}{TRACE} verwenden
		\item Sooft auf \textcolor{gray}{ENTER} drücken, bis ausreichend Striche zu sehen sind.
	\end{itemize}
\end{GTR-Tipp}


		\section{Auffällige Folgen}


	\subsection{Arithmetische Folgen}

\begin{Definition}
Eine Folge wird arithmetisch genannt, wenn die Differenz zweier aufeinander folgender Glieder konstant ist.
\begin{enumerate}
\item Rekursive Darstellung:\\
\indent $a_{n}=a_{n-1}+d$
\item Explizite Darstellung:\\
\indent Mit Startglied $a_{0}$: $a_{n}=a_{0}+n\cdot d$\\
\indent Mit Startglied $a_{1}$: $a_{n}=a_{1}+(n-1)\cdot d$\\
\indent Mit Startglied $a_{x}$: $a_{n}=a_{x}+(n-x)\cdot d$
\end{enumerate}
\end{Definition}

\begin{Bemerkung}
Letzteres gilt auch für beliebige Folgeglieder, also ist $a_{n}=a_{p}+(n-p)\cdot d;\;n,p\in\N$
\end{Bemerkung}

\begin{Beispiel}
$a_{n}=a_{n-1}+3;\;a_{0}=0\Leftrightarrow a_{n}=0+n\cdot3=3n$
\end{Beispiel}

\begin{Bemerkung}
Jedes Folgeglied einer solchen Folge ist das arithmetische Mittel seines Vorgängers und Nachgängers: $a_{n}=\dfrac{a_{n-1}+a_{n+1}}{2}$
\end{Bemerkung}

	\subsection{Geometrische Folgen}

\begin{Definition}
Eine Folge wird geometrisch genannt, wenn der Quotient zweier aufeinander folgender Glieder konstant ist.
\begin{itemize}
\item Rekursive Darstellung:\\
\indent $a_{n}=a_{n-1}\cdot q$
\item Explizite Darstellung:\\
\indent Mit Startglied $a_{0}$: $a_{n}=a_{0}\cdot q^n$\\
\indent Mit Startglied $a_{1}$: $a_{n}=a_{1}\cdot q^{n-1}$\\
\indent Mit Startglied $a_{x}$: $a_{n}=a_{x}\cdot q^{n-x}$\\
\end{itemize}
\end{Definition}

\begin{Bemerkung}
Letzteres gilt auch für beliebige Folgeglieder, also ist $a_{n}=a_{p}\cdot q^{n-p};n,p\in\N$
\end{Bemerkung}

\begin{Beispiel}
$a_{n}=a_{n-1}\cdot3;a_{0}=2\Leftrightarrow a_{n}=2\cdot3^n$
\end{Beispiel}

\begin{Bemerkung}
Jedes Folgeglied einer solchen Folge ist das geometrische Mittel seines Vorgängers und Nachgängers:\\
 $a_{n}=\sqrt{a_{n-1}\cdot a_{n+1}}$
\end{Bemerkung}

		\section{Klassifizierung von Folgen}


	\subsection{Monotonie}

\begin{Definition}
	Eine Folge $(a_n)_{n\in\N}$ heißt monoton
	$\begin{cases}
		\text{\textcolor{red}{steigend/wachsend}} \\
		\text{\textcolor{orange}{fallend/abnehmend}}
	\end{cases}$
	wenn
	$\begin{cases}
		\textcolor{red}{a_{n+1}\geq a_n} \\
		\textcolor{orange}{a_{n+1}\leq a_n}
	\end{cases}$.
	
	Gelten dabei sagar \textbf{strikte} Ordnungsrelationen ($>$ oder $<$), dann ist $(a_n)$ \textbf{streng} monoton wachsend bzw. abnehmend.
\end{Definition}

\underline{Strategie:}
\begin{itemize}
\item Das \textcolor{blue}{Vorzeichen von $a_{n+1}-a_n$ bestimmen}, ist es $\leq0$ dann ist die Folge fallend, ist es $\geq0$, dann ist die Folge wachsend.
\item \textcolor{blue}{$\dfrac{a_{n+1}}{a_n}$ mit 1 vergleichen}, ist es $\leq1$, dann ist die Folge fallend, ist es $\geq1$, dann ist die Folge wachsend.
\end{itemize}

\begin{Beispiel}
	Untersucht wird die Monotonie der Folge $a_n=\dfrac{8n}{n^2+1}$.
	\begin{align*}
		a_{n+1}-a_{n}&=\dfrac{8(n+1)}{(n+1)^2+1}-\dfrac{8n}{n^2+1}\\
		&=\dfrac{8n+8}{n^2+2n+1+1}-\dfrac{8n}{n^2+1}\\
		&=\dfrac{8n^3+8n^2+8n+8-(8n^3+16n^2+16n)}{(n^2+2n+2)(n^2+1)}\\
		&=\dfrac{-8n^2-8n+8}{(n^2+2n+2)(n^2+1)}\\
		&=-\underbrace{\dfrac{8(n^2+n-1)}{(n^2+2n+2)(n^2+1)}}_{\geq2\forall n\in \N}<0\text{ für } (n^2+n+1)>0\Leftrightarrow \text{ für } (n\geq 1) (\text{weil } n \in \N)
	\end{align*}

	$\Rightarrow \, a_n$ ist für $n\in\N$ weder steigend noch fallend.
	
	\indent $a_n$ ist für $n \in \N^*$ streng monoton fallend.
\end{Beispiel}

	\subsection{Beschränktheit}

\begin{Definition}
Man nennt eine Folge $(a_n)n \in \N$ nach
$\begin{cases}
	\text{\textcolor{red}{oben}} \\
	\text{\textcolor{orange}{unten}}
\end{cases}$
\textbf{beschränkt}

wenn es eine Zahl
$\begin{cases}
	\textcolor{red}{S \in \R} \\
	\textcolor{orange}{s \in \R}
\end{cases}$
gibt mit
$\begin{cases}
	\textcolor{red}{a_n \leq S} \\
	\textcolor{orange}{a_n \geq s}
\end{cases}$
$\forall n \in \N$.

\textcolor{red}{$S$} ist eine \textcolor{red}{obere }Schranke\\
\textcolor{orange}{$s$} ist eine \textcolor{orange}{untere} Schranke
\end{Definition}

\begin{Definition}
Die kleinste obere Schranke ist das \textbf{Supremum} der Menge $\{ a_n ; n \in \N \}$.\\
Die größte untere Schranke ist das \textbf{Infimum} der Menge $\{a_n;n \in \N \}$.
\end{Definition}


\begin{Definition}
Eine nach \textbf{oben und unten} beschränkte Folge heißt \textbf{beschränkte} Folge (suite bornée).}
\end{Definition}

\begin{Beispiel}
\definecolor{qqwuqq}{rgb}{0,0.39215686274509803,0}
\definecolor{rvwvcq}{rgb}{0.08235294117647059,0.396078431372549,0.7529411764705882}
\definecolor{sexdts}{rgb}{0.1803921568627451,0.49019607843137253,0.19607843137254902}
\definecolor{wrwrwr}{rgb}{0.3803921568627451,0.3803921568627451,0.3803921568627451}
\begin{tikzpicture}[line cap=round,line join=round,>=triangle 45,x=1cm,y=1cm;scale=0.5]
\begin{axis}[
x=1cm,y=1cm,
axis lines=middle,
ymajorgrids=true,
xmajorgrids=true,
xmin=-0.857875380692505,
xmax=15.77803725575799,
ymin=-1.7256298925712856,
ymax=2.757177100304307,
xtick={0,1,...,15},
ytick={-1.5,-1,...,2.5},]
\clip(-0.857875380692505,-1.7256298925712856) rectangle (15.77803725575799,2.757177100304307);
\draw [line width=2pt,color=sexdts,domain=-0.857875380692505:15.77803725575799] plot(\x,{(--1.1-0*\x)/1});
\draw [line width=2pt,color=rvwvcq,domain=-0.857875380692505:15.77803725575799] plot(\x,{(-0.05-0*\x)/1});
\begin{scriptsize}
\draw [fill=wrwrwr] (50,0.000014272476927059673) circle (2.5pt);
\draw[color=sexdts] (0.19615572528107583,1.2745490084679743) node {$f$};
\draw[color=rvwvcq] (0.24695240508703153,0.18876997761567066) node {$g$};
\draw [fill=qqwuqq] (0,1) circle (2.5pt);
\draw [fill=qqwuqq] (1,0.8) circle (2.5pt);
\draw [fill=qqwuqq] (2,0.64) circle (2.5pt);
\draw [fill=qqwuqq] (3,0.512) circle (2.5pt);
\draw [fill=qqwuqq] (4,0.4096) circle (2.5pt);
\draw [fill=qqwuqq] (5,0.32768) circle (2.5pt);
\draw [fill=qqwuqq] (6,0.26214400000000015) circle (2.5pt);
\draw [fill=qqwuqq] (7,0.20971520000000016) circle (2.5pt);
\draw [fill=qqwuqq] (8,0.16777216000000014) circle (2.5pt);
\draw [fill=qqwuqq] (9,0.13421772800000012) circle (2.5pt);
\draw [fill=qqwuqq] (10,0.10737418240000011) circle (2.5pt);
\draw [fill=qqwuqq] (11,0.08589934592000009) circle (2.5pt);
\draw [fill=qqwuqq] (12,0.0687194767360001) circle (2.5pt);
\draw [fill=qqwuqq] (13,0.05497558138880008) circle (2.5pt);
\draw [fill=qqwuqq] (14,0.04398046511104006) circle (2.5pt);
\draw [fill=qqwuqq] (15,0.03518437208883206) circle (2.5pt);
\draw [fill=qqwuqq] (16,0.028147497671065648) circle (2.5pt);
\draw [fill=qqwuqq] (17,0.02251799813685252) circle (2.5pt);
\draw [fill=qqwuqq] (18,0.018014398509482017) circle (2.5pt);
\draw [fill=qqwuqq] (19,0.014411518807585615) circle (2.5pt);
\draw [fill=qqwuqq] (20,0.011529215046068495) circle (2.5pt);
\end{scriptsize}
\end{axis}
\end{tikzpicture}
Die abgebildete Folge $a_n=0,8^n$ besitzt als mögliche obere Schranke die Gerade $f:y=1,1$, unten die Gerade $g:y=-0,05$. $a_n$ ist also \textbf{beschränkt}.
\end{Beispiel}

		\subsubsection{Unbeschränktheit}

\begin{Beispiel}
Unbeschränktheit mit Abschätzungen zeigen:\\
$a_n=\dfrac{n^2}{n+2}\geq\dfrac{n^2}{n+2n}=\dfrac{n^2}{3n}=\dfrac{1}{3}n\qquad\forall n\geq1$\\\\
\\$u_n=\dfrac{1}{3}n$ ist eine unbeschränkte Folge, $a_n$ ist ab einem bestimmten Glied ($a_1$) immer darüber:
\\$a_n$ ist ebenfalls unbeschränkt, sie divergiert nach $+\infty$
\end{Beispiel}


	\subsection{Konvergenz}

\begin{Definition}
Eine Folge $(a_n)n\in\N$ ist konvergent, wenn sie einen Grenzwert besitzt.\\
Man sagt $a_n$ konvergiert gegen $g=\lim\limits_{n\to\infty}a_n$.
\end{Definition}



\begin{Theorem}

$$\lim\limits_{n\to\infty}a_n=g\Leftrightarrow \lim\limits_{n\to\infty}(a_n-g)=0$$
\end{Theorem}
\\
\underline{Wörtlich:} Eine Folge $(a_n)$ konvergiert gegen $g$ genau dann, wenn $(a_n-g)$ gegen den Wert $0$ konvergiert.

\begin{Beispiel}
$a_n=\dfrac{3n^4-1}{n^4}$\\\\
$a_n$ hat vermutlich den Grenzwert $g=3$.\\\\
$a_n-3=\dfrac{3n^4-1}{n^4}-3=\dfrac{3n^4-1-3n^4}{n^4}=-\dfrac{1}{n^4}\underrightarrow{n\rightarrow \infty}0$
\end{Beispiel}
\subsubsection{Monotone Konvergenz}


\begin{Theorem}
Eine monotone Folge ist genau dann konvergent, wenn sie beschränkt ist.
\end{Theorem}

\begin{Beweis}
Ohne Beschränkung der Allgemeinheit nehmen wir an, dass die Folge monoton wachsend sei und sei $a$ die kleinste obere Schranke (Supremum).
\begin{itemize}
\item Sei $\varepsilon\geq0$ dann ist $a-\varepsilon$ keine obere Schranke der Folgeglieder $\{a_n;n\in\N\}$.
\item $(a_n)$ ist wachsend
\item $\exists n_0\in\N$ mit $a_{n_{0}} \geq a-\varepsilon$
\item $\forall n>n_0$ gilt $a-\varepsilon<a_{n_{0}}\leq a_n<a+\varepsilon$\qquad|$-a$\\
$\qquad\qquad\Leftrightarrow -\varepsilon < a_{n_{0}}-a\leq a_n-a<\varepsilon$
\item $\Rightarrow \lim\limits_{n\to\infty}a_n=a$
\end{itemize}
Analog kann man mit einer nach unten beschränkten monoton fallenden Folge argumentieren.
\end{Beweis}

Besser zu verstehen, wenn man es so sagt:

\begin{Theorem}
$$\exists\, S\in\R \quad \exists \,n_0\in\N \quad \forall n\geq n_0\,:\, a_n\leq a_{n+1}\; \land \; a_n\leq S$$
$$\Rightarrow \exists\, g\in\R \,:\, \lim\limits_{n\to\infty}a_n=g\;\land\; g\leq S $$
 \begin{center}und\end{center}
$$\exists \, s\in\R \quad\exists \, n_0\in\N \quad \forall n\geq n_0\,:\, a_n\geq a_{n+1}\; \land \;a_n\geq s$$
$$\Rightarrow \exists \, g\in\R \,:\, \lim\limits_{n\to\infty}a_n=g\;\land\; g\geq s$$
\end{Theorem}

	\underline{Wörtlicch:}
\begin{itemize}
\item Eine monoton \textbf{wachsende} Folge $(a_n)$ ist genau dann \textbf{konvergent}, wenn sie nach \textbf{oben beschränkt} ist. Ihr Grenzwert $g$ ist kleiner oder gleich der oberen Schranke $S\in\R$.
\item  Eine monoton \textbf{fallende} Folge $(a_n)$ ist genau dann \textbf{konvergent}, wenn sie nach \textbf{unten beschränkt} ist. Ihr Grenzwert $g$ ist größer oder gleich der oberen Schranke $s\in\R$.
\end{itemize}

\begin{Beispiel}
Untersucht wird die Folge $u_n=\dfrac{n^2}{2^n}$\\
\definecolor{qqwuqq}{rgb}{0,0.39215686274509803,0}
\begin{tikzpicture}[line cap=round,line join=round,>=triangle 45,x=1cm,y=1cm]
\begin{axis}[
x=1cm,y=1cm,
axis lines=middle,
ymajorgrids=true,
xmajorgrids=true,
xmin=-1.378638916426288,
xmax=14.41366474370366,
ymin=-3.4891981146801427,
ymax=3.730407375415932,
xtick={-1,0,...,25},
ytick={-3,-2,...,3},]
\clip(-1.378638916426288,-3.4891981146801427) rectangle (25.41366474370366,3.730407375415932);
\begin{scriptsize}
\draw [fill=qqwuqq] (0,0) circle (2.5pt);
\draw [fill=qqwuqq] (1,0.5) circle (2.5pt);
\draw [fill=qqwuqq] (2,1) circle (2.5pt);
\draw [fill=qqwuqq] (3,1.125) circle (2.5pt);
\draw [fill=qqwuqq] (4,1) circle (2.5pt);
\draw [fill=qqwuqq] (5,0.78125) circle (2.5pt);
\draw [fill=qqwuqq] (6,0.5625) circle (2.5pt);
\draw [fill=qqwuqq] (7,0.3828125) circle (2.5pt);
\draw [fill=qqwuqq] (8,0.25) circle (2.5pt);
\draw [fill=qqwuqq] (9,0.158203125) circle (2.5pt);
\draw [fill=qqwuqq] (10,0.09765625) circle (2.5pt);
\draw [fill=qqwuqq] (11,0.05908203125) circle (2.5pt);
\draw [fill=qqwuqq] (12,0.03515625) circle (2.5pt);
\draw [fill=qqwuqq] (13,0.0206298828125) circle (2.5pt);
\draw [fill=qqwuqq] (14,0.011962890625) circle (2.5pt);
\draw [fill=qqwuqq] (15,0.006866455078125) circle (2.5pt);
\draw [fill=qqwuqq] (16,0.00390625) circle (2.5pt);
\draw [fill=qqwuqq] (17,0.00220489501953125) circle (2.5pt);
\draw [fill=qqwuqq] (18,0.0012359619140625) circle (2.5pt);
\draw [fill=qqwuqq] (19,0.0006885528564453125) circle (2.5pt);
\draw [fill=qqwuqq] (20,0.0003814697265625) circle (2.5pt);
\draw [fill=qqwuqq] (21,0.00021028518676757812) circle (2.5pt);
\draw [fill=qqwuqq] (22,0.00011539459228515625) circle (2.5pt);
\draw [fill=qqwuqq] (23,0.00006306171417236328) circle (2.5pt);
\draw [fill=qqwuqq] (24,0.000034332275390625) circle (2.5pt);
\draw [fill=qqwuqq] (25,0.00001862645149230957) circle (2.5pt);
\draw [fill=qqwuqq] (26,0.000010073184967041016) circle (2.5pt);
\draw [fill=qqwuqq] (27,0.000005431473255157471) circle (2.5pt);
\draw [fill=qqwuqq] (28,0.0000029206275939941406) circle (2.5pt);
\draw [fill=qqwuqq] (29,0.0000015664845705032349) circle (2.5pt);
\draw [fill=qqwuqq] (30,8.381903171539307E-7) circle (2.5pt);
\draw [fill=qqwuqq] (31,4.4750049710273743E-7) circle (2.5pt);
\draw [fill=qqwuqq] (32,2.384185791015625E-7) circle (2.5pt);
\draw [fill=qqwuqq] (33,1.26776285469532E-7) circle (2.5pt);
\draw [fill=qqwuqq] (34,6.728805601596832E-8) circle (2.5pt);
\draw [fill=qqwuqq] (35,3.565219230949879E-8) circle (2.5pt);
\draw [fill=qqwuqq] (36,1.885928213596344E-8) circle (2.5pt);
\draw [fill=qqwuqq] (37,9.96078597381711E-9) circle (2.5pt);
\draw [fill=qqwuqq] (38,5.2532413974404335E-9) circle (2.5pt);
\draw [fill=qqwuqq] (39,2.7666828827932477E-9) circle (2.5pt);
\draw [fill=qqwuqq] (40,1.4551915228366852E-9) circle (2.5pt);
\draw [fill=qqwuqq] (41,7.644302968401462E-10) circle (2.5pt);
\draw [fill=qqwuqq] (42,4.0108716348186135E-10) circle (2.5pt);
\draw [fill=qqwuqq] (43,2.1020696294726804E-10) circle (2.5pt);
\draw [fill=qqwuqq] (44,1.1004885891452432E-10) circle (2.5pt);
\draw [fill=qqwuqq] (45,5.7553961596568115E-11) circle (2.5pt);
\draw [fill=qqwuqq] (46,3.007016857736744E-11) circle (2.5pt);
\draw [fill=qqwuqq] (47,1.5695889032940613E-11) circle (2.5pt);
\draw [fill=qqwuqq] (48,8.185452315956354E-12) circle (2.5pt);
\draw [fill=qqwuqq] (49,4.265032771400001E-12) circle (2.5pt);
\draw [fill=qqwuqq] (50,2.220446049250313E-12) circle (2.5pt);
\end{scriptsize}
\end{axis}
\end{tikzpicture}
\begin{itemize}
\item \underline{Beschränktheit:}\\\\$\forall n\in\N\quad a_n\geq0, \text{ für } n=0,\quad a_n=0\Rightarrow$ die untere Schranke \textcolor{red}{$s=0$ ist das Infimum}.
\item\underline{Monotonie:}
\begin{align*}
\textcolor{blue}{\dfrac{u_{n+1}}{u_n}}&=\dfrac {(n+1)^2}{2^{n+1}}\cdot\dfrac{2^n}{n^2}\\
&=\dfrac{(n+1)^2}{2n^2}\\
&=\dfrac{1}{2}\cdot\left(\dfrac{n+1}{n}\right)^2\\
&=\dfrac{1}{2}\cdot\left(1+\dfrac{1}{n}\right)^2\textcolor{blue}{\leq \dfrac{8}{9}<1}\qquad \forall n\geq3
\end{align*}
\begin{align*}\text{N.R.:}\qquad1+\dfrac{1}{n}&\leq\dfrac{4}{3}\qquad \forall n\geq3\\
\Rightarrow \quad \left(1+\dfrac{1}{n}\right)^2&\leq \dfrac{16}{9}
\end{align*}
Somit ist $u_n$ ab dem 3. Folgeglied \textcolor{blue}{streng monoton fallend} $\Rightarrow S=u_3=\dfrac{9}{8}$
\end{itemize}
$\Rightarrow a_n$ konvergiert gegen den Grenzwert $g=0$
\end{Beispiel}

\subsubsection{Divergenz}
\begin{Definition}
Eine Folge $(a_n)$, die keinen Grenzwert $g\in\R$ besitzt (nicht kovergiert), wird \textbf{divergent} genannt.\\\\
Kann man ihr trotzdem einen Grenzwert wie $\pm\infty$ zuordnen ist sie \textbf{bestimmt divergent}.\\
Besitzt die Folge überhaupt keinen Grenzwert, so heißt sie \textbf{unbestimmt divergent}.
\end{Definition}

\begin{Bemerkung}
Man nennt einen Grenzwert $g=+\infty$ oder $g=-\infty$ einen \textbf{uneigentlichen Grenzwert}.
\end{Bemerkung}

\begin{Beispiel}
\begin{itemize}
\item Die Folge $u_n=-4n^5$ ist bestimmt divergent ($g=-\infty$).
\item Die Folge $a_n=-n\sin n$ ist unbestimmt divergent, sie besitzt überhaupt keinen Grenzwert.\\
\definecolor{qqwuqq}{rgb}{0,0.39215686274509803,0}
\begin{tikzpicture}[line cap=round,line join=round,>=triangle 45,x=1cm,y=1cm,scale=0.3]
\begin{axis}[
x=1cm,y=1cm,
axis lines=middle,
ymajorgrids=true,
xmajorgrids=true,
xmin=-2.896396943553246,
xmax=46.558472172527566,
ymin=-17.739859682103166,
ymax=18.07474916976965,
xtick={-5,0,...,60},
ytick={-15,-10,...,15},]
\clip(-5.896396943553246,-17.739859682103166) rectangle (60.558472172527566,18.07474916976965);
\begin{scriptsize}
\draw [fill=qqwuqq] (0,0) circle (7.5pt);
\draw [fill=qqwuqq] (1,-0.8414709848078965) circle (7.5pt);
\draw [fill=qqwuqq] (2,-1.8185948536513634) circle (7.5pt);
\draw [fill=qqwuqq] (3,-0.4233600241796016) circle (7.5pt);
\draw [fill=qqwuqq] (4,3.027209981231713) circle (7.5pt);
\draw [fill=qqwuqq] (5,4.794621373315692) circle (7.5pt);
\draw [fill=qqwuqq] (6,1.6764929891935552) circle (7.5pt);
\draw [fill=qqwuqq] (7,-4.598906191031523) circle (7.5pt);
\draw [fill=qqwuqq] (8,-7.914865972987054) circle (7.5pt);
\draw [fill=qqwuqq] (9,-3.7090663671758093) circle (7.5pt);
\draw [fill=qqwuqq] (10,5.440211108893697) circle (7.5pt);
\draw [fill=qqwuqq] (11,10.999892272057739) circle (7.5pt);
\draw [fill=qqwuqq] (12,6.43887501600522) circle (7.5pt);
\draw [fill=qqwuqq] (13,-5.462171478746332) circle (7.5pt);
\draw [fill=qqwuqq] (14,-13.868502979728184) circle (7.5pt);
\draw [fill=qqwuqq] (15,-9.754317602356753) circle (7.5pt);
\draw [fill=qqwuqq] (16,4.606453066641045) circle (7.5pt);
\draw [fill=qqwuqq] (17,16.343757361952466) circle (7.5pt);
\draw [fill=qqwuqq] (18,13.517770441890171) circle (7.5pt);
\draw [fill=qqwuqq] (19,-2.8476669835960946) circle (7.5pt);
\draw [fill=qqwuqq] (20,-18.258905014552553) circle (7.5pt);
\draw [fill=qqwuqq] (21,-17.56976840925718) circle (7.5pt);
\draw [fill=qqwuqq] (22,0.1947288043888853) circle (7.5pt);
\draw [fill=qqwuqq] (23,19.463069296028923) circle (7.5pt);
\draw [fill=qqwuqq] (24,21.73388068815897) circle (7.5pt);
\draw [fill=qqwuqq] (25,3.3087937524443256) circle (7.5pt);
\draw [fill=qqwuqq] (26,-19.826519712469672) circle (7.5pt);
\draw [fill=qqwuqq] (27,-25.82215006692158) circle (7.5pt);
\draw [fill=qqwuqq] (28,-7.585362072620333) circle (7.5pt);
\draw [fill=qqwuqq] (29,19.245382642176057) circle (7.5pt);
\draw [fill=qqwuqq] (30,29.640948722785854) circle (7.5pt);
\draw [fill=qqwuqq] (31,12.525167005015016) circle (7.5pt);
\draw [fill=qqwuqq] (32,-17.6456537997341) circle (7.5pt);
\draw [fill=qqwuqq] (33,-32.99709138353982) circle (7.5pt);
\draw [fill=qqwuqq] (34,-17.98881132808081) circle (7.5pt);
\draw [fill=qqwuqq] (35,14.986393432365286) circle (7.5pt);
\draw [fill=qqwuqq] (36,35.704038723952166) circle (7.5pt);
\draw [fill=qqwuqq] (37,23.81091093420898) circle (7.5pt);
\draw [fill=qqwuqq] (38,-11.262005990956641) circle (7.5pt);
\draw [fill=qqwuqq] (39,-37.58802006507943) circle (7.5pt);
\draw [fill=qqwuqq] (40,-29.804526419173953) circle (7.5pt);
\draw [fill=qqwuqq] (41,6.503529420993068) circle (7.5pt);
\draw [fill=qqwuqq] (42,38.49390501245662) circle (7.5pt);
\draw [fill=qqwuqq] (43,35.76631393302973) circle (7.5pt);
\draw [fill=qqwuqq] (44,-0.7788847046381974) circle (7.5pt);
\draw [fill=qqwuqq] (45,-38.29065860403533) circle (7.5pt);
\draw [fill=qqwuqq] (46,-41.48226399184522) circle (7.5pt);
\draw [fill=qqwuqq] (47,-5.807936769025528) circle (7.5pt);
\draw [fill=qqwuqq] (48,36.87622374353601) circle (7.5pt);
\draw [fill=qqwuqq] (49,46.733879985214124) circle (7.5pt);
\draw [fill=qqwuqq] (50,13.118742685196437) circle (7.5pt);
\end{scriptsize}
\end{axis}
\end{tikzpicture}
\end{itemize}
\end{Beispiel}


\subsubsection{Epsilon-n0-Definition}
\begin{Definition}
$$\lim\limits_{n\to\infty}a_n=g \ \Leftrightarrow\ \forall  \varepsilon > 0\quad \exists\, n_0 \in\N \quad \forall n\geq n_0\,:\, |a_n-g|<\varepsilon$$
\end{Definition}

\\
\underline{Strategie:}
\begin{itemize}
\item den Ausdruck $|a_n-g|$ vereinfachen
\item die Ungleichung $|a_n-g|<\varepsilon$ zu einer Ungleichung der Form $n>...$ umformen.
\item man hat jetzt die Bedingung für $n$, $n_0$ ist das erste Glied, dass sie erfüllt.
\item Beweis führen: sei $n\geq n_0$ beliebig, dann ist $|a_n-g|=...<\varepsilon$.
\end{itemize}

\begin{Bemerkung}
Wenn ein bestimmtes $\varepsilon$ angegeben ist, dann verwendet man, um das \textcolor{red}{gesuchte $n_0$} zu finden die Gaußklammern. Angewendet werden diese $\lfloor x \rfloor$ um eine Zahl abzurunden, diese $\lceil x \rceil$ um aufzurunden. In unserem Fall wollen wir eine ganze Zahl, für die die Ungleichung auf jeden Fall erfüllt wird, deshalb rundet man auf also \textcolor{red}{Gaußklammer $\lceil x \rceil$}.
\end{Bemerkung}

\begin{Bemerkung}
\begin{center}Auch Divergenz kann so gezeigt werden:\end{center}
$$\forall a \in \R \;\exists \varepsilon >0\; \forall n_0 \in\N \;\exists n\geq n_0\,:\, |a_n-a|\geq \varepsilon$$
\end{Bemerkung}

\newpage
\begin{Beispiel}
Die Folge $a_n= \dfrac{1-2n}{5+3n}$ wird untersucht, es wird geschätzt, dass $(a_n)$ gegen $-\dfrac{2}{3}$ konvergiert.

\begin{center}
\definecolor{qqzzqq}{rgb}{0,0.6,0}
\definecolor{qqttcc}{rgb}{0,0.2,0.8}
\definecolor{ttzzqq}{rgb}{0.2,0.6,0}
\definecolor{rvwvcq}{rgb}{0.08235294117647059,0.396078431372549,0.7529411764705882}
\definecolor{ffqqqq}{rgb}{1,0,0}
\definecolor{qqwuqq}{rgb}{0,0.39215686274509803,0}
\begin{tikzpicture}[line cap=round,line join=round,>=triangle 45,x=1cm,y=1cm,scale=0.9]
\begin{axis}[
x=1cm,y=5cm,
axis lines=middle,
ymajorgrids=true,
xmajorgrids=true,
xmin=-1.3380762576126726,
xmax=14.36327999137987,
ymin=-1.3,
ymax=1.1,
xtick={-1,0,...,14},
ytick={-4,-3,...,4},]
\clip(-1.3380762576126726,-4.1973484399012415) rectangle (14.36327999137987,4.2646038591741595);
\draw [line width=0.8pt,color=ffqqqq,domain=-1.3380762576126726:14.36327999137987] plot(\x,{(-0.6666666666666666-0*\x)/1});
\draw [line width=2pt,color=rvwvcq,domain=-1.3380762576126726:14.36327999137987] plot(\x,{(-0.5666666666666667-0*\x)/1});
\draw [line width=2pt,color=ttzzqq,domain=-1.3380762576126726:14.36327999137987] plot(\x,{(-0.7666666666666666-0*\x)/1});
\draw [color=ffqqqq](1.0710631363167173,-0.35) node[anchor=north west] {$g=-\dfrac{2}{3}$};
\draw [color=qqttcc](6.764302997592639,-0.3) node[anchor=north west] {$a:y=g+\varepsilon$};
\draw [color=qqzzqq](6.8961464470116605,-0.85) node[anchor=north west] {$b:y=g-\varepsilon$};
\begin{scriptsize}
\draw [fill=qqwuqq] (0,0.2) circle (2.5pt);
\draw [fill=qqwuqq] (1,-0.125) circle (2.5pt);
\draw [fill=qqwuqq] (2,-0.2727272727272727) circle (2.5pt);
\draw [fill=qqwuqq] (3,-0.35714285714285715) circle (2.5pt);
\draw [fill=qqwuqq] (4,-0.4117647058823529) circle (2.5pt);
\draw [fill=qqwuqq] (5,-0.45) circle (2.5pt);
\draw [fill=qqwuqq] (6,-0.4782608695652174) circle (2.5pt);
\draw [fill=qqwuqq] (7,-0.5) circle (2.5pt);
\draw [fill=qqwuqq] (8,-0.5172413793103449) circle (2.5pt);
\draw [fill=qqwuqq] (9,-0.53125) circle (2.5pt);
\draw [fill=qqwuqq] (10,-0.5428571428571428) circle (2.5pt);
\draw [fill=qqwuqq] (11,-0.5526315789473685) circle (2.5pt);
\draw [fill=qqwuqq] (12,-0.5609756097560976) circle (2.5pt);
\draw [fill=qqwuqq] (13,-0.5681818181818182) circle (2.5pt);
\draw [fill=qqwuqq] (14,-0.574468085106383) circle (2.5pt);
\draw [fill=qqwuqq] (15,-0.58) circle (2.5pt);
\draw [fill=qqwuqq] (16,-0.5849056603773585) circle (2.5pt);
\draw [fill=qqwuqq] (17,-0.5892857142857143) circle (2.5pt);
\draw [fill=qqwuqq] (18,-0.5932203389830508) circle (2.5pt);
\draw [fill=qqwuqq] (19,-0.5967741935483871) circle (2.5pt);
\draw [fill=qqwuqq] (20,-0.6) circle (2.5pt);
\draw [fill=qqwuqq] (21,-0.6029411764705882) circle (2.5pt);
\draw [fill=qqwuqq] (22,-0.6056338028169014) circle (2.5pt);
\draw [fill=qqwuqq] (23,-0.6081081081081081) circle (2.5pt);
\draw [fill=qqwuqq] (24,-0.6103896103896104) circle (2.5pt);
\draw [fill=qqwuqq] (25,-0.6125) circle (2.5pt);
\draw [fill=qqwuqq] (26,-0.6144578313253012) circle (2.5pt);
\draw [fill=qqwuqq] (27,-0.6162790697674418) circle (2.5pt);
\draw [fill=qqwuqq] (28,-0.6179775280898876) circle (2.5pt);
\draw [fill=qqwuqq] (29,-0.6195652173913043) circle (2.5pt);
\draw [fill=qqwuqq] (30,-0.6210526315789474) circle (2.5pt);
\draw [fill=qqwuqq] (31,-0.6224489795918368) circle (2.5pt);
\draw [fill=qqwuqq] (32,-0.6237623762376238) circle (2.5pt);
\draw [fill=qqwuqq] (33,-0.625) circle (2.5pt);
\draw [fill=qqwuqq] (34,-0.6261682242990654) circle (2.5pt);
\draw [fill=qqwuqq] (35,-0.6272727272727273) circle (2.5pt);
\draw [fill=qqwuqq] (36,-0.6283185840707964) circle (2.5pt);
\draw [fill=qqwuqq] (37,-0.6293103448275862) circle (2.5pt);
\draw [fill=qqwuqq] (38,-0.6302521008403361) circle (2.5pt);
\draw [fill=qqwuqq] (39,-0.6311475409836066) circle (2.5pt);
\draw [fill=qqwuqq] (40,-0.632) circle (2.5pt);
\draw [fill=qqwuqq] (41,-0.6328125) circle (2.5pt);
\draw [fill=qqwuqq] (42,-0.6335877862595419) circle (2.5pt);
\draw [fill=qqwuqq] (43,-0.6343283582089553) circle (2.5pt);
\draw [fill=qqwuqq] (44,-0.635036496350365) circle (2.5pt);
\draw [fill=qqwuqq] (45,-0.6357142857142857) circle (2.5pt);
\draw [fill=qqwuqq] (46,-0.6363636363636364) circle (2.5pt);
\draw [fill=qqwuqq] (47,-0.636986301369863) circle (2.5pt);
\draw [fill=qqwuqq] (48,-0.6375838926174496) circle (2.5pt);
\draw [fill=qqwuqq] (49,-0.6381578947368421) circle (2.5pt);
\draw [fill=qqwuqq] (50,-0.6387096774193548) circle (2.5pt);
\end{scriptsize}
\end{axis}
\end{tikzpicture}
\end{center}
\begin{itemize}
\item \begin{align*}|a_n-g|&=\left|\dfrac{1-2n}{5+3n}+\dfrac{2}{3}\right|\\
&=\left|\dfrac{3(1-2n)+2(5+3n)}{3(5+3n)}\right|\\
&=\left|\dfrac{3-6n+10+6n}{15+9n\right}|\\
&=\left|\dfrac{13}{15+9n}\right|\\
&=\dfrac{13}{15+9n}
\end{align*}
\item
\begin{array}{cccc}
&\dfrac{13}{15+9n}&<&\varepsilon\\
\Leftrightarrow&13&<&\varepsilon(15+9n)\\
\Leftrightarrow&13-15\epsilon&<&9n\varepsilon\\
\Leftrightarrow&n&>&\dfrac{13-15\varepsilon}{9\varepsilon}\\
\end{array}
\item $n_0=\left\lceil \dfrac{13-15\varepsilon}{9\varepsilon}\right\rceil$
\item Sei $\varepsilon$ beliebig und $n\geq n_0$, dann gilt $|a_n-g|=\left|\dfrac{1-2n}{5+3n}+\dfrac{2}{3}\right|=\dfrac{13}{15+9n}<\epsilon$
\end{itemize}

\end{Beispiel}
\newpage
\subsubsection{Grenzwertsätze}


Die Grenzwertsätze führen die Grenzwerte komplizierter Folgen auf einfachere Grenzwertbetrachtungen bekannter Folgen zurück.\\
\begin{Theorem}
Seien $u_n$ und $v_n$ sind konvergente Folgen mit Grenzwerten U und V.
\begin{itemize}
\item Die Folge $u_n\pm v_n$ ist konvergent und besitzt den Grenzwert $U\pm V$.
\item Die Folge $u_n\cdot v_n$ ist konvergent und besitzt den Grenzwert $U\cdot V$.
\item Die Folge $\dfrac{u_n}{v_n}$ ist konvergent und besitzt für $V\not=0$ den Grenzwert $\dfrac{U}{V}$.
\end{itemize}
\end{Theorem}

\begin{Beispiel}
$a_n=\dfrac{4\sqrt{n}-n}{\sqrt{n}-2n}=\dfrac{\not n\left(\frac{4}{\sqrt{n}}-1\right)}{\not n\left(\frac{1}{\sqrt{n}}-2\right)}=\dfrac{\frac{4}{\sqrt{n}}-1}{\frac{1}{\sqrt{n}}-2}$\\\\
$g=\lim\limits_{n\to\infty}a_n=\lim\limits_{n\to\infty}\dfrac{\frac{4}{\sqrt{n}}-1}{\frac{1}{\sqrt{n}}-2}=\dfrac{\lim\limits_{n\to\infty}\frac{4}{\sqrt{n}}-\lim\limits_{n\to\infty}1}{\lim\limits_{n\to\infty}\frac{1}{\sqrt{n}}-\lim\limits_{n\to\infty}2}=\dfrac{0-1}{0-2}=\dfrac{1}{2}$
\end{Beispiel}
\end{document}
