\documentclass[11p,a4paper,fleqn]{report}
\nonstopmode

\usepackage[scale=.9]{helvet} % or tgheros        %Schriftart
\renewcommand{\familydefault}{\sfdefault}

\usepackage{subfiles}                              %Kapitel einbinden
\usepackage{amsmath}		                          %Matheschreibweise...
\usepackage{amssymb}                              %...Und Symbole
\usepackage{mathrsfs}                             %für Geogebra-Zeichnungen
\usepackage[T1]{fontenc}
\usepackage[most]{tcolorbox}
\usepackage{tikz,pgf,pgfplots}                    %Grafiken zeichnen
\usepackage{epstopdf}
\usepackage[ngerman]{babel}
\usepackage{mdframed}	                            %Box			%\begin{mdframed} 		und			%\end{mdframed}
\usepackage{fancyhdr}	                            %Header
\usepackage[margin=2cm]{geometry}	                %Seitenrand
\usepackage{tabu}                                 %Bessere Tabellen
\usepackage{esvect}                               %Schönere Vektoren
\usepackage{listings}                             %Für Pseudocode/Code...
\usepackage{esdiff}				%Dérivées french style
\usepackage{color}                                %...Und dessen Einfärbung
\usepackage{enumitem}                             %Angepasstes enumerieren
\usepackage{enumerate}                            %Zeichen der Enumerierung beliebig wählbar
\usepackage[hidelinks=true]{hyperref}
\usepackage[makeroom]{cancel}                     %Durchgestrichene Terme mit \cancel{}
\usepackage{tkz-tab}                              %Für Monotonietabellen??
\usepackage{marvosym}                             %Symbole wie Blitze und co. ...
\usepackage{stackengine}                          %Zeichen übereinander "stapeln"
\usepackage{epigraph}                             %Fürs Titelblatt
\usepackage{tocloft,titletoc,titlesec}            %Fürs Inhaltsverzeichnis
\usepackage{flowfram}                              %Für die Seitenangabe
\usepackage{nopageno}                             %Same
\pgfplotsset{compat=1.14}
\usetikzlibrary{arrows}
%\usetikzlibrary{fit}


\usepackage{../STYLE}                             %STYLESHEET

\begin{document}
  \begin{titlepage}

\begin{center}
	\hfill  \\
	[3,5cm]

	\titlefont Schülerskript SMP\\
	[0,5cm]
\end{center}
\epigraph{Mach' dir keine Sorgen wegen deiner Schwierigkeiten mit der Mathematik. Ich kann dir versichern, daß meine noch größer sind.}
{\textit{Brief an ein Schulmädchen, 1943}\\ \textsc{Albert Einstein}}
\null\vfill
\vspace*{1cm}
\noindent
\hfill
\begin{minipage}{0.35\linewidth}
    \begin{flushright}
        \printauthor
    \end{flushright}
\end{minipage}
%
\begin{minipage}{0.02\linewidth}
    \rule{1pt}{100pt}
\end{minipage}
\titlepagedecoration
\begin{center}
	\small{\today} \\
\end{center}
\end{titlepage}

  \blankpage
  \setcounter{tocdepth}{1}
  \tableofcontents
  \begin{dynamiccontents*}{frameD-1a}
\begin{tikzpicture}
\draw(0,0) node [fill=titlepagecolor, minimum width=1cm, minimum height=1cm]{
{\sffamily\bfseries\large\color{white}\thepage}
};
\end{tikzpicture}
\end{dynamiccontents*}

\begin{dynamiccontents*}{frameD-1b}
\begin{tikzpicture}
\draw(0,0) node [fill=titlepagecolor, minimum width=1cm, minimum height=1cm]{
{\sffamily\bfseries\large\color{white}\thepage}
};
\end{tikzpicture}
\end{dynamiccontents*}

  %%%%%%%%%%%%%%%%%%%%%%%%%%%%%%%%%%%%%%%%%%%%%%%%%%%%%%%%%%%%%%%%%%%%%%%%%%%%%%
  %Kapitel bitte hier einbinden:
  \subfile{../kap1/kap1}
  \subfile{../kap2/kap2}
  \subfile{../kap3/kap3}
  \subfile{../kap4/kap4}
  \subfile{../kap12/kap12}
  \subfile{../kap13/kap13}
  \subfile{../kap11/kap11}
  \subfile{../kap5/kap5}
  \subfile{../kap6/kap6}
  \subfile{../kap8/kap8}
  \subfile{../kap7/kap7}
  \subfile{../kap9/kap9}
  \subfile{../kap10/kap10}
  %ANHANG: Physik
  \subfile{../kap14/kap14}

  %und so weiter
  %%%%%%%%%%%%%%%%%%%%%%%%%%%%%%%%%%%%%%%%%%%%%%%%%%%%%%%%%%%%%%%%%%%%%%%%%%%%%%
\end{document}
