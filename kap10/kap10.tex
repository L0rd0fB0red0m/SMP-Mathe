\chapter{Algorythmik}
\section{Algorithmen und Programmierung}
\begin{Definition}
	Algorithmen besitzen die folgenden charakteristischen Eigenschaften:
	\begin{enumerate}
		\item Eindeutigkeit: ein Algorithmus darf keine widersprüchliche Beschreibung haben. Diese muss eindeutig sein.
		\item Eindeutigkeit: ein Algorithmus darf keine widersprüchliche Beschreibung haben. Diese muss eindeutig sein.
		\item Ausführbarkeit: jeder Einzelschritt muss ausführbar sein.
		\item Finitheit (= Endlichkeit): die Beschreibung des Algorithmus muss endlich sein.
		\item Terminierung: nach endlich vielen Schritten muss der Algorithmus enden und ein Ergebnis liefern.
		\item Determiniertheit: der Algorithmus muss bei gleichen Voraussetzungen stets das gleiche Ergebnis liefern.
		\item   Determinismus: zu jedem Zeitpunkt der Ausführung besteht höchstens eine Möglichkeit der Fortsetzung. Der Folgeschritt ist also eindeutig bestimmt.
	\end{enumerate}
\end{Definition}
\\
Diese Eigenschaften können in der Mathematik genutzt werden, um Probleme zu lösen. Hierfür bedarf es einer einheitlichen Schreibweise.\\
Insbesondere vor dem Abitur stehen den Schülern mehrere Möglichkeiten zur Verfügung, die im Folgenden behandelt werden.
\subsection{Pseudocode}
\subsection{Python}
Programmiersprache, bekannt durch ihre einfach verständliche Syntax, sie gilt als "high-level", was sie zu einer auf die (gesprochene) Sprache angepasste Sprache macht. Sie ist somit gerade für Einsteiger interessant und eignet sich dennoch für größere Projekte. Ein weiterer Vorteil ist die mitllerweile allgegenwärtige Präsenz der Sprache, mittlerweile wird sie auch für Apps und Webdeveloppement verwendet.
\subsection{Syntax}
Folgendes macht die Syntax Pythons aus:\\
\begin{enumerate}
\item gute Lesbarkeit des Quellcodes\\
\item  englische Schlüsselwörter\\
\item wenig syntaktische Konstruktionen
\end{enumerate}
\subsection{Funktionsweise}
Python verwendet Schleifen und Verzweigungen.\\
\begin{lstlisting}[language=Python]
	#Schleifen:
	for: #(Wiederholung über Elemente einer Sequenz (Zahl, Liste, Zeit...))
		#Inhalt der Schleife
	while: #(Wiederholung, solange ein logischer Ausdruck wahr ist)
		#Inhalt der Schleife
	#Verzweigungen:
	if: #(prüft,ob ein logischer Ausdruck wahr ist)
		#Inhalt der Verzweigung
	elif: #(prüft,ob ein anderer logischer Ausdruck wahr ist)
		#Inhalt der Verzweigung
	else: #(letzter Fall, tritt ein, wenn keine der obigen Bedingungen erfüllt wurde)
		#Inhalt der Verzweigung
\end{lstlisting}
\subsection{Beispiel}
Man nehme, den Algorythmus, der die Fibonacci-Sequenz bis zum $n$-ten Glied generiert:\\
\begin{lstlisting}[language=Python]
	a = 0
	b = 1
	n = 10
	for iteration in range(n):
		print(a)
		a = b
		b = a+b
\end{lstlisting}
KP, obs läuft
\section{Algorithmen und mathematische Anwendungen}
\subsection{Newton-Rhapson Verfahren}
\subsection{Allgemeines Iterationsverfahren}
