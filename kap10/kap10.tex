\chapter{Algorythmik}
\section{Algorithmen und Programmierung}
\begin{Definition}
	Algorithmen besitzen die folgenden charakteristischen Eigenschaften:
	\begin{enumerate}
		\item Eindeutigkeit: ein Algorithmus darf keine widersprüchliche Beschreibung haben. Diese muss eindeutig sein.
		\item Eindeutigkeit: ein Algorithmus darf keine widersprüchliche Beschreibung haben. Diese muss eindeutig sein.
		\item Ausführbarkeit: jeder Einzelschritt muss ausführbar sein.
		\item Finitheit (= Endlichkeit): die Beschreibung des Algorithmus muss endlich sein.
		\item Terminierung: nach endlich vielen Schritten muss der Algorithmus enden und ein Ergebnis liefern.
		\item Determiniertheit: der Algorithmus muss bei gleichen Voraussetzungen stets das gleiche Ergebnis liefern.
		\item   Determinismus: zu jedem Zeitpunkt der Ausführung besteht höchstens eine Möglichkeit der Fortsetzung. Der Folgeschritt ist also eindeutig bestimmt.
	\end{enumerate}
\end{Definition}
\\
Diese Eigenschaften kann man in der Mathematik genutzt werden, um Probleme zu lösen. Hierfür bedarf es einer einheitlichen Schreibweise.\\
Insbesondere vor dem Abitur stehen den Schülern mehrere Möglichkeiten zur Verfügung, die im Folgenden behandelt werden.
\subsection{Pseudocode}
\subsection{Python}
\begin{lstlisting}[language=Python]
	import turtle as t
	print("I love U")
	print(69)
	while True:
		khdfslh

\end{lstlisting}

\section{Algorithmen und mathematische Anwendungen}
\subsection{Newton-Rhapson Verfahren}
\subsection{Allgemeines Iterationsverfahren}
