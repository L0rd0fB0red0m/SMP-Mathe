\documentclass[../MAIN/main.tex]{subfiles}
\begin{document}
\chapter{Matrizen}
\chapterauthor{Bruno}

	\section{Überblick und Grundrechenarten}


\subsection{Definition und Einordnung}

\begin{Definition}
Eine Matrix ist eine rechteckige Anordnung von Elementen. Sie tauchen in fast allen Gebieten der Mathematik und der Mechanik auf und erleichtern Rechen- und Gedankenvorgänge. Eine Matrix ist eindeutig durch ihre \textbf{Komponenten} und ihr \textbf{Typ} definiert: Eine Matrix mit $m$ Zeilen und $n$ Spalten wird $m \times n$-Matrix genannt. Stammen ihre Komponenten aus einem Körper $K$, spricht man von einer \textit{Matrix über K}.\\

\tikzset{
        round/.style = { rounded corners=2mm },
        node style ge/.style={circle},
}
\begin{tikzpicture}
\tikzset{BarreStyle/.style =   {opacity=.3,line width=5 mm,line cap=round,color=#1}}

    \draw (0,0) node {};
    \draw[round] (6.4,1.7) rectangle (5.65,-1.72) [color=titlepagecolor];
    \draw[round] (10.75,1.7) rectangle (5.65,1.2) [color=titlepagecolor];
     \draw [BarreStyle=titlepagecolor]  (5.9,1.45) -- (10.5,-1.65) ;

    \draw (7.85,0) node {$A= \begin{pmatrix}
a_{1,1}&a_{1,2}&...&a_{1,j}&...&a_{1,n}\\\\
a_{2,1}&a_{2,2}&...&a_{2,j}&...&a_{2,n}\\
&&\vdots&&&\\
a_{i,1}&a_{i,2}&...&a_{i,j}&...&a_{3,n}\\
&&\vdots&&&\\
a_{m,1}&a_{m,2}&...&a_{m,j}&...&a_{m,n}\\
\end{pmatrix}$};
    \draw[<-] (5.1,1.5) -- (4,1.5) node[color=titlepagecolor,left] {Zeile};
    \draw[<-] (6,1.9) -- (6,2.5) node[color=titlepagecolor ] {Spalte};
    \draw[<-] (5.4,1.7) -- (5.1,1.9) -- (4,1.9) node[color=titlepagecolor,left] {Hauptdiagonale};
\end{tikzpicture}\\\\
\end{Definition}




\begin{Bemerkung}
Eine Matrix, die aus nur einer Spalte oder nur einer Zeile besteht, wird auch Vektor genannt.\\\\
Ganz formal gesehen ist eine Matrix eine Funktion, die jedem Funktionswert $(i;j)$ einen Eintrag zuordnet. Matrizen bilden mit der Matrizenaddition und der Skalarmultiplikation einen abgeschlossenen Körper.
\end{Bemerkung}


\begin{GTR-Tipp}
Im Matrix Menü: $[\text{matrix}] + [\text{MATH}]$ kann man mit dem Befehl...\\
\begin{enumerate}
\item ... ref$[B]$ die \textbf{Dreiecksform} einer Matrix $B$ ausrechnen lassen:\\\\
$$B= \begin{pmatrix}
b_{1,1}&b_{1,2}&b_{1,3}&b_{1,4}\\
0&b_{2,2}&b_{2,3}&b_{2,4}\\
0&0&b_{3,3}&b_{3,4}\\
0&0&0&b_{4,3}\\
\end{pmatrix}$$\\
\item ... rref$[C]$ die \textbf{Diagonalform} einer Matrix $C$ ausrechnen lassen. Die einzigen Nicht-Null Einträge sind auf der Hauptdiagonalen:\\\\
$$C= \begin{pmatrix}
c_{1,1}&0&0&0\\
0&c_{2,2}&0&0\\
0&0&c_{3,3}&0\\
0&0&0&c_{4,4}\\
\end{pmatrix}$$\\
\end{enumerate}
\end{GTR-Tipp}


\subsection{Addition und Multiplikation}

\subsubsection{Matritzenaddition}

Matrizen können addiert werden, wenn sie vom gleichen Typ sind. Eine $m\times n$ Matrix $A$ kann nur mit einer $m\times n$ Matrix $B$ addiert werden. Hierfür wird jede Komponente $a_{i,j}$ mit ihrem zugehörigen Komponenten $b_{i,j}$ addiert:
$$A+B = \begin{pmatrix} a_{1,1} & a_{1,2} & a_{1,3} \\ a_{2,1} & a_{2,2} & a_{2,3} \end{pmatrix} + \begin{pmatrix} b_{1,1} & b_{1,2} & b_{1,3} \\ b_{2,1} & b_{2,2} & b_{2,3} \end{pmatrix}  =  \begin{pmatrix} a_{1,1} + b_{1,1} & a_{1,2} + b_{1,2} & a_{1,3} + b_{1,3} \\ a_{2,1} + b_{2,1} & a_{2,2} + b_{2,2} & a_{2,3} + b_{2,3} \end{pmatrix}$$
Das Neutrale Element der Addition ist logischerweise eine leere Matrix, das inverse findet man mit der Skalierung $\cdot (-1)$

\subsubsection{Skalarmultiplikation}

Die Skalarmultiplikation, die man aus Vektoren kennt, kann auf Matrizen erweitert werden: 
$$k \cdot A = k \cdot \begin{pmatrix} a_{1,1} & a_{1,2} & a_{1,3} \\ a_{2,1} & a_{2,2} & a_{2,3} \end{pmatrix} = \begin{pmatrix} k \cdot a_{1,1} & k \cdot a_{1,2} & k \cdot a_{1,3} \\ k \cdot a_{2,1} & k \cdot a_{2,2} & k \cdot a_{2,3} \end{pmatrix} $$

\subsubsection{Matrizenmultiplikation}

BLABLABLA kommt noch $$c_{i,j} = \sum_{k=1}^{m} a_{i,k} \cdot b_{k,j}$$



	\section{Lineare Gleichungssysteme und Gaußalgorithmus}

Lineare Gleichungssysteme lassen sich aufwendig mit Einsetzungsverfahren oder Additionsverfahren l"osen, Carl Friedrich Gauß (1777-1855) hat ein Algorithmus erfunden, mit dem sie sich ohne Taschenrechner leicht und relativ schnell l"osen lassen.\\
Am Besten wird dieser mit einem Beispiel Erl"autert:\\
\\
$\Rightarrow$ $\left\{ \begin{array}{rccl}
4x+3y+z&=&13& (1)\\
2x-5y+3z& =& 1 &(2)\\
7x-y-2z&=&-1&(3)\\
\end{array}\right.$ \qquad \{ $1\cdot (1) -2\cdot (2)$ \}  und \{ $7\cdot (1) -4\cdot (3)$ \}  \qquad \qquad $D=\R^3$\\
\\
\\
Hier versucht man in Zeile (2) und (3) die erste Variabel zu eliminieren\\
\\
$\Leftrightarrow$ $\left\{ \begin{array}{rcl}
4x+3y+z&=&13\\
0x +13y -5z&=& 11\\
0x +25y +15z&=& 95\\
\end{array}\right.$ \qquad \{ $25\cdot(2) -13\cdot(3)$ \}  \\
\\
\\
Jetzt versucht man die zweite Variabel in der dritten Gleichung zu eliminieren\\
\\
$\Leftrightarrow$ $\left\{ \begin{array}{rcl}
4x+3y+z&=&13\\
0x +13y -5z&=& 11\\
0x+0y-320z&=&-960 $\qquad$ \Leftrightarrow z=3 \\
\end{array}\right.$\\
\\
\\
Jetzt wird eingesetzt\\
\\
$\Leftrightarrow$ $\left\{ \begin{array}{rcl}
4x+3y+z&=&13\\
0x+13y-5\cdot3&=&11 $\qquad$ \Leftrightarrow y=2\\
0x+0y+z&=&3\\
\end{array}\right.$\\
\\
\\
$\Leftrightarrow$ $\left\{ \begin{array}{rcl}
x+0y+0z&=&1\\
0x+y+0z&=&2\\
0x+0y+z&=&3\\
\end{array}\right.$\\
\\
\\
$\mathbb{L}=\{(1;2;3) \}$ Die Lösungsmenge wird als n-Tupel (geordente Objekte) alphabetisch sortiert.

	\section{LGS mit dem Taschenrechner l"osen}

	\subsection{Eindeutig l"osbare lineare Gleichungssysteme}

Ein lineares Gleichungssystem l"asst sich sehr viel schneller mit dem Taschgenrechner l"osen:\\

$\Rightarrow$ $\left\{ \begin{array}{rcl}
4x+3y+z&=&13\\
2x-5y+3z& =& 1\\
7x-y-2z&=&-1\\
\end{array}\right.$\\
\\
Hierf"ur geht man beim Taschenrechner auf [matrix] und auf [edit]. Dann gibt man seine Matrix (hier als Beispiel) ein:\\
\\
$\Rightarrow$ $\left\vert \begin{array}{rccl}
4&3&1&13\\
2&-5&3& 1 \\
7&-1&-2&-1\\
\end{array}\right\vert$\\
\\
Dann geht man wieder in den rechnen-Modus und gibt ein:  [matrix], dann geht man auf [math], [rref]. dann geht man nochmal auf [matrix], [A] (die gerade bearbeitete Matrix):\\
\\
$\Leftrightarrow$ $\left\vert \begin{array}{rccl}
1&0&0&1\\
0&1&0&2 \\
0&0&1&3\\
\end{array}\right\vert$ \qquad $\Rightarrow \mathbb{L}=\{(1;2;3) \}$ \\

Die L"osungsmenge wird als n-Tupel angegeben.\\
Wenn sich Werte mit Kommazahlen ergeben, ist es n"utzlich, im Taschenrechner [Math] $+$ [1](Frac) einzugeben, um sich die Werte in Br"uchen anzeigen zu lassen.\\

	\subsection{Nicht eindeutig l"osbare lineare Gleichungssysteme}

Oft begegnen einem auch unterbestimmte LGS, sei es in der Geometrie (Zwei Ebenengleichungen, die in einem Gleichungssystem als L"osung die Schnittgerade ergeben) oder in anderen Teilbereichen. Sie sind auch recht aufwendig von Hand zu l"osen, deshalb hier den schnelleren GTR-L"osungsweg:\\

$\Rightarrow$ $\left\{ \begin{array}{rcl}
x_{1}-2x_{2}+0x_{3}&=&10\\
x_{1}-x_{2}-x_{3}& =& 5\\
\end{array}\right.$\\
\\
Unterbestimmte LGS erkennt man daran, dass es mehr unbekannte als Gleichungen gibt:\\

$\Rightarrow$ $\left\vert \begin{array}{rccl}
1&-2&0&10\\
1&-1&-1& 5 \\
\end{array}\right\vert$\\
\\
\\
$\Leftrightarrow$ $\left\vert \begin{array}{rccl}
1&-2&0&10\\
0&0&1&5 \\
\end{array}\right\vert$  \qquad $\Rightarrow \mathbb{L}=\{(2t+10;t;t+5|t\in\R) \}$ \\

Auch hier wird die L"osungsmenge als n-Tupel angegeben, in Abh"angigkeit eines Faktors, dessen Wertebereich in der L"osungsmenge ebenfalls angegeben werden muss.\\\\
\end{document}
