\documentclass[../MAIN/main.tex]{subfiles}
\begin{document}
\chapter{Matrizen}
\chapterauthor{Bruno}

	\section{Überblick und Grundrechenarten}


\subsection{Arten von Matritzen}

\begin{Definition}
Matritzen sind ........ und bilden einen abgeschlossenen Körper. Sie wird durch ihre Koeffizienten und ihr Dimension definiert: Eine Matrix der Dimension $m \times n$ besitzt $m$ Zeilen und $n$ Spalten.\\

$$A= \left( \begin{array}{rccccl}
a_{1,1}&a_{1,2}&...&a_{1,j}&...&a_{1,n}\\
a_{2,1}&a_{2,2}&...&a_{2,j}&...&a_{2,n}\\
&&\vdots&&&\\
a_{i,1}&a_{i,2}&...&a_{i,j}&...&a_{,n}\\
&&\vdots&&&\\
a_{m,1}&a_{m,2}&...&a_{m,j}&...&a_{m,n}\\
\end{array}\right)$$\\
Sie wird quadratische Matrix genannt wenn $m=n$.
\end{Definition}

\begin{GTR-Tipp}
Im Matrix Menü: $[\text{matrix}] + [\text{MATH}]$ kann man mit dem Befehl...\\
\begin{enumerate}
\item ... rref$[B]$ die \textbf{Dreiecksform} einer Matrix $B$ ausrechnen lassen:\\\\
$$B= \left( \begin{array}{rccl}
b_{1,1}&b_{1,2}&b_{1,3}&b_{1,4}\\
0&b_{2,2}&b_{2,3}&b_{2,4}\\
0&0&b_{3,3}&b_{3,4}\\
0&0&0&b_{4,3}\\
\end{array}\right)$$\\
\item ... rref$[C]$ die \textbf{Diagonalform} einer Matrix $C$ ausrechnen lassen:\\\\
$$C= \left( \begin{array}{rccl}
c_{1,1}&0&0&0\\
0&c_{2,2}&0&0\\
0&0&c_{3,3}&0\\
0&0&0&c_{4,4}\\
\end{array}\right)$$\\
\end{enumerate}
\end{GTR-Tipp}


\subsection{Addition und Multiplikation}
ADDITION MATRIX
MULTIPLIKATION MATRIXXXXX


	\section{Lineare Gleichungssysteme und Gaußalgorithmus}

Lineare Gleichungssysteme lassen sich aufwendig mit Einsetzungsverfahren oder Additionsverfahren l"osen, Carl Friedrich Gauß (1777-1855) hat ein Algorithmus erfunden, mit dem sie sich ohne Taschenrechner leicht und relativ schnell l"osen lassen.\\
Am Besten wird dieser mit einem Beispiel Erl"autert:\\
\\
$\Rightarrow$ $\left\{ \begin{array}{rccl}
4x+3y+z&=&13& (1)\\
2x-5y+3z& =& 1 &(2)\\
7x-y-2z&=&-1&(3)\\
\end{array}\right.$ \qquad \{ $1\cdot (1) -2\cdot (2)$ \}  und \{ $7\cdot (1) -4\cdot (3)$ \}  \qquad \qquad $D=\R^3$\\
\\
\\
Hier versucht man in Zeile (2) und (3) die erste Variabel zu eliminieren\\
\\
$\Leftrightarrow$ $\left\{ \begin{array}{rcl}
4x+3y+z&=&13\\
0x +13y -5z&=& 11\\
0x +25y +15z&=& 95\\
\end{array}\right.$ \qquad \{ $25\cdot(2) -13\cdot(3)$ \}  \\
\\
\\
Jetzt versucht man die zweite Variabel in der dritten Gleichung zu eliminieren\\
\\
$\Leftrightarrow$ $\left\{ \begin{array}{rcl}
4x+3y+z&=&13\\
0x +13y -5z&=& 11\\
0x+0y-320z&=&-960 $\qquad$ \Leftrightarrow z=3 \\
\end{array}\right.$\\
\\
\\
Jetzt wird eingesetzt\\
\\
$\Leftrightarrow$ $\left\{ \begin{array}{rcl}
4x+3y+z&=&13\\
0x+13y-5\cdot3&=&11 $\qquad$ \Leftrightarrow y=2\\
0x+0y+z&=&3\\
\end{array}\right.$\\
\\
\\
$\Leftrightarrow$ $\left\{ \begin{array}{rcl}
x+0y+0z&=&1\\
0x+y+0z&=&2\\
0x+0y+z&=&3\\
\end{array}\right.$\\
\\
\\
$\mathbb{L}=\{(1;2;3) \}$ Die Lösungsmenge wird als n-Tupel (geordente Objekte) alphabetisch sortiert.

	\section{LGS mit dem Taschenrechner l"osen}

	\subsection{Eindeutig l"osbare lineare Gleichungssysteme}

Ein lineares Gleichungssystem l"asst sich sehr viel schneller mit dem Taschgenrechner l"osen:\\

$\Rightarrow$ $\left\{ \begin{array}{rcl}
4x+3y+z&=&13\\
2x-5y+3z& =& 1\\
7x-y-2z&=&-1\\
\end{array}\right.$\\
\\
Hierf"ur geht man beim Taschenrechner auf [matrix] und auf [edit]. Dann gibt man seine Matrix (hier als Beispiel) ein:\\
\\
$\Rightarrow$ $\left\vert \begin{array}{rccl}
4&3&1&13\\
2&-5&3& 1 \\
7&-1&-2&-1\\
\end{array}\right\vert$\\
\\
Dann geht man wieder in den rechnen-Modus und gibt ein:  [matrix], dann geht man auf [math], [rref]. dann geht man nochmal auf [matrix], [A] (die gerade bearbeitete Matrix):\\
\\
$\Leftrightarrow$ $\left\vert \begin{array}{rccl}
1&0&0&1\\
0&1&0&2 \\
0&0&1&3\\
\end{array}\right\vert$ \qquad $\Rightarrow \mathbb{L}=\{(1;2;3) \}$ \\

Die L"osungsmenge wird als n-Tupel angegeben.\\
Wenn sich Werte mit Kommazahlen ergeben, ist es n"utzlich, im Taschenrechner [Math] $+$ [1](Frac) einzugeben, um sich die Werte in Br"uchen anzeigen zu lassen.\\

	\subsection{Nicht eindeutig l"osbare lineare Gleichungssysteme}

Oft begegnen einem auch unterbestimmte LGS, sei es in der Geometrie (Zwei Ebenengleichungen, die in einem Gleichungssystem als L"osung die Schnittgerade ergeben) oder in anderen Teilbereichen. Sie sind auch recht aufwendig von Hand zu l"osen, deshalb hier den schnelleren GTR-L"osungsweg:\\

$\Rightarrow$ $\left\{ \begin{array}{rcl}
x_{1}-2x_{2}+0x_{3}&=&10\\
x_{1}-x_{2}-x_{3}& =& 5\\
\end{array}\right.$\\
\\
Unterbestimmte LGS erkennt man daran, dass es mehr unbekannte als Gleichungen gibt:\\

$\Rightarrow$ $\left\vert \begin{array}{rccl}
1&-2&0&10\\
1&-1&-1& 5 \\
\end{array}\right\vert$\\
\\
\\
$\Leftrightarrow$ $\left\vert \begin{array}{rccl}
1&-2&0&10\\
0&0&1&5 \\
\end{array}\right\vert$  \qquad $\Rightarrow \mathbb{L}=\{(2t+10;t;t+5|t\in\R) \}$ \\

Auch hier wird die L"osungsmenge als n-Tupel angegeben, in Abh"angigkeit eines Faktors, dessen Wertebereich in der L"osungsmenge ebenfalls angegeben werden muss.\\\\
\end{document}
