\chapter{Matrizen}

	\section{Lineare Gleichungssysteme und Gau"salgorythmus}

Lineare Gleichungssysteme lassen sich aufwendig mit Einsetzungsverfahren oder Additionsverfahren l"osen, Carl Friedrich Gauß (1777-1855) hat ein ''Algorythmus'' erfunden, mit dem sie sich ohne Taschenrechner leicht und relativ schnell l"osen lassen.\\
Am Besten wird dieser mit einem Beispiel Erl"autert:\\
\\
$\Rightarrow$ $\left\{ \begin{array}{rccl}
4x+3y+z&=&13& (1)\\
2x-5y+3z& =& 1 &(2)\\
7x-y-2z&=&-1&(3)\\
\end{array}\right.$ \qquad \{ $1\cdot (1) -2\cdot (2)$ \}  und \{ $7\cdot (1) -4\cdot (3)$ \}  \qquad \qquad $D=\R^3$\\
\\
\\
Hier versucht man in Zeile (2) und (3) die erste Variabel zu eliminieren\\
\\
$\Leftrightarrow$ $\left\{ \begin{array}{rcl}
4x+3y+z&=&13\\
0x +13y -5z&=& 11\\
0x +25y +15z&=& 95\\
\end{array}\right.$ \qquad \{ $25\cdot(2) -13\cdot(3)$ \}  \\
\\
\\
Jetzt versucht man die zweite Variabel in der dritten Gleichung zu eliminieren\\
\\
$\Leftrightarrow$ $\left\{ \begin{array}{rcl}
4x+3y+z&=&13\\
0x +13y -5z&=& 11\\
0x+0y-320z&=&-960 $\qquad$ \Leftrightarrow z=3 \\
\end{array}\right.$\\
\\
\\
Jetzt wird eingesetzt\\
\\
$\Leftrightarrow$ $\left\{ \begin{array}{rcl}
4x+3y+z&=&13\\
0x+13y-5\cdot3&=&11 $\qquad$ \Leftrightarrow y=2\\
0x+0y+z&=&3\\
\end{array}\right.$\\
\\
\\
$\Leftrightarrow$ $\left\{ \begin{array}{rcl}
x+0y+0z&=&1\\
0x+y+0z&=&2\\
0x+0y+z&=&3\\
\end{array}\right.$\\
\\
\\
$\mathbb{L}=\{(1|2|3) \}$ Die Lösungsmenge wird als n-Tupel alphabetisch sortiert.

	\section{LGS mit dem Taschenrechner l"osen}
	
Ein lineares Gleichungssystem l"asst sich sehr viel schneller mit dem Taschgenrechner l"osen:\\

$\Rightarrow$ $\left\{ \begin{array}{rcl}
4x+3y+z&=&13\\
2x-5y+3z& =& 1\\
7x-y-2z&=&-1\\
\end{array}\right.$\\
\\
Hierf"ur geht man beim Taschenrechner auf [matrix] und auf [edit]. Dann gibt man seine Matrix (hier als Beispiel) ein:\\
\\
$\Rightarrow$ $\left\vert \begin{array}{rccl}
4&3&1&13\\
2&-5&3& 1 \\
7&-1&-2&-1\\
\end{array}\right\vert$\\
\\
Dann geht man wieder in den rechnen-Modus und gibt ein:  [matrix], dann geht man auf [math], [rref]. dann geht man nochmal auf [matrix], [A] (die gerade bearbeitete Matrix):\\
\\
$\Rightarrow$ $\left\vert \begin{array}{rccl}
1&0&0&1\\
0&1&0&2 \\
0&0&1&3\\
\end{array}\right\vert$ \qquad $\Rightarrow \mathbb{L}=\{(1|2|3) \}$ \\ 

Wenn sich Werte mit Kommazahlen ergeben, ist es n�tzlich, im Taschenrechner [Math] $+$ [1](Frac) einzugeben, um sich die Werte in Br�chen anzeigen zu lassen.\\

