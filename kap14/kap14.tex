\documentclass[../MAIN/main.tex]{subfiles}
\begin{document}
\chapter{ANHANG: Physik}
\chapterauthor{Bruno}


\section{La physique des particules}

Le modele standard de la physique dans sa beauté incontestée.\\\\

\usetikzlibrary{calc,positioning,shadows.blur,decorations.pathreplacing}

\tikzset{%
        brace/.style = { decorate, decoration={brace, amplitude=5pt} },
       mbrace/.style = { decorate, decoration={brace, amplitude=5pt, mirror} },
        label/.style = { black, midway, scale=0.5, align=center },
     toplabel/.style = { label, above=.5em, anchor=south },
    leftlabel/.style = { label,rotate=-90,left=.5em,anchor=north },
  bottomlabel/.style = { label, below=.5em, anchor=north },
        force/.style = { rotate=-90,scale=0.4 },
        round/.style = { rounded corners=2mm },
       legend/.style = { right,scale=0.4 },
        nosep/.style = { inner sep=0pt },
   generation/.style = { anchor=base }
}

\newcommand\particle[7][white]{%
  \begin{tikzpicture}[x=1cm, y=1cm]
    \path[fill=#1,blur shadow={shadow blur steps=5}] (0.1,0) -- (0.9,0)
        arc (90:0:1mm) -- (1.0,-0.9) arc (0:-90:1mm) -- (0.1,-1.0)
        arc (-90:-180:1mm) -- (0,-0.1) arc(180:90:1mm) -- cycle;
    \ifstrempty{#7}{}{\path[fill=purple!50!white]
        (0.6,0) --(0.7,0) -- (1.0,-0.3) -- (1.0,-0.4);}
    \ifstrempty{#6}{}{\path[fill=green!50!black!50] (0.7,0) -- (0.9,0)
        arc (90:0:1mm) -- (1.0,-0.3);}
    \ifstrempty{#5}{}{\path[fill=orange!50!white] (1.0,-0.7) -- (1.0,-0.9)
        arc (0:-90:1mm) -- (0.7,-1.0);}
    \draw[\ifstrempty{#2}{dashed}{black}] (0.1,0) -- (0.9,0)
        arc (90:0:1mm) -- (1.0,-0.9) arc (0:-90:1mm) -- (0.1,-1.0)
        arc (-90:-180:1mm) -- (0,-0.1) arc(180:90:1mm) -- cycle;
    \ifstrempty{#7}{}{\node at(0.825,-0.175) [rotate=-45,scale=0.2] {#7};}
    \ifstrempty{#6}{}{\node at(0.9,-0.1)  [nosep,scale=0.17] {#6};}
    \ifstrempty{#5}{}{\node at(0.9,-0.9)  [nosep,scale=0.2] {#5};}
    \ifstrempty{#4}{}{\node at(0.1,-0.1)  [nosep,anchor=west,scale=0.25]{#4};}
    \ifstrempty{#3}{}{\node at(0.1,-0.85) [nosep,anchor=west,scale=0.3] {#3};}
    \ifstrempty{#2}{}{\node at(0.1,-0.5)  [nosep,anchor=west,scale=1.5] {#2};}
  \end{tikzpicture}
}

\begin{center}
  \begin{tikzpicture}[x=1.2cm, y=1.2cm]
    \draw[round] (-0.5,0.5) rectangle (4.4,-1.5);
    \draw[round] (-0.6,0.6) rectangle (5.0,-2.5);
    \draw[round] (-0.7,0.7) rectangle (5.6,-3.5);

    \node at(0, 0)   {\particle[gray!20!white]
                     {$u$}        {up}       {$2.3$ MeV}{1/2}{$2/3$}{R/G/B}};
    \node at(0,-1)   {\particle[gray!20!white]
                     {$d$}        {down}    {$4.8$ MeV}{1/2}{$-1/3$}{R/G/B}};
    \node at(0,-2)   {\particle[gray!20!white]
                     {$e$}        {éléctron}       {$511$ keV}{1/2}{$-1$}{}};
    \node at(0,-3)   {\particle[gray!20!white]
                     {$\nu_e$}    {$e$ neutrino}         {$<2$ eV}{1/2}{}{}};
    \node at(1, 0)   {\particle
                     {$c$}        {charm}   {$1.28$ GeV}{1/2}{$2/3$}{R/G/B}};
    \node at(1,-1)   {\particle
                     {$s$}        {strange}  {$95$ MeV}{1/2}{$-1/3$}{R/G/B}};
    \node at(1,-2)   {\particle
                     {$\mu$}      {muon}         {$105.7$ MeV}{1/2}{$-1$}{}};
    \node at(1,-3)   {\particle
                     {$\nu_\mu$}  {$\mu$ neutrino}    {$<190$ keV}{1/2}{}{}};
    \node at(2, 0)   {\particle
                     {$t$}        {top}    {$173.2$ GeV}{1/2}{$2/3$}{R/G/B}};
    \node at(2,-1)   {\particle
                     {$b$}        {bottom}  {$4.7$ GeV}{1/2}{$-1/3$}{R/G/B}};
    \node at(2,-2)   {\particle
                     {$\tau$}     {tau}          {$1.777$ GeV}{1/2}{$-1$}{}};
    \node at(2,-3)   {\particle
                     {$\nu_\tau$} {$\tau$ neutrino}  {$<18.2$ MeV}{1/2}{}{}};
    \node at(3,-3)   {\particle[orange!20!white]
                     {$W^{\hspace{-.3ex}\scalebox{.5}{$\pm$}}$}
                                  {}              {$80.4$ GeV}{1}{$\pm1$}{}};
    \node at(4,-3)   {\particle[orange!20!white]
                     {$Z$}        {}                    {$91.2$ GeV}{1}{}{}};
    \node at(3.5,-2) {\particle[green!50!black!20]
                     {$\gamma$}   {photon}                        {}{1}{}{}};
    \node at(3.5,-1) {\particle[purple!20!white]
                     {$g$}        {gluon}                    {}{1}{}{color}};
    \node at(5,0)    {\particle[gray!50!white]
                     {$H$}        {Higgs}              {$125.1$ GeV}{0}{}{}};
    \node at(6.1,-3) {\particle
                     {}           {graviton}                       {}{}{}{}};

    \node at(4.25,-0.5) [force]      {interaction nucléaire forte (couleur)};
    \node at(4.85,-1.5) [force]    {interaction éléctromagnétique (charge)};
    \node at(5.45,-2.4) [force] {interaction nucléaire faible (isospin faible)};
    \node at(6.75,-2.5) [force]        {interaction gravitationelle (masse)};

    \draw [<-] (2.5,0.3)   -- (2.7,0.3)          node [legend] {charge};
    \draw [<-] (2.5,0.15)  -- (2.7,0.15)         node [legend] {couleurs};
    \draw [<-] (2.05,0.25) -- (2.3,0) -- (2.7,0) node [legend]   {masse};
    \draw [<-] (2.5,-0.3)  -- (2.7,-0.3)         node [legend]   {spin};

    \draw [mbrace] (-0.8,0.5)  -- (-0.8,-1.5)
                   node[leftlabel] {6 quarks\\(+6 anti-quarks)};
    \draw [mbrace] (-0.8,-1.5) -- (-0.8,-3.5)
                   node[leftlabel] {6 leptons\\(+6 anti-leptons)};
    \draw [mbrace] (-0.5,-3.6) -- (2.5,-3.6)
                   node[bottomlabel]
                   {12 fermions\\(+12 anti-fermions)\\ masse++ $\to$};
    \draw [mbrace] (2.5,-3.6) -- (5.5,-3.6)
                   node[bottomlabel] {5 bosons\\(+1 charge opposée $W$)};

    \draw [brace] (-0.5,.8) -- (0.5,.8) node[toplabel]         {matiere ordinaire};
    \draw [brace] (0.5,.8)  -- (2.5,.8) node[toplabel]         {matiere instable};
    \draw [brace] (2.5,.8)  -- (4.5,.8) node[toplabel]          {transmetteurs de force};
    \draw [brace] (4.5,.8)  -- (5.5,.8) node[toplabel]       {bosons\\Goldstone};
    \draw [brace] (5.5,.8)  -- (7,.8)   node[toplabel] {en dehors\\du modèle standard};

    \node at (0,1.3)   [génération] {1\tiny ère};
    \node at (1,1.3)   [génération] {2\tiny ème};
    \node at (2,1.3)   [génération] {3\tiny ème};
    \node at (2.8,1.3) [génération] {\tiny génération};
  \end{tikzpicture}
\end{center}

\section{Interaction gravitationelle}

\begin{Definition}
L'interaction gravitationnelle est une force toujours \textcolor{red}{attractive} qui agit sur tout ce qui poss�de une masse, mais avec une intensité extr�mement faible (c'est l'interaction la plus faible). Son domaine d'action est \textcolor{red}{l'infini}.\\
Un corps est considéré ponctuel si sa taille $\leq \dfrac{\text{distance d'observation}}{100}$
\\\\
$$ \vv{F_{g}} =  -\dfrac{G \cdot m_{a} \cdot m_{b}}{r^2} \cdot \vv{u_{AB}}$$  \\

Si: \qquad \qquad $r=$AB \qquad \qquad $G=6,67\cdot10^{-11}$(S.I) \qquad \qquad $\vv{u_{AB}} \to$ vecteur normé
\end{Definition}

\subsection{Le champ de gravitation}

\begin{Definition}
Tout objet de Masse M et d'origine spaciale $O$ crée autour de lui un champ gravitationnel.\\
En un point quelconque $P$, ce champ s'écrit $\vv{\mathcal{G}}_{(P)}$. \\
Un deuxi�me objet de masse $m$ placé en ce point P est soumis a la force de gravitation:
$$\vv{F}_{O/P} = m \cdot \vv{\mathcal{G}}_{(P)} $$
D'ou on peut tirer la formule pour le champ de gravitation d'un objet considéré ponctuel de masse $M$ a une distance $d$:
$${\mathcal{G}}_{o} = \dfrac{G\cdot M}{d^2} $$
\end{Definition}


\section{Interaction électromagnétique}

\begin{Definition}
L'interaction éléctromagn�tique est une force \textcolor{red}{attractive} ou \textcolor{red}{répulsive} qui agit sur tout ce qui poss�de une charge éléctrique. Son domaine d'action est également \textcolor{red}{l'infini.}
\end{Definition}

\subsection{Le champ électrique}

\begin{Definition}
La loi de Coulomb\\
Dans le vide, 2 corps ponctuels A et B de charges $q_{a}$ et $q_{b}$ exercent l'un sur l'autre des forces : \\
$$ \vv{F}_{A/B} = K \cdot \dfrac{q_{a}\cdot q_{b}}{r^2} \cdot \vv{U}_{A/B} $$
avec\\
$$K=\dfrac{1}{4\pi\varepsilon_{0}}= 9,0\cdot 10^9(S.I.)$$
$\varepsilon_{0}$: permittivité du vide (réponse d'un milieu donné a un champ électrique appliqué) ($8,85\cdot 10^9$)\\
\danger[5ex] \qquad $\vv{F}$ et $\vv{E}$ n'ont pas forcément le meme sens, cela dépend de la charge $q$\\\\

La relation entre force électrique et champ électrique s'exprime avec $q$ (Coulombs), charge de source:
$$\vv{F_{e}}=q\cdot \vv{E}$$
$$\Rightarrow F_{e}=|q|\cdot E$$\\\\
Le champ électrique s'exprime donc de cette maniere:\\\\
\begin{tikzpicture}
    \draw (0,0) node {};
    \draw (7.85,0) node {$\vv{E}= \dfrac{1}{4\pi\varepsilon_{0}}\cdot\dfrac{q}{r^2}\cdot\vv{U}_{A/B}$};
    \draw[line width=0.4pt] (5.9,-0.05) -- (5.1,-0.4) node[color=blue,left] {$J$};
    \draw[line width=0.4pt] (8.5,0.25) -- (9.2,0.7) node[color=blue,right] {$C$};
    \draw[line width=0.4pt] (8.51,-0.3) -- (9.7,-0.7) node[color=blue,right] {$m^{2}$};
\end{tikzpicture}\\\\
$\vv{E}$ va dans le sens des potentiels décroissants\\
Les lignes de champ sont tangentes aux vecteurs champ électrique tandis que les équipotentielles relient les points ou le champ électrique possede la meme valeur(intensité) \\\\
Dans un condensateur plan, le champ électrique est uniforme (lignes de champ paralleles) et la valeur du champ électrique est\\
\begin{tikzpicture}
    \draw (0,0) node {};
    \draw (7.85,0) node {$E=\dfrac{|U_{ab}|}{d}$};
    \draw[line width=0.4pt] (6.9,0) -- (5.9,-0.45) node[color=blue,left] {$J$};
    \draw[line width=0.4pt] (8.7,0.25) -- (9.5,0.25) node[color=blue,right] {$V$};
    \draw[line width=0.4pt] (8.6,-0.3) -- (9.7,-0.6) node[color=blue,right] {$m$};
\end{tikzpicture}\\
et, avec $Q$ (charge totale) et $S$ (surface des armatures)
\\
\definecolor{betterPurple}{rgb}{0.6,0,1}
\begin{tikzpicture}
    \draw (0,0) node {};
    \draw (7.85,0) node {$E=\dfrac{Q}{\varepsilon_{0} \cdot S}$};
    \draw[line width=0.4pt] (6.9,-0.05) -- (5.9,-0.45) node[color=blue,left] {$J$};
    \draw[line width=0.4pt] (8.5,0.25) -- (9.5,0.25) node[color=blue,right] {$C$};
    \draw[line width=0.4pt] (8.7,-0.3) -- (9.7,-0.7) node[color=blue,right] {$m^{2}$};
    \draw[line width=0.4pt] (7.7,-0.3) -- (6.7,-1) node[color=betterPurple,left] {$S.I.$};
\end{tikzpicture}

\end{Definition}




\subsection{Le champ magnétique}

\begin{Definition}

\textbf{Dans une bobine:} Soit $B_{i}$ l'intensité du champ magnétique, $I$ l'intensité du courant, $N$ le nombre de spires (jointives) et $l$ la longueur de la bobine,

\definecolor{betterPurple}{rgb}{0.6,0,1}
\begin{tikzpicture}
    \draw (0,0) node {};
    \draw (7.85,0) node {$B_{i}=\mu_{0}\cdot\dfrac{N\cdot I}{l}}$};
    \draw[line width=0.4pt] (6.75,-0.05) -- (5.9,-0.45) node[color=blue,left] {$T$};
    \draw[line width=0.4pt] (8.7,-0.3) -- (9.7,-0.7) node[color=blue,right] {$m$};
    \draw[line width=0.4pt] (7.7,-0.3) -- (6.7,-1) node[color=betterPurple,left] {$S.I.$};
\end{tikzpicture}

avec $\mu_{0}$ la perméabilité du vide
$$\mu_{0}=4\pi\cdot10^{-7} $$\\
On obtient deux bobines de Helmholz quand $d=R$; Le champ est donc uniforme

\end{Definition}

\section{Mouvement, vitesse et accélération d'un système physique}

\begin{Definition}

Dans la \texfbf{base de Frenet:} avec $a_{\tau}$ l'accélération tangentielle et $a_{\eta}$ l'accélération normale et $\rho$ le rayon de courbure,

\begin{tikzpicture}
    \draw (0,0) node {};
    \draw (7.85,0) node {$a_{\tau}=\dfrac{dV}{dt} \qquad \text{et} \qquad a_{\eta}=\dfrac{v^2}{\rho}$};
    \draw[line width=0.4pt] (10,-0.3) -- (11,-0.7) node[color=blue,right] {$m$};
\end{tikzpicture}

$$a=\sqrt{{a_{\tau}}^2 + {a_{\eta}}^2}$$\\

Voici les trois formules magiques pour un mouvement rectiligne uniformément varié:
$$a_{x}=cste $$
$$v_{x}=a_{x}\cdot t +v_{x0}$$
$$x=\frac{1}{2}a_{x}\cdot t^2 + v_{x0}\cdot t +x_{0} $$\\

Dans un mouvement circulaire de rayon $R$, avec $\omega = \dfrac{\Delta \theta}{\Delta t}$ étant la vitesse angulaire,\\

\begin{tikzpicture}
    \draw (0,0) node {};
    \draw (7.85,0) node {$V=\omega\cdot R$};
    \draw[line width=0.4pt] (7.7,-0.3) -- (7.2,-0.6) node[color=blue,left] {rad/s};
\end{tikzpicture}


La fréquence $f$ est définie
$$f(\text{Hz}) = \dfrac{1}{T(\text{s})} = \dfrac{\omega\, (\text{rad/s})}{2\pi(\text{rad})}$$
\end{Definition}


\section{Les 3 lois de Newton}

\subsection{$1^{ere}$ loi}

\begin{Definition}
Dans un référentiel galiléen, le centre d'inertie d'un solide isolé ou pseudo-isolé est animé d'un mouvement \textbf{rectiligne uniforme} (et réciproquement) :
$$\sum \vv{F}_{ext} = \vv{0} \qquad \Leftrightarrow \qquad \text{Mvt rect. uniforme}$$
\end{Definition}

\subsection{$2^{eme}$ loi}

\begin{Definition}
Dans un référentiel galiléen, le PFD prédit que
$$\sum \vv{F}_{ext} = m\cdot \vv{a} $$
\end{Definition}

\subsection{$3^{eme}$ loi}

\begin{Definition}
C'est le principe de l'action et de la réaction. Soient $A$ et $B$ deux centres d'inertie de deux objets dans un référentiel galiléen :
$$\vv{F}_{A\rightarrow B} = - \vv{F}_{B\rightarrow A}$$

\end{Definition}


\section{Énergies et TEC}

\begin{Definition}
L'énergie fait bouger des choses... Il existe plusieures formes d'énergie :
\begin{itemize}
\item $E_{c}$ L'énergie cinétique, Vitesse $v$ : $E_{c} = 1/2 \cdot m \cdot v^2 $
\item $E_{pp}$ L'énergie potentielle de pesanteur, Altitude $z$ : $E_{pp} = m \cdot g \cdot z$
\item $E_{pe}$ L'énergie potentielle élastique, Longueur déformée $x$ : $E_{pe} = 1/2 \cdot k \cdot x^2$
\item $E_{th}$ L'énergie thermique, Température $T$ 
\item $E_{c}$ L'énergie potentielle électrique : $E_{c} = 1/2 \cdot C \cdot U_{c}^2$
\item $E_{m}$ L'énergie potentielle magnétique : $E_{m} = 1/2 \cdot L \cdot i^2$
\item Les énergies physique, chimique et nucléaire, Masse des corps $m$\\\\
\end{itemize}

L'énergie peut cependant changer de forme par un travail, transfert d'énergie :
\begin{itemize}
\item travail mécanique $W_{m}$ (force) : $W_{A\rightarrow B}(\vv{F}) = W_{AB}(\vv{F}) = \vv{F}\cdot\vv{AB} = F \cdot AB \cdot \cos{\alpha} $
\item travail électrique $W_{e}$ (courant électrique) : $W_{A\rightarrow B}(\vv{F}_{e}) = q \cdot (V_{A}-V_{B})$
\item travail rayonnant $W_{r}$ (rayonnement)
\item chaleur $Q$ (chaleur)
\end{itemize}

On retient aussi :
$$W_{AB}(\vv{P}) = m\cdot g \cdot (z_{A}-z_{B})$$
$$W_{AB}(\vv{f}) = - f \cdot \widetilde{AB} $$
$$W_{AB}(\vv{R}_{n}) = 0$$
$$W_{AB}(\vv{F}_{R}) = 1/2 \cdot k \cdot ({x_{B}}^2 -{x_{A}}^2)$$\\\\

La \textbf{puissance moyenne} mesure la quantité d'énergie transférée par seconde :

\begin{tikzpicture}
    \draw (0,0) node {};
    \draw (7.85,0) node {$P_{m} (\vv{F})= \dfrac{W_{AB}(\vv{F})}{\Delta t}$};
    \draw[line width=0.4pt] (9.5,0.2) -- (10.4,0.35) node[color=blue,right] {$J$};
    \draw[line width=0.4pt] (9,-0.3) -- (10,-0.7) node[color=blue,right] {$s$};
\end{tikzpicture}

La \textbf{puissance instantanée}:
$$P(\vv{F}) = \vv{F}\cdot \vv{v} = F\cdot v \cdot \cos{\alpha}$$\\

Le \textbf{TEC} :
$$\Delta {E_{c}}_{A\rightarrow B} = \sum W_{AB}(\vv{F}_{ext})$$

\end{Definition}

\section{L'analyse dimensionnelle}

\begin{Definition}

Les différentes dimensions sont :
\begin{itemize}
\item Longueur $L$
\item Masse $M$
\item Durée $T$
\item Température $\theta$
\item Intensité électrique $I$
\item Intensité lumineuse $J$
\item Quantité de matière $N$
\end{itemize}
Il ne faut pas oublier les grandeurs sans dimensions, comme les angles ou les facteurs...
\end{Definition}


\section{Mouvements dans le champ de pesanteur uniforme}

\begin{Definition}

La force de frottement du fluide sur le corps peut être donnée par l'expression:
$$f=k\cdot v^n$$
où $k$ contient le coefficient de pénétration du corps et la nature du fluide et $n$ est réel. Ce sont des coefficients seulement déterminables par des expériences.\\\\
Pour des petites vitesses (quelques cm/s), $n=1$, la force de frottement est donc proportionnelle à la vitesse du corps. Dans le cas d'une sphère de rayon $R$ et $\eta$ la viscosité,
$$f=(6\pi\cdot\R\cdot\eta )\cdot v$$
Pour des vitesses plus grandes (quelques m/s), $n=2$ convient mieux. Si $S$ est la section, $C_{x}$ le coefficient de trainée, $\rho_{fluide}$ la masse volumique,
$$ f=\frac{1}{2}C_{x}\cdot\rho_{fluide}\cdot S\cdot v^2 $$  \\\\
Dans une chute non libre, le régime \textbf{permanent} est atteint quand $\vv{P}=-\vv{f}$, donc quand $a=\dfrac{dv}{dt}=0$.
\end{Definition}


\section{Mouvements de satellites et des planètes}

\begin{Definition}

L'accélération d'un satellite terrestre est égal au champ de gravitation:
$$\vv{a}=\vv{\mathcal{G}}$$
Des formules souvent retrouvées sont :

$$g_{0}\approx\mathcal{G}_{0}=\dfrac{G\cdot M_{T}}{{R_{T}}^2} \Leftrightarrow G\cdot M_{T} = g_{0}\cdot {R_{T}}^2$$

$$a=a_{n}=\dfrac{G\cdot M_{T}}{r^2} = \dfrac{v^2}{r} \Leftrightarrow v=\sqrt{\dfrac{G\cdot M_{T}}{r}} = \sqrt{\dfrac{g_{0}\cdot{R_{T}}^2}{(R_{T} + h) }}$$

$$V=\dfrac{d}{t}=\dfrac{2\pi r}{T} \Leftrightarrow T = \dfrac{2\pi r}{v} = \dfrac{2\pi r}{\sqrt{\dfrac{G\cdot M_{T}}{r}}} = 2\pi\cdot\sqrt{\dfrac{r^3}{G\cdot M_{T}}} = 2\pi\cdot\sqrt{\dfrac{(R_{T} + h)^3}{g_{0} \cdot {R_{T}}^2}} $$
\end{Definition}


\subsection{Les 3 lois de Kepler}

\begin{Definition}
\textbf{$1^{\text{ère}}$ loi :}\\
La trajectoire d'un astre (solaire) est une éllipse dont le soleil est un des foyers\\\\
\textbf{$2^{\text{ème}}$ lo :}\\
Si la planète met la mème durée pour aller de $A\rightarrow B$ que de $C\rightarrow D$, alors $\Big{\mathcal{A}}_{AMB}=\Big{\mathcal{A}}_{CMD}$\\\\
\textbf{$3^{\text{ème}}$ loi :}\\
$$\dfrac{T^2}{a^3}=cste$$
\end{Definition}



\section{Systèmes oscillants}

\begin{Definition}
A COMPLETER APRES AVOIT FAIT LES OSCILLATEURS ÉLECTRIQUES

La période propre d'un pendule simple, non amorti est :
$$T_{0}=2\pi\sqrt{\dfrac{l}{g}}$$
Bien que $T_{0}$ est aussi proportionnel à l'amplitude $\theta$, ce facteur peut être négligé dans nos calculs\\

La période propre d'un pendule élastique horizontal non amorti est :
$$T_{0}=2\pi\sqrt{\dfrac{m}{k}}$$

On remarque que la \textbf{pulsation propre} $\omega_{0}$ d'un pendule élastique horizontal non amorti est un peu comme la vitesse angulaire pour un oscillateur :
$$\omega_{0}=\dfrac{2\pi}{T_{0}}=\sqrt{\dfrac{k}{m}}$$

Dans le cas d'un système oscillant amorti, on distingue entre un régime \textbf{pseudo-périodique} et un régime \textbf{apériodique} (aucune oscillation)

\end{Definition}



\section{Oscillateurs mécaniques en régime forcé}

\begin{Definition}

La bande passante est l'intervalle des fréquences pour lesquelles le résonnateur donne une réponse importante en amplitude. Elle est déterminable sur le graphique de l'amplitude en fonction de la fréquence. On fait $\dfrac{X_{mres}}{\sqrt{2}}$ et on retrouve la largeur de la bande passante $\Delta f = f_{2} - f_{1}$. Il en résulte le facteur de qualité $Q$ :
$$Q=\dfrac{f_{res}}{\Delta f} \approx \dfrac{f_{0}}{f_{2}-f_{1}} $$

\end{Definition}



\section{Émission, propagation et réception des ondes}

\begin{Definition}

La célérité (vitesse de propagation) de l'onde est constante dans un milieu non dispersif.
$$v=\dfrac{d}{\Delta t} = \dfrac{\lambda}{T} = \lambda \cdot f$$

Une onde est périodique dans l'espace et dans le temps. Dans un moment donné, les points $P_{1}$ et $P_{2}$ du milieu (en gros, deux courbes d'onde) sont en phase si
$$d(P_{1};P_{2})=k\cdot\lambda\quad;k\in\Z$$
et ils sont en opposition de phase si
$$d(P_{1};P_{2})=(2k+1)\cdot\dfrac{\lambda}{2};\quadk\in\Z$$

L'indice de réfraction $n$ et la célérité $v_{\Phi}$ des ondes électromagnétiques sont proportionnels:
$$v_{\Phi} = \dfrac{c}{n}$$
\end{Definition}


\section{Diffraction des ondes}

\begin{Definition}

Il y a seulement diffraction si l'obstacle de l'onde a une dimension du même ordre de grandeur que celle-ci.\\
L'écart angulaire pour la diffraction par une fente est, avec $a$ largeur de la fente, $\theta$, mesuré entre le milieu de la tache centrale et la première extinction :
$$\theta=\dfrac{\lambda}{a}$$

\end{Definition}


\section{Particules chargées dans un champ électrique ou magnétique}

\begin{Definition}

La force magnétique s'exerçant sur une particule de charge $q$ est
$$\vv{F}_{m} = q \cdot (\vv{v}\times \vv{B})$$,
d'après la loi de Lorentz. On utilise la règle de la main droite pour le produit vectoriel. \\
Et dans la plupart des cas, $\alpha$ vaut $\ang{90}$
$$\Rightarrow F_{m} = |q| \cdot v \cdot B \cdot \sin(\alpha)$$

La trajectoire d'une particule chargée dans un champ magnétique est circulaire, lorsqu'on reste dans un plan. La trajectoire dans un champ électrique est une parabole. La trajectoire rectiligne suivant l'accélération dans $E$ passe par le milieu d'une des faces du condensateur.

\end{Definition}


\section{Action d'un champ magnétique sur un circuit parcouru par un courant}


\begin{Definition}
L'intensité instantanée dans un circuit électrique est définie par :
$$i = \diff{Q}{t}$$
Un conducteur rectiligne de longueur $l$, parcouru par un courant continu d'intensité $I$ et placé dans un champ magnétique uniforme $\vv{B} \quad$  ($\vv{l}: \text{pouce}, \vv{B}: \text{index}, \vv{F_{l}}: \text{majeur} $) :
$$\vv{F_{l}} = I \cdot (\vv{l} \times \vv{B})$$

Le flux magnétique dans une bobine a l'unité Weber (\textcolor{blue}{Wb}) : 
$$ \Phi = N \cdot \vv{B} \cdot \vv{S} = N \cdot B \cdot S \cdot \cos{\alpha}$$

La fem d'induction suite a une variation du flux :
$$e=-\diff{\Phi}{t}$$
$$\Rightarrow i = \dfrac{e}{R} = - \dfrac{1}{R} \cdot \diff{\Phi}{t}$$
\end{Definition}




\section{Dipôles dans un circuit}

\subsection{Lois générales de dipôles}
\begin{Definition}
La loi d'Ohm généralisée, avec $e$, force électromotrice éventuelle
$$U_{AB} = R_{AB} \cdot I_{AB} - e_{AB}$$

La puissance électrique instantanée aux bornes d'un dipôle AB est exprimée en Watt (\textcolor{blue}{W}) :
$$\mathcal{P} = U_{AB} \cdot i_{AB} $$

La puissance électrique instantanée (la bobine reçoit un travail électrique W) :
$$P = \diff{W}{t}$$

\end{Definition}

\subsection{Le dipôle (R,L)}
\begin{Definition}
La Loi de Faraday-Lenz, avec $L$, inductance de la Bobine (Henri \textcolor{blue}{H})
$$e = - L \cdot \diff{i}{t}$$

Le flux propre appelé flux d'autoinduction de la bobine (\textcolor{blue}{WB})
$$\Phi = L \cdot i$$

La constante de temps $\tau$ : 
$$\tau = \dfrac{L}{R} $$

La tension aux bornes d'une Bobine avec une résistance propre $r$ si $i_{A\rightarrow B}$
$$u_{AB} = r \cdot i + L \cdot \diff{i}{t}$$

L'équation différentielle à établir aura toujours des $+$
$$U_{G} = R_{T} \cdot i + L \cdot \diff{i}{t}$$

L'énergie (potentielle) magnétique emmagasinée par une bobine est :
$$E_{m} = \dfrac{1}{2} \cdot L \cdot i^2$$
\end{Definition}



\subsection{Le dipôle (R,C)}
\begin{Definition}
La capacité du condensateur $C$ (Farad \textcolor{blue}{F}) est sa capacité á acquérir une certaine charge :
$$q = C \cdot U_{c}$$

La capacité $C$ en fonction de l'aire de la surface des armatures $A$, de la distance entre les armatures $d$ et la permittivité absolue de l'isolant $\varepsilon$. $\varepsilon_{0}$ est la permittivité du vide et $\varepsilon_{r}$ la permittivité relative ou constante diéléctrique du matériau utilisé :
$$C = \dfrac{\varepsilon_{0} \cdot \varepsilon_{r} \cdot A}{d}$$

Lors de l'association en série, l'inverse des capacités est additionné :
$$\dfrac{1}{C_{eq}} = \dfrac{1}{C_{1}} + \dfrac{1}{C_{2}} + ... + \dfrac{1}{C_{n}}$$

Branchés en parallèle, les capacités des condensateurs sont additionnées.\\\\

La constante de temps $\tau$ : \qquad $\tau = R \cdot C$\\

L'intensité aux bornes du condensateur lors de la charge et de la décharge : 
$$i = C \cdot \diff{U_{C}}{t} = C \cdot \diff{U_{AB}}{t}$$

L'équation différentielle à établir aura toujours des $+$ 
$$U_{G}  = \diff{q}{t} + \dfrac{q}{RC} $$

L'énergie (potentielle) électrique emmagasinée par le condensateur initialement déchargé est :
$$E_{c} = \dfrac{1}{2} \cdot C \cdot U_{c}^2$$

\end{Definition}


\subsection{Le dipôle (R,L,C) et les oscillations électriques}
\begin{Definition}
On rappelle la pulsation propre $\omega_{0} = \dfrac{2\pi}{T_{0}} $\\\\

La période propre $T_{0}$ des oscillations électriques est :
$$T_{0} = 2 \pi \sqrt{LC}$$
$$\Rightarrow \omega_{0} = \dfrac{1}{\sqrt{LC}}$$

\end{Definition}



\end{document}
